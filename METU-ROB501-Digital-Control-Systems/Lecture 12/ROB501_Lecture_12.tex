%
% This is a borrowed LaTeX template file for lecture notes for CS267,
% Applications of Parallel Computing, UCBerkeley EECS Department.
% Now being used for CMU's 10725 Fall 2012 Optimization course
% taught by Geoff Gordon and Ryan Tibshirani.  When preparing 
% LaTeX notes for this class, please use this template.
%
% To familiarize yourself with this template, the body contains
% some examples of its use.  Look them over.  Then you can
% run LaTeX on this file.  After you have LaTeXed this file then
% you can look over the result either by printing it out with
% dvips or using xdvi. "pdflatex template.tex" should also work.
%

\documentclass[twoside]{article}
\setlength{\oddsidemargin}{0.25 in}
\setlength{\evensidemargin}{-0.25 in}
\setlength{\topmargin}{-0.6 in}
\setlength{\textwidth}{6.5 in}
\setlength{\textheight}{8.5 in}
\setlength{\headsep}{0.75 in}
\setlength{\parindent}{0 in}
\setlength{\parskip}{0.1 in}

%
% ADD PACKAGES here:
%

\usepackage{amsmath,amsfonts,graphicx}

%
% The following commands set up the lecnum (lecture number)
% counter and make various numbering schemes work relative
% to the lecture number.
%
\newcounter{lecnum}
\renewcommand{\thepage}{\thelecnum-\arabic{page}}
\renewcommand{\thesection}{\thelecnum.\arabic{section}}
\renewcommand{\theequation}{\thelecnum.\arabic{equation}}
\renewcommand{\thefigure}{\thelecnum.\arabic{figure}}
\renewcommand{\thetable}{\thelecnum.\arabic{table}}

%
% The following macro is used to generate the header.
%
\newcommand{\lecture}[4]{
   \pagestyle{myheadings}
   \thispagestyle{plain}
   \newpage
   \setcounter{lecnum}{#1}
   \setcounter{page}{1}
   \noindent
   \begin{center}
   \framebox{
      \vbox{\vspace{2mm}
    \hbox to 6.28in { {\bf ROB501 - Fundamentals \& Emerging Topics in Robotics - Digital Control Systems } }
       \vspace{4mm}
       \hbox to 6.28in { {\Large \hfill Lecture #1 \hfill} }
       \vspace{2mm}
       \hbox to 6.28in { {\it Lecturer: #2 \hfill } }
      \vspace{2mm}}
   }
   \end{center}
   \markboth{Lecture #1}{Lecture #1}

   \vspace*{4mm}
}
%
% Convention for citations is authors' initials followed by the year.
% For example, to cite a paper by Leighton and Maggs you would type
% \cite{LM89}, and to cite a paper by Strassen you would type \cite{S69}.
% (To avoid bibliography problems, for now we redefine the \cite command.)
% Also commands that create a suitable format for the reference list.
\renewcommand{\cite}[1]{[#1]}
\def\beginrefs{\begin{list}%
        {[\arabic{equation}]}{\usecounter{equation}
         \setlength{\leftmargin}{2.0truecm}\setlength{\labelsep}{0.4truecm}%
         \setlength{\labelwidth}{1.6truecm}}}
\def\endrefs{\end{list}}
\def\bibentry#1{\item[\hbox{[#1]}]}

%Use this command for a figure; it puts a figure in wherever you want it.
%usage: \fig{NUMBER}{SPACE-IN-INCHES}{CAPTION}
\newcommand{\fig}[3]{
			\vspace{#2}
			\begin{center}
			Figure \thelecnum.#1:~#3
			\end{center}
	}
% Use these for theorems, lemmas, proofs, etc.
\newtheorem{theorem}{Theorem}[lecnum]
\newtheorem{lemma}[theorem]{Lemma}
\newtheorem{proposition}[theorem]{Proposition}
\newtheorem{claim}[theorem]{Claim}
\newtheorem{corollary}[theorem]{Corollary}
\newtheorem{definition}[theorem]{Definition}
\newenvironment{proof}{{\bf Proof:}}{\hfill\rule{2mm}{2mm}}

% **** IF YOU WANT TO DEFINE ADDITIONAL MACROS FOR YOURSELF, PUT THEM HERE:

\begin{document}

% Lecture Details
\lecture{12}{Asst. Prof. M. Mert Ankarali}



\section{Discrete-time Luenberger Observer}

In general the state, $x[k]$, of a system
is not accessible and \textit{observers, estimators, filters})
have to be used to extract this information.
The output, $y[k]$, represents the measurements
which is a function of $x[k]$ and $u[k]$.
%
\begin{align*}
  x[k+1] &= G x[k] + H u[k]
  \\
  y[k] &= C x[k] 
\end{align*}
%
A Luenberger observers is built using a ``simulated'' model of the 
system and the errors caused by the mismatched initial conditions 
$x_0 \neq \hat{x}_0$ (or other types of perturbations)
are reduced by introducing output error feedback.

Let's assume that the state vector of the simulated system
is$\hat{x}[k]$, then the state space equation of this
synthetic system takes the form
%
\begin{align*}
  \hat{x}[k+1] &= G \hat{x}[k] + H u[k]
  \\
  \hat{y}[k] &= C \hat{x}[k] 
\end{align*}
%
Note that since $u[k]$ is the input that is supplied by the 
controller, we assume that it is known apriori. If $x[0] = \hat{x}[0]$ and
when there is no model mismatch or uncertainty in the system
then we expect that $x[k] = \hat{x}[k]$ and $y[k] = \hat{y}[k]$ 
for all $k$. When $x[0] \neq \hat{x}[0]$, then we should observe a 
difference between the measured and predicted output
$y[k] \neq \hat{y}[k]$. The core idea in Luenberger observer
is feeding the error in the output prediction 
$y[k] - \hat{y}[k]$ to the simulated system via a linear feedback gain.
%
\begin{align*}
  \hat{x}[k+1] &= G \hat{x}[k] + H u[k] + L \left( y[k] - \hat{y}[k] \right) 
  \\
  \hat{y}[k] &= C \hat{x}[k] 
\end{align*}
%
In order to understand how a Luenberger observer works and
to choose a proper observer gain $L$, we define an error signal
$e[k] = x[k] - \hat{x}[k]$. The dynamics w.r.t $e[k]$ can be derived
as
%
\begin{align*}
  e[k+1] &= x[k+1] - \hat{x}[k+1]
        \\
     &= \left( G x[k] + H u[k] \right)
  - \left( G \hat{x}[k] + H u[k] + L \left( y[k] - \hat{y}[k] \right)
       \right)
\\
   e[k+1] &= \left( G - L C \right) e[k]
\end{align*}
%
where $e[0] = x[0] - \hat{x}[0]$ denotes the error in the initial
condition. 

If the matrix $\left( G - L C \right)$ is stable then the errors in
initial condition will diminish eventually. Moreover, in order
to have a good observer/estimator performance the observer
convergence should be sufficiently fast. 

\newpage

\subsection{Observer Gain \& Pole Placement}

Similar to the state-feedback gain design,
the fundamental principle of ``pole-placement'' Observer design is that
we first define a desired closed-loop eigenvalue set and 
compute the associated desired characteristic polynomial. 
%
\begin{align*}
 \mathcal{E}^* &= \lbrace \lambda_1^* , \ \cdots, \  \lambda_n^*
                 \rbrace
  \\
  p^*(z) &= \left( z - \lambda_1^* \right) \cdots \left( z - \lambda_n^*
         \right)                         
  \\
  &= z^n + a_1^* z^{n-1} + \cdots + a_{n-1}^* z + a_n^*
\end{align*}
%
The necessary and sufficient condition on arbitrary observer pole-placement
is that the system should be fully Observable. Then, we tune $L$ such
that 
%
\begin{align*}
  \mathrm{det} \left( z I - ( G - L C ) \right) = p^*(z)
\end{align*}
%
\subsection*{Direct Design of Observer Gain}

If $n$ is small, the most efficient method could be the direct
design. 

\textbf{Example:} Consider the following DT system
%
\begin{align*}
 x[k+1] &= \left[ \begin{array}{cc} 1 & 0 \\ 0 & 2 \end{array} \right] x[k]
    + \left[ \begin{array}{c} 1 \\ 1 \end{array} \right] u[k]
\\
 y[k] &= \left[ \begin{array}{cc} 1 & -1 \end{array} \right] u[k]
\end{align*}
% 
Design an observer such that estimater poles are located at 
$\lambda_{1,2} = 0$ (Dead-beat Observer)

\textbf{Solution:} Desired characteristic equation can be computed as
%
\begin{align*}
  p^*(z) = z^2
\end{align*}
%
Let $L = \left[ \begin{array}{c} l_2 \\ l_1 \end{array} \right]$, then
the characteristic equation of $(G - L C)$ can be computed as
%
\begin{align*}
  \mathrm{det} \left( z I - ( G - L C ) \right) &= 
  \mathrm{det} \left(
  \left[ \begin{array}{cc} z - 1 + l_2 & -l_2 \\ l_1 & z - 2 - l_1 \end{array} \right]
  \right)
\\
&= z^2 + z (l_2 - l_1 - 3) + (l_1 - 2 l_2 + 2)
\end{align*}
%
If we match the equations
%
\begin{align*}
  l_2 - l_1 &= 3
\\
  -l_1 + 2 l_2 &= 2
\\
 l_2 &= -1
\\
 l_1 &= -4
\end{align*}
%
Thus $L = \left[ \begin{array}{c} -1 \\ -4 \end{array} \right]$

\newpage

\section{Closed-Loop Observer \& State-Feedback}

In the state-feedback control policy the input is ideally defined
by the following law
%
\begin{align*}
 u[k] = - K x[k]
\end{align*}
%
However, as mentioned in Observer lecture, in general we don't have
direct access to the all states of the system. In this case, we learnt
how to design an Observer/Estimator of the states. In this respect,
it is natural to assume that in a closed-loop system, the control
policy that define the input should depend on the estimated states
%
\begin{align*}
 u[k] = - K \hat{x}[k]
\end{align*}
%
However the important question how this coupling affect the
closed-loop behavior, and even deeper question can be even 
use such a policy. The advantage of LTI systems is that 
state-feedback gain, and observer gain can be seperatelly
designed and we guarntee a stable closed-loop performance. 
In this section, we will analyze the coupled system
%
Equations of motion for the closed-loop observer \& state-feedback
based control system is given below
%
\begin{align*}
   x[k+1] &= G x[k] + H u[k]
  \\ 
   y[k] &= C x[k] 
  \\
  \hat{x}[k+1] &= G \hat{x}[k] + H u[k] + L \left( y[k] - \hat{y}[k] \right) 
  \\
  \hat{y}[k] &= C \hat{x}[k] 
  \\
   u[k] &= -K \hat{x}[k]
\end{align*}
%
If we eliminate $u[k]$ and $\hat{y}[k]$ we obtain following 
dynamical representation
%
\begin{align*}
   x[k+1] &= G x[k] - H K \hat{x}[k] 
  \\
  \hat{x}[k+1] &= G \hat{x}[k] - H K \hat{x}[k] + L C \left( x[k] -
                 \hat{x}[k] \right) 
  \\ 
   y[k] &= C x[k] - D K \hat{x}[k] 
\end{align*}
%
Now let's replace $\hat{x}[k]$ with $e[k] = x[k] - \hat{x}[k]$
%
\begin{align*}
   x[k+1] &= ( G - H K ) x[k] + H K e[k]
  \\
  e[k+1] &= ( G - LC ) e[k]
  \\ 
   y[k] &= (C - D K) x[k] + D K e[k]
\end{align*}
%
Now let's defina a state for the whole system, 
$z[k] = \left[ \begin{array}{c} x[k] \\ e[k] \end{array} \right]$
then the state-space representation is given by
%
\begin{align*}
  z[k+1] = \left[ \begin{array}{cc} (G - H K) & H K\\ 0_{n \times n}
                                              & (G - LC) \end{array}
                                                \right] z[k]
\\
 y[k] = \left[ \begin{array}{cc} (C - DK) & D K \end{array}
                                                \right] z[k]
\end{align*}
%
The system matrix is in block diagonal form and the eigenvalues
of this new system matrix is find by taking the union of eigenvalues
of $(G - H K)$ and eigenvalues of $(G - L C)$. Thus a seperate
pole-placement can be performed for the state-feedback controller
and the observer. 



% **** This ENDS THE EXAMPLES. DON'T DELETE THE FOLLOWING LINE:
\end{document}
