% 


\documentclass[twoside]{article}
\setlength{\oddsidemargin}{0.25 in}
\setlength{\evensidemargin}{-0.25 in}
\setlength{\topmargin}{-0.6 in}
\setlength{\textwidth}{6.5 in}
\setlength{\textheight}{8.5 in}
\setlength{\headsep}{0.75 in}
\setlength{\parindent}{0 in}
\setlength{\parskip}{0.1 in}

%
% ADD PACKAGES here:
%

\usepackage{amsmath,amsfonts,graphicx,mathdots}


\newcounter{lecnum}
\renewcommand{\thepage}{\thelecnum-\arabic{page}}
\renewcommand{\thesection}{\thelecnum.\arabic{section}}
\renewcommand{\theequation}{\thelecnum.\arabic{equation}}
\renewcommand{\thefigure}{\thelecnum.\arabic{figure}}
\renewcommand{\thetable}{\thelecnum.\arabic{table}}

%
% The following macro is used to generate the header.
%
\newcommand{\lecture}[4]{
   \pagestyle{myheadings}
   \thispagestyle{plain}
   \newpage
   \setcounter{lecnum}{#1}
   \setcounter{page}{1}
   \noindent
   \begin{center}
   \framebox{
      \vbox{\vspace{2mm}
    \hbox to 6.28in { {\bf EE302 - Discrete Time Systems
	\hfill Spring 2019} }
       \vspace{4mm}
       \hbox to 6.28in { {\Large \hfill Lecture #1 \hfill} }
       \vspace{2mm}
       \hbox to 6.28in { {\it Lecturer: #2 \hfill } }
      \vspace{2mm}}
   }
   \end{center}
   \markboth{Lecture #1}{Lecture #1}

   \vspace*{4mm}
}

\renewcommand{\cite}[1]{[#1]}
\def\beginrefs{\begin{list}%
        {[\arabic{equation}]}{\usecounter{equation}
         \setlength{\leftmargin}{2.0truecm}\setlength{\labelsep}{0.4truecm}%
         \setlength{\labelwidth}{1.6truecm}}}
\def\endrefs{\end{list}}
\def\bibentry#1{\item[\hbox{[#1]}]}


\newcommand{\fig}[3]{
			\vspace{#2}
			\begin{center}
			Figure \thelecnum.#1:~#3
			\end{center}
	}

% Use these for theorems, lemmas, proofs, etc.
\newtheorem{theorem}{Theorem}[lecnum]
\newtheorem{lemma}[theorem]{Lemma}
\newtheorem{proposition}[theorem]{Proposition}
\newtheorem{claim}[theorem]{Claim}
\newtheorem{corollary}[theorem]{Corollary}
\newtheorem{definition}[theorem]{Definition}
\newenvironment{proof}{{\bf Proof:}}{\hfill\rule{2mm}{2mm}}

% **** IF YOU WANT TO DEFINE ADDITIONAL MACROS FOR YOURSELF, PUT THEM HERE:

\begin{document}

% Lecture Details
\lecture{21}{Asst. Prof. M. Mert Ankarali}

\par

\section{Reachability/Controllability}

For a LTI continuous time state-space representation
%
\begin{align*}
  \dot{x}(t) &= A x(t) + B u(t)
\\
  y(t) &= C x(t) + D u(t)
\end{align*}
%
\begin{itemize}
  \item A state $x_d$ is said to be \textbf{reachable}
  if there exist a finite time interval $t \in [0 ,t_f]$ and
  an input signal defined on this interval, $u(t)$, that transfers the state vector 
  $x(t)$ from the origin (i.e. $x(0) = 0$) to the state $x_d$ within
  this time interval, i.e. $x(t_f) = x_d$.
 
  \item A state $x_d$ is said ti be \textbf{controllable}
  if there exist a finite time interval $t \in [0 ,t_f]$ and
  an input signal defined on this interval, $u(t)$, that transfers the state vector 
  $x(t)$ from the initial state $x_d$ (i.e. $x(0) = x_d$) to the origin
  this time interval, i.e. $x(t_f) = 0$.  
\end{itemize}

\begin{itemize}
  \item The set $\mathcal{R}$ of all reachable states is a linear
(sub)space: $\mathcal{R} \subset \mathbb{R}^n$
  \item The set $\mathcal{C}$ of all controllable states is a linear
(sub)space: $\mathcal{C} \subset \mathbb{R}^n$
\end{itemize}

For CT systems $x_d \in \mathcal{R}$ if and only if  $x_d \in
\mathcal{C}$, the Reachability and Controllability conditions are 
equivalent. 

\begin{itemize}
  \item If the reachable (or controllable) set is the entire state
    space, i.e., if $\mathcal{R} = \mathbb{R}^n$, then the system is
    called fully reachable/controllable.
\end{itemize}

One way of testing reachability/controllability is checking the rank
(or the range space) the of reachability/controllability matrix
%
\begin{align*}
  Q = \left[ \begin{array}{cccc} B & A B & \cdots & A^{n-1} B \end{array} \right]
\end{align*} 
%

A CT system is reachable/controllable if and only of
%
\begin{align*}
  \mathrm{rank} ( Q ) &= n
\end{align*} 
%
or equivalently 
%
\begin{align*}
 \mathrm{Ra} ( \mathbf{M} ) = \mathbb{R}^n
\end{align*} 

%%%%%%%%%%%%%%%%%%%%%%%%%%

\textbf{Example:} Consider the following system
%
\begin{align*}
  \dot{x} &= \left[ \begin{array}{cc} 0 & 1 \\ 0 & 0 \end{array}
                                                   \right] x + B u 
\end{align*}
%
Analyze the controllability for two different input, $B$,
matrices
%
\begin{align*}
 B_1 &= \left[ \begin{array}{c} 1 \\ 0 \end{array} \right] 
  \\
 B_2 &= \left[ \begin{array}{c} 0 \\ 1 \end{array} \right] 
\end{align*}
%

\textbf{Solution:}

Let's start with $B_1$ and derive the controllability matrix
%
\begin{align*}
 Q_1 &= \left[ \begin{array}{cc} B_1 & A B_1 \end{array}
                                               \right]
= \left[ \begin{array}{cc} 1 & 0 \\ 0 & 0 \end{array} \right]
\\
\mathrm{det} ( Q_1) &= 0 
\end{align*}
% 
Thus the system is not fully controllable. Now let's get some
intuition by writing the ODEs for individual states $x_1$ and $x_2$,
where $x = \left[ \begin{array}{c} x_1 \\ x_2 \end{array} \right] $.
%
\begin{align*}
  \dot{x_1} &= x_2 + u
              \\
  \dot{x_2} &= 0
\end{align*}
%
Let's go with reachability definition, which is based on starting
from zero initial conditions and going to a desired state,
we can derive the following relations
%
\begin{align*}
  x_1(T) &= \int\limits_{0}^T u(t) d t
              \\
  x_2(T) &= 0
\end{align*}
%
Neither the input nor the first state has an affect on
the second state, so $x_2 = \alpha$ for $\alpha \neq 0$
is not controllable (or reachable). However it is also easy to 
see that we can always find $u(t)$ that will drive the $x_1$ to a
desired state, $x_1^*(T)$.

Now let's analyze the case with $B_2$ 
%
\begin{align*}
 Q_2 &= \left[ \begin{array}{cc} B_2 & A B_2 \end{array}
                                               \right]
= \left[ \begin{array}{cc} 0 & 1 \\ 1 & 0 \end{array} \right]
\\
\mathrm{det} ( Q_2) &= -1 \neq 0
\end{align*}
% 
Now the system is fully controllable. 

\vspace{6pt}

\textbf{Remark:} A state-space representation that is
in controllable canonical form is always fully controllable.

\newpage

\section{State-Feedback \& Pole Placement}

Given a continuous-time state-evolution equation
%
\begin{align*}
  \dot{x}(t) = A x(t) + B u(t)
\end{align*}
%
If direct measurements of all of the states of the
system (e.g. $y(t) = x(t)$) are available, one of the most
popular and powerful control method is the linear state feedback
control,
%
\begin{align*}
 u(t) = - K x(t)
\end{align*}
%
or when there is a reference input 
%
\begin{align*}
 u(t) = r(t) - K x(t)
\end{align*}
%
which can be thought as a generalization of P controller
to the vector form. Under this control law, state-equations
for the closed-loop system takes the form
%
\begin{align*}
  \dot{x}(t) &= A x(t) + B \left( r(t) - K x(t) \right)
  \\
  \dot{x}(t) &= \left( A - B K \right) x(t) + B r(t) 
\end{align*}
%
The system matrix of this autonomous system is
$\hat{A} = A - B K$. 
%
Important questions is how to choose $K$. Note that 
%
\begin{align*}
  K &\in \mathbb{R}^n \quad \mathrm{Single-Input} 
\end{align*}
%
As in all of the control design techniques, the most critical criterion
is stability, thus we want all of the eigenvalues to be in the
open-left-half s-plane. However, we know that there could be 
different requirements on the poles/eigenvalues of the system.

The fundamental principle of ``pole-placement'' design is that
we first define a desired closed-loop eigenvalue set 
$\mathcal{E}^* = \lbrace \lambda_1^* , \ \cdots, \  \lambda_1^* \rbrace$, and
then if possible we choose $K^*$ such that the closed-loop
eigenvalues match the desired ones.

The necessary and sufficient condition on arbitrary pole-placement
is that the system should be fully Controllable/Reachable.

In Pole-Placement, first step is computing the desired 
characteristic polynomial. 
%
\begin{align*}
 \mathcal{E}^* &= \lbrace \lambda_1^* , \ \cdots, \  \lambda_n^*
                 \rbrace
  \\
  p^*(s) &= \left( s - \lambda_1^* \right) \cdots \left( s - \lambda_n^*
         \right)                         
  \\
  &= s^n + a_1^* s^{n-1} + \cdots + a_{n-1}^* s + a_n^*
\end{align*}
%
Then we tune $K$ such that 
\begin{align*}
  \mathrm{det} \left( s I - ( A - B K ) \right) = p^*(s)
\end{align*}

\subsection{Direct Design of State-Feedback Gain}

If $n$ is small, the most efficient method could be the direct
design. 

\textbf{Example:} Consider the following system
%
\begin{align*}
 \dot{x} &= \left[ \begin{array}{cc} 1 & 0 \\ 0 & 2 \end{array} \right] x
    + \left[ \begin{array}{c} 1 \\ 1 \end{array} \right] u
\end{align*}
% 
Design a state-feedback rule such that poles are located at 
$\lambda_{1,2} = -1$ 

\textbf{Solution:} Desired characteristic equation can be computed as
%
\begin{align*}
  p^*(s) = s^2 + 2 s + 1
\end{align*}
%
Let $K = \left[ \begin{array}{cc} k_1 & k_2 \end{array} \right]$, then
the characteristic equation of $\hat{A}$ can be computed as
%
\begin{align*}
  \mathrm{det} \left( s I - ( A - B K ) \right) &= 
  \mathrm{det} \left(
  \left[ \begin{array}{cc} s - 1 + k_1 & k_1 \\ k_2 & s - 2 + k_2 \end{array} \right]
  \right)
\\
&= s^2 + s (k_1 + k_2 - 3) + (2 -2 k_1 - k_2)
\end{align*}
%
If we match the equations
%
\begin{align*}
s^2 + s (k_1 + k_2 - 3) + (2 -2 k_1 - k_2) &= s^2 + 2 s + 1
\\
k_1 + k_2 &= 5
\\
2k_1 + k_2 &= 1
\\
k_1 &= -4
\\
k_2 &= 9
\end{align*}
%
Thus $K = \left[ \begin{array}{cc} -4 & 9 \end{array} \right]$.

\subsection{Design of State-Feedback Gain Using Controllable Canonical
Form}

Let's assume that the state-space representation is in controllable
canonical form and we have access to the all states of this form
%
\begin{align*}
\dot{x} &= \left[ \begin{array}{ccccc} 0 & 1 & 0 & \cdots & 0 \\ 0 & 0 & 1 &
                                                                      \cdots & 0
\\ \vdots & \vdots & \vdots & & \vdots
\\ 0 & 0 & 0 & \cdots & 1
    \\ -a_n & -a_{n-1} & -a_{n-2} & \cdots & -a_1 \end{array} \right] x +
\left[ \begin{array}{c} 0\\ 0 \\ \vdots \\ 0
    \\ 1 \end{array} \right] u
\end{align*}
%
Let $K = \left[ \begin{array}{ccc} k_n & \cdots & k_1 \end{array} \right]$, then
closed-loop system takes the form
%
\begin{align*}
\dot{x} &= \left[ \begin{array}{ccccc} 0 & 1 & 0 & \cdots & 0 \\ 0 & 0 & 1 &
                                                                      \cdots & 0
\\ \vdots & \vdots & \vdots & & \vdots
\\ 0 & 0 & 0 & \cdots & 1
    \\ -a_n & -a_{n-1} & -a_{n-2} & \cdots & -a_1 \end{array} \right] x 
+ \left[ \begin{array}{c} 0\\ 0 \\ \vdots \\ 0 \\ 1 \end{array}
  \right] 
 \left( r - 
\left[ \begin{array}{ccc} k_n & \cdots & k_1 \end{array} \right]
x \right)
\\
\dot{x} &= \left[ \begin{array}{ccccc} 0 & 1 & 0 & \cdots & 0 \\ 0 & 0 & 1 &
                                                                      \cdots & 0
\\ \vdots & \vdots & \vdots & & \vdots
\\ 0 & 0 & 0 & \cdots & 1
    \\ -(a_n+k_n) & -(a_{n-1} + k_{n-1}) & -(a_{n-2} + k_{n-2}) &
                                                                  \cdots
                                                         & -(a_1 +
                                                           k_1) \end{array}
                                                           \right] x
+ \left[ \begin{array}{c} 0\\ 0 \\ \vdots \\ 0 \\ 1 \end{array}
  \right] r
\end{align*}
%
Let $p^*(s) = s^2 + a_1^* s + \cdots + a_{n-1}^* s + a_n^*$, then
computation of $K$ is super straight-forward
%
\begin{align*}
  K = \left[ \begin{array}{ccc} (a^*_n - a_n) & \cdots & (a^*_1 - a_1) \end{array} \right]
\end{align*}

However, what if the system is not in controllable canonical form. We can
find a transformation which finds the controllable canonical
representation. 

The controllability matrix of a state-space representation is given as
%
\begin{align*}
  Q = \left[ \begin{array}{c|c|c|c} B & A B & \cdots & A^{n-1} B\end{array} \right]
\end{align*}
%
Let's define a transformation matrix $T$ as follows:
%
\begin{align*}
  T &= Q W \quad , \quad x(t) = T \hat{x}(t)
\\
 \dot{\hat{x}} &= \left[ T^{-1} A T \right] \hat{x} + T^{-1} B u
\end{align*}
%
where
%
\begin{align*}
  W = \left[ \begin{array}{ccccc} a_{n-1} & a_{n-2} & \cdots & a_1 & 1
               \\ 
a_{n-2} & a_{n-3} & \cdots & 1 & 0
\\ \vdots & \vdots & \iddots & & \vdots
\\ a_1 & 1 &  & & 
    \\ 1 & 0 & \cdots &  & 0 \end{array} \right] 
\end{align*}
%
Then it is given that
%
\begin{align*}
  T^{-1} A T &= \hat{A} =  \left[ \begin{array}{ccccc} 0 & 1 & 0 & \cdots & 0 \\ 0 & 0 & 1 &
                                                                      \cdots & 0
\\ \vdots & \vdots & \vdots & & \vdots
\\ 0 & 0 & 0 & \cdots & 1
    \\ -a_n & -a_{n-1} & -a_{n-2} & \cdots & -a_1 \end{array} \right]
\\
  T^{-1} B &= \hat{B} = \left[ \begin{array}{c} 0\\ 0 \\ \vdots \\ 0
    \\ 1 \end{array} \right]
\end{align*}
%
We know how to design a
state-feedback gain $\hat{K}$ for the controllable
canonical form. Given $\hat{K}$ 
$u(t)$ is given as
%
\begin{align*}
  u(t) &= r(t) -\hat{K} \hat{x}(t)
\\ 
&= r(t) - \hat{K} T^{-1} \hat{x}(t)
\\
K &= \hat{K} T^{-1}
\end{align*}

\textbf{Example 2:} Consider the following system
%
\begin{align*}
 \dot{x} &= \left[ \begin{array}{cc} 1 & 0 \\ 0 & 2 \end{array} \right] x
    + \left[ \begin{array}{c} 1 \\ 1 \end{array} \right] u
\end{align*}
% 
Design a state-feedback rule using the controllable canonical form
approach, such that poles are located at 
$\lambda_{1,2} = -1$ 

\textbf{Solution:} Characteristic equation of $A$ can be derived as
%
\begin{align*}
 \mathrm{det} \left(  \left[ \begin{array}{cc} s-1 & 0 \\ 0 &
                                                              s-2 \end{array} \right]  \right)
 = s^2 - 3 s + 2
\end{align*}
%
The controllability matrix can be computed as
%
\begin{align*}
  Q = \left[ \begin{array}{c|c} B & A B \end{array} \right] = \left[ \begin{array}{cc} 1 & 1 \\ 1 & 2 \end{array} \right]
\end{align*}
%
The matrix $W$ can be computed as
%
\begin{align*}
  W =  \left[ \begin{array}{cc} -3 & 1 \\ 1 & 0 \end{array} \right]
\end{align*}
%
Transformation matrix, $T$ and its inverse $T^{-1}$ can be computed 
as
\begin{align*}
  T &= Q W =  
\left[ \begin{array}{cc} 1 & 1 \\ 1 & 2 \end{array} \right]
\left[ \begin{array}{cc} -3 & 1 \\ 1 & 0 \end{array} \right]
= \left[ \begin{array}{cc} -2 & 1 \\ -1 & 1 \end{array} \right]
\\
T^{-1} &= \left[ \begin{array}{cc} -1 & 1 \\ -1 & 2 \end{array} \right]
\end{align*}
%
Given that desired characteristic polynomial is $p^*(s) = s^2 + 2 s + 1$,
$\hat{K}$ of controllable canonical from can be computed as
%
\begin{align*}
  \hat{K} &= \left[ \begin{array}{cc} - a_2 & -
                                                       a_1 \end{array}
                                                       \right]
\\
&= \left[ \begin{array}{ccc} 1-2 & 2-(-3) \end{array}
                                                       \right]
= \left[ \begin{array}{ccc} -1 & 5 \end{array}
                                                       \right]
\end{align*}
%
Finally $K$ can be computed as
%
\begin{align*}
  K &= \hat{K} T^{-1} = \left[ \begin{array}{ccc} -1 & 5 \end{array}
                                                       \right]
\left[ \begin{array}{cc} -1 & 1 \\ -1 & 2 \end{array} \right]
\\
&= \left[ \begin{array}{ccc} -4 & 9 \end{array}
                                                       \right]
\end{align*}
%
As expected this is the same result with the one found with Direct-Method.

\newpage

\section{Observability}

It turns out that it is more natural to think in terms of
``un-observability'' as reflected in the following definitions.

\begin{itemize}
  \item A state $x_o$ of a finite dimensional CT linear dynamical
    system is said to be unobservable, if with $x(0) = x_o$ and
    for every $u(t)$ we get the same $y(t)$ as we would with $x(0) =
    0$. 
\end{itemize}

  The set $\bar{\mathcal{O}}$ of all unobservable states is a linear
  (sub)space: $\bar{\mathcal{O}} \subset \mathbb{R}^n$

\begin{itemize}
%
 \item If the unobservable set only contains the origin, 
   i.e., if $\bar{\mathcal{O}} = \lbrace 0 \rbrace$, 
%
 \item If the dimension of unobservable subspace is equal to 0,
   $\mathrm{dim} = \left( \bar{\mathcal{O}} \right) = 0$,
  %
 \item If any initial condition, $x(0)$ or $x[0]$, can be uniquely 
 determined from input-output measurement,
%
\end{itemize} 
then the system is called Observable. 

\subsubsection{Test of Observability on CT Systems}

One way of testing Observability of CT systems is checking the rank
(or the range space, or null space) the of the Observability matrix
%
\begin{align*}
  \mathbf{O} =
\left[ \begin{array}{c}
C 
\\
C A
\\
C A^2 
\\
  \vdots
\\
C A^{n-1}
\end{array}
\right] 
\end{align*} 
%

A CT system is Observable if and only of
%
\begin{align*}
  \mathrm{rank} ( \mathbf{O} ) &= n
\end{align*} 
%
or equivalently 
%
\begin{align*}
 \mathrm{Ra} ( \mathbf{O} ) = \mathbb{R}^n
\end{align*} 
%
or equivalently 
%
\begin{align*}
 \mathrm{dim} \left ( \mathcal{N} ( \mathbf{O} ) \right) = 0
\end{align*} 

\vspace{6pt}

\textbf{Remark:} A state-space representation that is
in observable canonical form is always fully Observable

\vspace{6pt}

\textbf{Remark:} A state-space representation is called minimal 
if it is both fully Controllable and Observable. 


\section{Luenberger Observer}

In general the state, $x(t)$, of a system
is not accessible and \textit{observers, estimators, filters})
have to be used to extract this information.
The output, $y(t)$, represents the measurements
which is a function of $x(t)$ and $u(t)$.
%
\begin{align*}
  \dot{x} &= A x + B u
  \\
  y &= C x + D u
\end{align*}
%
A Luenberger observers is built using a ``simulated'' model of the 
system and the errors caused by the mismatched initial conditions 
$x_0 \neq \hat{x}_0$ (or other types of perturbations)
are reduced by introducing output error feedback.

Let's assume that the state vector of the simulated system
is $\hat{x}$, then the state space equation of this
synthetic system takes the form
%
\begin{align*}
  \dot{\hat{x}} &= A \hat{x} + H u
  \\
  \hat{y} &= C \hat{x} + D u
\end{align*}
%
Note that since $u$ is the input that is supplied by the 
controller, we assume that it is known apriori. If $x(0) = \hat{x}[=(0)$ and
when there is no model mismatch or uncertainty in the system
then we expect that $x(t) = \hat{x}(t)$ and $y(t) = \hat{y}(t)$ 
for all $t \in \mathbb{R}^+$. When $x(0) \neq \hat{x}(0)$, then we should observe a 
difference between the measured and predicted output
$y(t) \neq \hat{y}(t)$ (if the initial condition is not in the 
unobservable sub-space). The core idea in Luenberger observer
is feeding the error in the output prediction 
$y(t) - \hat{y}(t)$ to the simulated system via a linear feedback gain.
%
\begin{align*}
  \dot{\hat{x}} &= A \hat{x} + H u + L \left( y - \hat{y} \right) 
  \\
  \hat{y} &= C \hat{x} + D u
\end{align*}
%
In order to understand how a Luenberger observer works and
to choose a proper observer gain $L$, we define an error signal
$e = x - \hat{x}$. The dynamics w.r.t $e$ can be derived
as
%
\begin{align*}
  \dot{e} &= \dot{x} - \dot{\hat{x}}
        \\
     &= \left( A x + H u \right)
  - \left( A \hat{x} + H u + L \left( y - \hat{y} \right)
       \right)
\\
   \dot{e} &= \left( A - L C \right) e
\end{align*}
%
where $e(0) = x(0) - \hat{x}(0)$ denotes the error in the initial
condition. 

If the matrix $\left( A - L C \right)$ is stable then the errors in
initial condition will diminish eventually. Moreover, in order
to have a good observer/estimator performance the observer
convergence should be sufficiently fast. 

\newpage

\section*{Observer Gain \& Pole Placement}

Similar to the state-feedback gain design,
the fundamental principle of ``pole-placement'' Observer design is that
we first define a desired closed-loop eigenvalue set and 
compute the associated desired characteristic polynomial. 
%
\begin{align*}
 \mathcal{E}^* &= \lbrace \lambda_1^* , \ \cdots, \  \lambda_n^*
                 \rbrace
  \\
  p^*(s) &= \left( s - \lambda_1^* \right) \cdots \left( s - \lambda_n^*
         \right)                         
  \\
  &= s^n + a_0^* z^{n-1} + \cdots + a_{n-2}^* z + a_{n-1}^*
\end{align*}
%
The necessary and sufficient condition on arbitrary observer pole-placement
is that the system should be fully Observable. Then, we ca tune $L$ such
that 
%
\begin{align*}
  \mathrm{det} \left( s I - ( A - L C ) \right) = p^*(s)
\end{align*}
%
\subsection*{Direct Design of Observer Gain}

If $n$ is small, the most efficient method could be the direct
design. 

\textbf{Example:} Consider the following DT system
%
\begin{align*}
 \dot{x} &= \left[ \begin{array}{cc} 1 & 0 \\ 0 & 2 \end{array} \right] x
    + \left[ \begin{array}{c} 1 \\ 1 \end{array} \right] u
\\
 y &= \left[ \begin{array}{cc} 1 & -1 \end{array} \right] x
\end{align*}
% 
Design an observer such that estimater poles are located at 
$\lambda_{1,2} = -5$ (Dead-beat Observer)

\textbf{Solution:} Desired characteristic equation can be computed as
%
\begin{align*}
  p^*(s) = s^2 + 10 s + 25
\end{align*}
%
Let $L = \left[ \begin{array}{c} l_2 \\ l_1 \end{array} \right]$, then
the characteristic equation of $(G - L C)$ can be computed as
%
\begin{align*}
  \mathrm{det} \left( s I - ( A - L C ) \right) &= 
  \mathrm{det} \left(
  \left[ \begin{array}{cc} s - 1 + l_2 & -l_2 \\ l_1 & s - 2 - l_1 \end{array} \right]
  \right)
\\
&= s^2 + s (l_2 - l_1 - 3) + (l_1 - 2 l_2 + 2)
\end{align*}
%
If we match the equations
%
\begin{align*}
  l_2 - l_1 &= 13
\\
  l_1 - 2 l_2 &= 23
\\
 l_2 &= -49
\\
 l_1 &= -36
\end{align*}
%
Thus $L = \left[ \begin{array}{c} -49 \\ -36 \end{array} \right]$




% **** This ENDS THE EXAMPLES. DON'T DELETE THE FOLLOWING LINE:
\end{document}