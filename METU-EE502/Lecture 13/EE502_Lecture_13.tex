% 


\documentclass[twoside]{article}
\setlength{\oddsidemargin}{0.25 in}
\setlength{\evensidemargin}{-0.25 in}
\setlength{\topmargin}{-0.6 in}
\setlength{\textwidth}{6.5 in}
\setlength{\textheight}{8.5 in}
\setlength{\headsep}{0.75 in}
\setlength{\parindent}{0 in}
\setlength{\parskip}{0.1 in}

%
% ADD PACKAGES here:
%

\usepackage{amsmath,amsfonts,graphicx,arydshln}


\newcounter{lecnum}
\renewcommand{\thepage}{\thelecnum-\arabic{page}}
\renewcommand{\thesection}{\thelecnum.\arabic{section}}
\renewcommand{\theequation}{\thelecnum.\arabic{equation}}
\renewcommand{\thefigure}{\thelecnum.\arabic{figure}}
\renewcommand{\thetable}{\thelecnum.\arabic{table}}

%
% The following macro is used to generate the header.
%
\newcommand{\lecture}[4]{
   \pagestyle{myheadings}
   \thispagestyle{plain}
   \newpage
   \setcounter{lecnum}{#1}
   \setcounter{page}{1}
   \noindent
   \begin{center}
   \framebox{
      \vbox{\vspace{2mm}
    \hbox to 6.28in { {\bf EE502 - Linear Systems Theory II
	\hfill Spring 2023} }
       \vspace{4mm}
       \hbox to 6.28in { {\Large \hfill Lecture #1 \hfill} }
       \vspace{2mm}
       \hbox to 6.28in { {\it Lecturer: #2 \hfill } }
      \vspace{2mm}}
   }
   \end{center}
   \markboth{Lecture #1}{Lecture #1}

   \vspace*{4mm}
}

\renewcommand{\cite}[1]{[#1]}
\def\beginrefs{\begin{list}%
        {[\arabic{equation}]}{\usecounter{equation}
         \setlength{\leftmargin}{2.0truecm}\setlength{\labelsep}{0.4truecm}%
         \setlength{\labelwidth}{1.6truecm}}}
\def\endrefs{\end{list}}
\def\bibentry#1{\item[\hbox{[#1]}]}


\newcommand{\fig}[3]{
			\vspace{#2}
			\begin{center}
			Figure \thelecnum.#1:~#3
			\end{center}
	}

% Use these for theorems, lemmas, proofs, etc.
\newtheorem{theorem}{Theorem}[lecnum]
\newtheorem{lemma}[theorem]{Lemma}
\newtheorem{proposition}[theorem]{Proposition}
\newtheorem{claim}[theorem]{Claim}
\newtheorem{corollary}[theorem]{Corollary}
\newtheorem{definition}[theorem]{Definition}
\newenvironment{proof}{{\bf Proof:}}{\hfill\rule{2mm}{2mm}}
\newtheorem{exmp}[theorem]{Ex}

% **** IF YOU WANT TO DEFINE ADDITIONAL MACROS FOR YOURSELF, PUT THEM HERE:

\begin{document}

% Lecture Details
\lecture{13}{Asst. Prof. M. Mert Ankarali}

%%%%%%%%%%%%%%%%%%%%%%%%%%

\section{State Feedback}

The state-feedback based control-policies for LTI systems start with the assumption that 
we have ``access'' to all of the states of the systems either via direct measurement 
or through some observer/estimator/tracker. In that context a family of state-feedback 
controllers for CT- and DT-LTI systems can be constructed as
%
\begin{align*}
 u(t) = \gamma r(t) - K x(t) \ \& \   u[k] = \gamma r[k] - K x[k]
\end{align*}
%
where $r(t)$ can be considered as the reference signal (most of the time it is), 
$\gamma$ is a feed-forward scaling factor, and $K$ is the state-feedback gain. 
Now let's find a state-space representation for dynamics of the closed-loop system
for both CT- and DT-LTI systems under state-feedback rule proposed above
%
\begin{align*}
	\dot{x} &= A x + B \left( \gamma r(t) - K x(t) \right) \, \Rightarrow \, \dot{x} = \left( A - B K \right) x + \gamma B r
	\\
	x[k+1] &= A x + B \left( \gamma r[k] - K x[k] \right) \, \Rightarrow \, x[k+1] = \left( A - B K \right) x[k] + \gamma B r[k]
\end{align*}
%
In both cases the closed loop system and input matrices take the following form.
%
\begin{align*}
	A_c = A - B K \ , \  B_c = \gamma B r
\end{align*}
%
A key question in this domain is whether we can find a $K$ such that eigenvalues of $A_c$ is located at arbitrary 
desired locations. 

\textbf{Theorem: (Eigenvalue/Pole Placement)} Given $(A,B)$, $\exists K$ s.t. 
%
\begin{align*}
	\mathrm{det}\left[ \lambda I - (A - B K ) \right] &= \lambda^n + a_{n-1}^*  \lambda^{n-1}
	+ \cdots + a_{1}^* \lambda + a_0^*
	\\
	\forall \mathcal{A} &= \lbrace a_0^* , \, a_1^* \, \cdots \, a_{n-1}^* \rbrace , \, a_i^* \in \mathbb{R}
\end{align*} 
%
if and only if $(A,B)$ is reachable. 

\textbf{Proof:} For a general complete proof, we need to show that reachability of $(A,B)$ is necessary and sufficient. 

\textbf{Proof of necessity: } Let's assume that $(A,B)$ not reachable and $\exists (\lambda_u , w_u^T)$ pair such that
$w_u^T A = w_u^T \lambda_u $ and $w_u^T B = 0$. Now check weather $w_u^T$ is a left eigenvector of $A_c$
%
\begin{align*}
	w_u^T A_c &= w_u^T (A - B K ) = w_u^T A - w_u^T B K = w_u^T \lambda_u - 0 = w_u^T \lambda_u
	\\
	w_u^T B_c &= w_u^T B \gamma = 0
\end{align*} 
%
Here not only we showed that $\lambda_u$ could not be moved hence it is not possible to locate the poles
arbitrary locations, we also showed that state-feedback rules do not affect reachability.

\textbf{Proof of sufficiency:} We will only show the sufficiency for a multi-input case, i.e. $B \in \mathbb{R}^{n \times 1}$, 
however, the reader should note that a complete proof multi-input case must be analyzed. 

Let's assume that $(A,B)$ is reachable, and we know that reachability is invariant under similarity transformations, i.e.
%
\begin{align*}
	z &= T^{-1} x \, , \, \mathrm{det}(T) \neq 0 \, \Rightarrow \, \dot{z} = \bar{A} z + \bar{B} u \\
	\bar{A} &= T^{-1} A T \ , \ T^{-1} B 
\end{align*}
%
and $(\bar{A}, \bar{B})$ is reachable. Noe let's choose $T$ such that
%
\begin{align*}
T = \mathbf{R} = \left[ \begin{array}{c|c|c|c|c} A^{n-1} B & A^{n-2} B & \cdots & A B &  B \end{array} \right]
\end{align*}
%
then $\bar{B}$ can be derived as
% 
\begin{align*}
B = T \bar{B} = \left[ \begin{array}{c|c|c|c|c} A^{n-1} B & A^{n-2} B & \cdots & A B &  B \end{array} \right] \bar{B} 
\, \Rightarrow \, B = \begin{bmatrix} 0 \\ 0 \\ \vdots \\ 0 \\ 1 \end{bmatrix}
\end{align*}
%
and similarly $\bar{A}$ can be expressed as
\begin{align*}
A \mathbf{R} &= \mathbf{R} \bar{A}
\\
A \left[ \begin{array}{c|c|c|c|c} A^{n-1} B & A^{n-2} B & \cdots & A B &  B \end{array} \right] &= \left[ \begin{array}{c|c|c|c|c} A^{n-1} B & A^{n-2} B & \cdots & A B &  B \end{array} \right] \bar{A}
\\
\left[ \begin{array}{c|c|c|c|c} A^{n} B & A^{n-1} B & \cdots & A B &  B \end{array} \right] &= \left[ \begin{array}{c|c|c|c|c} A^{n-1} B & A^{n-2} B & \cdots & A B &  B \end{array} \right] \left[ \begin{array}{c|c|c|c|c|c} 
\bar{a}_{11} & 1 & 0 & 0 & \cdots & 0 \\
\bar{a}_{12} & 0 & 1 &  0 & \cdots & 0 \\
\vdots & \vdots & \vdots & \ddots &  & \vdots \\
\bar{a}_{1(n-2)} & 0 & 0 & \cdots  & 1 & 0 \\
\bar{a}_{1(n-1)} & 0 & 0 & \cdots  & 0 & 1 \\
\bar{a}_{1(n)} & 0 & 0 & \cdots  & 0 & 0 \\
\end{array} 
\right] 
\end{align*}
%
We can find $a_{1i}$'s using Cayley-Hamilton theorem
%
\begin{align*}
A^{n} B &= \bar{a}_{11} A^{n-1} B + \bar{a}_{12} A^{n-2} B + \cdots + \bar{a}_{1(n-1)} A B + \bar{a}_{1(n)} B
\\
A^{n} &= - \left( a_{n-1} A^{n-1} + a_{n-1} A^{n-2} + \cdots + a_1 A + a_ I \right) \, \mathrm{where}
\\
\mathrm{det}[\lambda I - A] &= \lambda^n + a_{n-1}  \lambda^{n-1}
	+ \cdots + a_{1} \lambda + a_0
\end{align*}
then $\bar{A}$ takes the form
\begin{align*}
\bar{A} &= \left[ \begin{array}{c|c|c|c|c|c} 
-a_{n-1} & 1 & 0 & 0 & \cdots & 0 \\
-a_{n-2} & 0 & 1 &  0 & \cdots & 0 \\
\vdots & \vdots & \vdots & \ddots &  & \vdots \\
-a_{2} & 0 & 0 & \cdots  & 1 & 0 \\
-a_{1} & 0 & 0 & \cdots  & 0 & 1 \\
-a_{0} & 0 & 0 & \cdots  & 0 & 0 
\end{array} 
\right] 
\end{align*}
%
Note that this similarity transformation is in a companion form however we will transform this into (more useful) 
reachable canonical form
%
\begin{align*}
\tilde{A} = \left[ \begin{array}{c|c|c|c|c|c} 
0 & 1 & 0 & 0 & \cdots & 0 \\
0 & 0 & 1 &  0 & \cdots & 0 \\
\vdots & \vdots & \vdots & \ddots &  & \vdots \\
0 & 0 & 0 & \cdots  & 1 & 0 \\
0 & 0 & 0 & \cdots  & 0 & 1 \\
-a_{0} & -a_{1} & -a_{2} & \cdots  & -a_{n-2} & -a_{n-1} 
\end{array} 
\right] 
\ , \ \tilde{B} = \begin{bmatrix} 0 \\ 0 \\ \vdots \\ 0 \\ 1 \end{bmatrix}
\end{align*}
%
Let's $\tilde{\mathbf{R}}$ be the reachability matrix of $(\tilde{A} , \tilde{B})$ then we know that 
%
%
\begin{align*}
	\bar{A} &= \tilde{\mathbf{R}}^{-1} A \tilde{\mathbf{R}} \ , \ \bar{B} = \tilde{\mathbf{R}}^{-1} B 
\end{align*}
%
where $(\bar{A},\bar{B})$ are the matrices of the companion form derived above. Thus if we let 
$T_{\mathbf{R} \tilde{\mathbf{R}}^{-1}} = \mathbf{R} \tilde{\mathbf{R}}^{-1}$, where $\mathbf{R}$ is the reachability matrix of the
original representation and $\tilde{\mathbf{R}}$ is the reachability matrix of the reachable
canonical form and adopt $T$ for similarity transformation we obtain
%
\begin{align*}
	q &= \left( T_{\mathbf{R} \tilde{\mathbf{R}}^{-1}} \right)^{-1} x \, \Rightarrow \, \dot{q} = \tilde{A} q + \tilde{B} u \\
	\tilde{A} &= \left( T_{\mathbf{R} \tilde{\mathbf{R}}^{-1}} \right)^{-1} \, A \, T_{\mathbf{R} \tilde{\mathbf{R}}^{-1}} \ , \ \tilde{B} = \left(  T_{\mathbf{R} \tilde{\mathbf{R}}^{-1}} \right)^{-1} B 
\end{align*}
%
Now let's apply the state feedback rule in the transformed coordinates.
%
%
\begin{align*}
	\dot{q} &= \tilde{A} q + \tilde{B} u \ , \ u = \alpha r - \tilde{K} q 
 \\
 \dot{q} &= ( \tilde{A} - \tilde{B} \tilde{K} ) q + \alpha \tilde{B} u 
 \\
 \\
 ( \tilde{A} - \tilde{B} \tilde{K} ) &= 
 \left[ \begin{array}{c|c|c|c|c|c} 
0 & 1 & 0 & 0 & \cdots & 0 \\
0 & 0 & 1 &  0 & \cdots & 0 \\
\vdots & \vdots & \vdots & \ddots &  & \vdots \\
0 & 0 & 0 & \cdots  & 1 & 0 \\
0 & 0 & 0 & \cdots  & 0 & 1 \\
-a_{0} & -a_{1} & -a_{2} & \cdots  & -a_{n-2} & -a_{n-1} 
\end{array} \right] - \begin{bmatrix} 0 \\ 0 \\ \vdots \\ 0 \\ 1 \end{bmatrix}
\begin{bmatrix} \tilde{k}_0 & \tilde{k}_1 & \cdots & \tilde{k}_{n-1} \end{bmatrix}
\\
&=  \left[ \begin{array}{c|c|c|c|c|c} 
0 & 1 & 0 & 0 & \cdots & 0 \\
0 & 0 & 1 &  0 & \cdots & 0 \\
 &  &  &   &  &  
\\ 
\vdots & \vdots & \vdots & \ddots &  & \vdots \\
&  &  &   &  & 
\\ 
0 & 0 & 0 & \cdots  & 1 & 0 \\
0 & 0 & 0 & \cdots  & 0 & 1 \\
-a_{0}-\tilde{k}_0 & -a_{1}-\tilde{k}_1 & -a_{2}-\tilde{k}_2 & \cdots  & -a_{n-2}-\tilde{k}_{n-2} & -a_{n-1} - \tilde{k}_{n-1}
\end{array} \right]
\\
\mathrm{det}\left( \tilde{A} - \tilde{B} \tilde{K} \right) &=
\lambda^n + ( a_{n-1} + \tilde{k}_{n-1}  \lambda^{n-1}
	+ \cdots + ( a_{1} + \tilde{k}_1 )  \lambda + ( a_0 + \tilde{k}_0 )
\end{align*}
%
It is evident that $\exists \tilde{K}$ s.t. 
%
\begin{align*}
	\mathrm{det}\left( \tilde{A} - \tilde{B} \tilde{K} \right) &= \lambda^n + a_{n-1}^*  \lambda^{n-1}
	+ \cdots + a_{1}^* \lambda + a_0^*
	\\
	\forall \mathcal{A} &= \lbrace a_0^* , \, a_1^* \, \cdots \, a_{n-1}^* \rbrace , \, a_i^* \in \mathbb{R}
\end{align*} 
%
Now we need to find the state-feedback gain and rule in the original coordinate frame
\begin{align*}
 u = \alpha r - \tilde{K} q = \alpha r - \tilde{K} \left( T_{\mathbf{R} \tilde{\mathbf{R}}^{-1}} \right)^{-1} x     
 \\
 \dot{x} = \left( A - B \tilde{K} \left( T_{\mathbf{R} \tilde{\mathbf{R}}^{-1}} \right)^{-1} \right) + \alpha B r \ , \ K = \tilde{K} \left( T_{\mathbf{R} \tilde{\mathbf{R}}^{-1}} \right)^{-1}
\end{align*} 
%
Now let's verify that $(A - B K)$ and $( \tilde{A} - \tilde{B} \tilde{K} )$ has the same characteristic equations
%
\begin{align*}
 \mathrm{det}( \lambda I - \left[ A - B K \right] ) &=
 \mathrm{det}\left( \lambda I - \left[ A - B \tilde{K} \left( T_{\mathbf{R} \tilde{\mathbf{R}}^{-1}} \right)^{-1} \right] \right) 
 \\
 &= \mathrm{det}\left( \lambda I - \left( T_{\mathbf{R} \tilde{\mathbf{R}}^{-1}} \right)^{-1} 
 \left[ A - B \tilde{K} \left( T_{\mathbf{R} \tilde{\mathbf{R}}^{-1}} \right)^{-1} \right] 
T_{\mathbf{R} \tilde{\mathbf{R}}^{-1}} 
 \right) 
 \\
 &= \mathrm{det}\left( \lambda I -
 \left[ \left( T_{\mathbf{R} \tilde{\mathbf{R}}^{-1}} \right)^{-1}  A T_{\mathbf{R} \tilde{\mathbf{R}}^{-1}} - \left( T_{\mathbf{R} \tilde{\mathbf{R}}^{-1}} \right)^{-1}  B \tilde{K}  \right] 
 \right) 
 \\
 &= \mathrm{det}\left( \lambda I - \left[ \tilde{A} - \tilde{B} \tilde{K} \right] \right)
\end{align*} 
%
this completes the proof for SISO case.

 \section{Stabilizability}

 \textbf{Definition:} An LTI system is called stabilizable, if 
 $\forall x_0 \in \mathbb{R}^{n}$, $\exists \, u(t)$ (or $u[k]$) such that
 $\lim_{t\to\infty} x(t) = 0$ (or $\lim_{k\to\infty} x[k] = 0$).

 An obvious \textbf{sufficient} condition on stabilizability for CT-LTI and DT-LTI systems is that, if $(A,B)$ is reachable then $(A,B)$ stabilizable, since
 in such a case $\forall x_0 \in \mathbb{R}^{n}$, $\exists \, u(t)$ or $u[k]$
 such that $x(T) = 0 \, \forall T < 0$ for CT systems, and $x[k] = 0 \, \forall  k > n$. 

Another \textbf{sufficient} (but weaker) condition on stabilizability for DT-LTI systems is that if $(A,B)$ is controllable then $(A,B)$ stabilizable, since
 in such a case $\forall x_0 \in \mathbb{R}^{n}$, $\exists \, u[k]$
 such that $x[k] = 0 \, \forall  k > n$. 

\textbf{Theorem} The following statements are equal 
\begin{itemize}
\item $(A,B)$ is stabilizable $\iff$
\item if unreachable modes of $(A,B)$ are asymptotically stable $\iff$
\item $\forall (\lambda_i , w_i)$ s.t. $w_i^T A = \lambda_i w_i^T$ $\iff$
and $\mathrm{Re}\lbrace \lambda_i \rbrace \geq 0$ (or $|\lambda_i|\geq0$),
$w_i \notin \mathcal{N}(B)$ $\iff$
\item $\mathrm{rank}\left[ A - \lambda I \, | \, B \right] = n$,
$\forall \lambda \in \mathbb{C}$ s.t. $\mathrm{Re}\lbrace \lambda_i \rbrace \geq 0$ for CT systems (or $\forall \lambda \in \mathbb{C}$ s.t. $|\lambda|  \geq 1$ for DT systems)
\end{itemize}

\textbf{Proof:} Given $(A,B)$ that is not necessarily reachable, we can transform it into the standard unreachable form
%
\begin{align*}
\dot{x} = \left[ \begin{array}{c} \dot{x}_r \\ \dot{x}_{\bar{r}} \end{array} \right]  = \left[ \begin{array}{c|c} A_{r} & A_{12} \\ \hline 0 & A_{\bar{r}} \end{array} \right] \left[ \begin{array}{c} x_r \\ x_{\bar{r}} \end{array} \right]  
+ \left[ \begin{array}{c} B_r \\ 0 \end{array} \right] u
\end{align*}
%
Let $\dot{x}_{\bar{r}} = A_{\bar{r}} x_{\bar{r}}$ is not asymptotically stable then obviously 
the whole system is unstable; thus, the system is not stabilizable .
%
Now let $\dot{x}_{\bar{r}} = A\bar{r} x_{\bar{r}}$ is asymptotically stable, then 
let $u = - K_{r} x_{r}$ then the closed-loop system takes the form
%
\begin{align*}
\dot{x} = \left[ \begin{array}{c} \dot{x}_r \\ \dot{x}_{\bar{r}} \end{array} \right]  = \left[ \begin{array}{c|c} A_{r} - B_r K_r & A_{12} \\ \hline 0 & A_{\bar{r}} \end{array} \right] \left[ \begin{array}{c} x_r \\ x_{\bar{r}} \end{array} \right]  
\end{align*}
%
Since $(A_r,B_r)$ pair is reachable we can find $K$ such that eigenvalues of  $(A_{r} - B_r K_r)$ are located in the
open-left-half of the complex plane, thus the system is stabilizable. The same logic can be adopted for DT systems.


% **** This ENDS THE EXAMPLES. DON'T DELETE THE FOLLOWING LINE:
\end{document}
