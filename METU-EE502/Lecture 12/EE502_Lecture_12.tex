% 


\documentclass[twoside]{article}
\setlength{\oddsidemargin}{0.25 in}
\setlength{\evensidemargin}{-0.25 in}
\setlength{\topmargin}{-0.6 in}
\setlength{\textwidth}{6.5 in}
\setlength{\textheight}{8.5 in}
\setlength{\headsep}{0.75 in}
\setlength{\parindent}{0 in}
\setlength{\parskip}{0.1 in}

%
% ADD PACKAGES here:
%

\usepackage{amsmath,amsfonts,graphicx,arydshln}


\newcounter{lecnum}
\renewcommand{\thepage}{\thelecnum-\arabic{page}}
\renewcommand{\thesection}{\thelecnum.\arabic{section}}
\renewcommand{\theequation}{\thelecnum.\arabic{equation}}
\renewcommand{\thefigure}{\thelecnum.\arabic{figure}}
\renewcommand{\thetable}{\thelecnum.\arabic{table}}

%
% The following macro is used to generate the header.
%
\newcommand{\lecture}[4]{
   \pagestyle{myheadings}
   \thispagestyle{plain}
   \newpage
   \setcounter{lecnum}{#1}
   \setcounter{page}{1}
   \noindent
   \begin{center}
   \framebox{
      \vbox{\vspace{2mm}
    \hbox to 6.28in { {\bf EE502 - Linear Systems Theory II
	\hfill Spring 2023} }
       \vspace{4mm}
       \hbox to 6.28in { {\Large \hfill Lecture #1 \hfill} }
       \vspace{2mm}
       \hbox to 6.28in { {\it Lecturer: #2 \hfill } }
      \vspace{2mm}}
   }
   \end{center}
   \markboth{Lecture #1}{Lecture #1}

   \vspace*{4mm}
}

\renewcommand{\cite}[1]{[#1]}
\def\beginrefs{\begin{list}%
        {[\arabic{equation}]}{\usecounter{equation}
         \setlength{\leftmargin}{2.0truecm}\setlength{\labelsep}{0.4truecm}%
         \setlength{\labelwidth}{1.6truecm}}}
\def\endrefs{\end{list}}
\def\bibentry#1{\item[\hbox{[#1]}]}


\newcommand{\fig}[3]{
			\vspace{#2}
			\begin{center}
			Figure \thelecnum.#1:~#3
			\end{center}
	}

% Use these for theorems, lemmas, proofs, etc.
\newtheorem{theorem}{Theorem}[lecnum]
\newtheorem{lemma}[theorem]{Lemma}
\newtheorem{proposition}[theorem]{Proposition}
\newtheorem{claim}[theorem]{Claim}
\newtheorem{corollary}[theorem]{Corollary}
\newtheorem{definition}[theorem]{Definition}
\newenvironment{proof}{{\bf Proof:}}{\hfill\rule{2mm}{2mm}}
\newtheorem{exmp}[theorem]{Ex}

% **** IF YOU WANT TO DEFINE ADDITIONAL MACROS FOR YOURSELF, PUT THEM HERE:

\begin{document}

% Lecture Details
\lecture{12}{Asst. Prof. M. Mert Ankarali}


%%%%%%%%%%%%%%%%%%%%%%%%%%

\section{The Kalman Decomposition}

In reachability and observability lectures, we derived two types of standards forms, 
specifically for unreachable systems and unobservable systems (separately). Now our goal is
to propose a general standard form for a unreachable and unobservable system, based on
the Kalman decomposition. The process is exactly same for bot DT and CT systems, thus 
we will present the decomposition for only CT systems. Let
%
\begin{align*}
  \dot{x} &= A x + B u \ , \ y &= C x + D u \ \& \   x \in \mathbb{R}^n
\end{align*}
%
Let's assume that system is neither reachable, nor observable and
%
\begin{align*}
  \mathrm{rank}[ \mathbf{R} ] = r < n \ , \ \mathrm{range}[ \mathbf{R} ] = \mathcal{R}
  \\
  \mathrm{dim}[ \mathcal{N} ( \mathbf{O} ) ] = \bar{o} > 0 \ , \ \bar{\mathcal{O}} = \mathcal{N} ( \mathbf{O} )
\end{align*}
%
Let's consider the following similarity transformation
%
\begin{align*}
  \hat{A} = T^{-1} A T \ , \ \hat{B} = T^{-1} B \ , \ \hat{C} = C T \ \& \ D = D
\end{align*}
%
Let  
%
\begin{align*}
  T = \left[ \begin{array}{c|c|c|c} T_{r\bar{o}} & T_{ro} & T_{\bar{r} \bar{o}} & T_{\bar{r} o} \end{array} \right]
\end{align*}
%
Let's define sub-matrices as follows:
%
\begin{enumerate}
\item Let $\mathcal{R}\bar{\mathcal{O}} = \mathcal{R} \cap \bar{\mathcal{O}}$, i.e. 
$x \in \mathcal{R}\bar{\mathcal{O}} \, \Rightarrow \, x \in \mathcal{R} \ \& \ x \in \bar{\mathcal{O}}$. 
Choose $T_{r\bar{o}}$ such that columns of $T_{r\bar{o}}$ forms a basis for $\mathcal{R}\bar{\mathcal{O}}$.
%
\item Choose a $T_{ro}$ such that $\mathrm{Ra}\left[ \begin{array}{c|c} T_{r\bar{o}} & T_{ro} \end{array} \right] = \mathcal{R} = \mathrm{Ra}[\mathbf{R}]$, i.e. columns of $T_{ro}$ complements $T_{r\bar{o}}$ in the reachable sub-space and  
%
\item Choose a $T_{\bar{r}\bar{o}}$ such that $\mathrm{Ra}\left[ \begin{array}{c|c} T_{r\bar{o}} & T_{\bar{r}\bar{o}} \end{array} \right] = \bar{\mathcal{O}} = \mathcal{N}[ \mathbf{O} ]$,
i.e. columns of $T_{\bar{r}\bar{o}}$ complements $T_{r\bar{o}}$ in the unobservable sub-space 
%
\item Choose  $T_{\bar{r}o}$ such that $\mathrm{Ra}[ T ] = \mathbb{R}^n$
%
\end{enumerate}
%
Let's remember the important sub-spaces invariant under $A$ and some critical features that will be helpful 
for constructing the Kalman decomposition
\begin{align*}
    &\begin{array}{c} A \mathcal{R} \subset \mathcal{R}
    \\
    A \bar{\mathcal{O}} \subset \bar{\mathcal{O}} \end{array} \, \Rightarrow \, A \mathcal{R}\bar{\mathcal{O}} \subset \mathcal{R}\bar{\mathcal{O}}
    \\
    &\mathrm{Ra}[B] \subset \mathcal{R}
    \\
    &\bar{\mathcal{O}} \subset \mathcal{N}[C]
\end{align*}
%
Let's analyze the similarity transformation of the system matrix.
%
\begin{align*}
    A T &=  T A
    \\
    A \left[ \begin{array}{c|c|c|c} T_{r\bar{o}} & T_{ro} & T_{\bar{r} \bar{o}} & T_{\bar{r} o} \end{array} \right]
    &= \left[ \begin{array}{c|c|c|c} T_{r\bar{o}} & T_{ro} & T_{\bar{r} \bar{o}} & T_{\bar{r} o} \end{array} \right]
    \left[ \begin{array}{c|c|c|c} A_{11} & A_{12} & A_{13} & A_{14} \\ \hline 
    A_{21} & A_{22} & A_{23} & A_{24} \\ \hline 
    A_{31} & A_{32} & A_{33} & A_{34} \\ \hline 
    A_{41} & A_{42} & A_{43} & A_{44} \\ 
    \end{array} \right]
\end{align*}
%
Let's expand $A T_{r\bar{o}}$
%
\begin{align*}
    A T_{r\bar{o}} = \left[ \begin{array}{c|c|c|c} T_{r\bar{o}} & T_{ro} & T_{\bar{r} \bar{o}} & T_{\bar{r} o} \end{array} \right]
    \left[ \begin{array}{c} A_{11} \\ \hline 
    A_{21} \\ \hline 
    A_{31} \\ \hline 
    A_{41} \\ 
    \end{array} \right]
\end{align*}
%
Since $\mathrm{Ra}[ T_{r\bar{o}} ]$ is invariant under $A$ (i.e. $A \mathcal{R}\bar{\mathcal{O}} \subset \mathcal{R}\bar{\mathcal{O}}$), 
$A_{i1} = 0 , \, \forall \, i>1$. 

Now let's focus on $A \left[ \begin{array}{c|c} T_{r\bar{o}} & T_{ro} \end{array} \right]$
%
\begin{align*}
    A \left[ \begin{array}{c|c} T_{r\bar{o}} & T_{ro} \end{array} \right] = \left[ \begin{array}{c|c|c|c} T_{r\bar{o}} & T_{ro} & T_{\bar{r} \bar{o}} & T_{\bar{r} o} \end{array} \right]
    \left[ \begin{array}{c|c} A_{11} & A_{12} \\ \hline 
    0 & A_{22} \\ \hline 
    0 & A_{32} \\ \hline 
    0 & A_{42} \\ 
    \end{array} \right]
\end{align*}
%
Since $\mathrm{Ra}\left[ \begin{array}{c|c} T_{r\bar{o}} & T_{ro} \end{array} \right]$ is invariant under $A$ (i.e. $A \mathcal{R} \subset \mathcal{R}$), $A_{32} = 0$ and $A_{42} = 0$. 

Now let's focus on $A \left[ \begin{array}{c|c} T_{r\bar{o}} & T_{\bar{r}\bar{o}} \end{array} \right]$
%
\begin{align*}
    A \left[ \begin{array}{c|c} T_{r\bar{o}} & T_{\bar{r}\bar{o}} \end{array} \right] = \left[ \begin{array}{c|c|c|c} T_{r\bar{o}} & T_{ro} & T_{\bar{r}\bar{o}} & T_{\bar{r} o} \end{array} \right]
    \left[ \begin{array}{c|c} A_{11} & A_{13} \\ \hline 
    0 & A_{23} \\ \hline 
    0 & A_{33} \\ \hline 
    0 & A_{43} \\ 
    \end{array} \right]
\end{align*}
%
Since $\mathrm{Ra}\left[ \begin{array}{c|c} T_{r\bar{o}} & T_{\bar{r}\bar{o}} \end{array} \right]$ is invariant under $A$ (i.e. $A \bar{\mathcal{O}} \subset \bar{\mathcal{O}}$), $A_{23} = 0$ and $A_{43} = 0$. 

As a results $\hat{A}$ takes the form
%
\begin{align*}
    \hat{A} = 
    \left[ \begin{array}{c|c|c|c} A_{11} & A_{12} & A_{13} & A_{14} \\ \hline 
    0 & A_{22} & 0 & A_{24} \\ \hline 
    0 & 0 & A_{33} & A_{34} \\ \hline 
    0 & 0 & 0 & A_{44} \\ 
    \end{array} \right] = 
    \left[ \begin{array}{c|c|c|c} A_{r\bar{o}} & A_{12} & A_{13} & A_{14} \\ \hline 
    0 & A_{ro} & 0 & A_{24} \\ \hline 
    0 & 0 & A_{\bar{r}\bar{o}} & A_{34} \\ \hline 
    0 & 0 & 0 & A_{\bar{r}o} \\ 
    \end{array} \right]
\end{align*}
%
Now let's focus on input matrix transformation.
%
\begin{align*}
    B &= T \hat{B} = \left[ \begin{array}{c|c|c|c} T_{r\bar{o}} & T_{ro} & T_{\bar{r} \bar{o}} & T_{\bar{r} o} \end{array} \right] \left[ \begin{array}{c} B_1 \\ B_2 \\ B_3 \\ B_4 \end{array} \right]
\end{align*}
%
Since $\mathrm{Ra}(B) \subset \mathcal{R} =\mathrm{Ra}\left[ \begin{array}{c|c} T_{r\bar{o}} & T_{ro} \end{array} \right]$, $B_3 = 0$
and $B_4 = 0$ and thus $\hat{B}$ takes the form
\begin{align*}
    \hat{B} = \left[ \begin{array}{c} B_{r\bar{o}} \\ B_{ro} \\ 0 \\ 0 \end{array} \right]
\end{align*}
%
Now let's focus on input matrix transformation.
%
\begin{align*}
    C T &= \hat{C} 
    \\
    C \left[ \begin{array}{c|c|c|c} T_{r\bar{o}} & T_{ro} & T_{\bar{r} \bar{o}} & T_{\bar{r} o} \end{array} \right] &= 
    \left[ \begin{array}{c|c|c|c} C_1 & C_2 & C_3 & C_4 \end{array} \right]
    \\
    \left[ \begin{array}{c|c|c|c} C T_{r\bar{o}} & C T_{ro} & C T_{\bar{r} \bar{o}} & C T_{\bar{r} o} \end{array} \right] &= 
    \left[ \begin{array}{c|c|c|c} C_1 & C_2 & C_3 & C_4 \end{array} \right]
    \\
    \left[ \begin{array}{c|c|c|c} 0 & C T_{ro} & 0 & C T_{\bar{r} o} \end{array} \right] &= 
    \left[ \begin{array}{c|c|c|c} C_1 & C_2 & C_3 & C_4 \end{array} \right]
\end{align*}
%
and thus $\hat{C}$ takes the form
\begin{align*}
    \hat{C} = \left[ \begin{array}{c|c|c|c} 0 & C_{ro} & 0 & C_{\bar{r}o} \end{array} \right]
\end{align*}

Based on Kalman decomposition, we can obtain reachable and observable (minimal) sub-system, reachable only sub-system, and observable only system.
\begin{itemize}
    \item Reachable and observable sub-system
    \begin{align*}
        \dot{x}_{ro} &= A_{ro} x_{ro} + B_{ro} u
        \\
        y &= C_{ro} x_{ro} + D u
    \end{align*}
    %
    \item Reachable only (but not fully observable) sub-system
    \begin{align*}
        \frac{d}{dt}\begin{bmatrix} x_{r\bar{o}} \\ x_{ro}\end{bmatrix} &= 
        \left[ \begin{array}{c|c} A_{r\bar{o}} & A_{12} \\ \hline 
    0 & A_{ro}  \end{array} \right]
    \begin{bmatrix} x_{r\bar{o}} \\ x_{ro}\end{bmatrix} + \left[ \begin{array}{c} B_{r\bar{o}} \\ B_{ro} \end{array} \right] u
        \\
        y &= \left[ \begin{array}{c|c} 0 & C_{ro} \end{array} \right] \begin{bmatrix} x_{r\bar{o}} \\ x_{ro}\end{bmatrix}  + D u
    \end{align*}
    %
    \item Observable only (but not fully reachable) sub-system
    \begin{align*}
        \frac{d}{dt}\begin{bmatrix} x_{ro} \\ x_{\bar{r}o}\end{bmatrix} &= 
        \left[ \begin{array}{c|c} A_{ro} & A_{24} \\ \hline 
    0 & A_{\bar{r}o}  \end{array} \right]
    \begin{bmatrix} x_{ro} \\ x_{\bar{r}o}\end{bmatrix} + \left[ \begin{array}{c} B_{ro} \\ 0 \end{array} \right] u
        \\
        y &= \left[ \begin{array}{c|c} C_{ro} & C_{\bar{r}o} \end{array} \right] \begin{bmatrix} x_{ro} \\ x_{\bar{r}o}\end{bmatrix} + D u
    \end{align*}
\end{itemize}

\begin{exmp}
Let}
\end{exmp}
\begin{align*}
    x[k+1] &= \left[ \begin{array}{cccc} 0 & 1 & 0 & 1 \\ 
    2 & 1 & 0 & 3 \\ 
    0 & 1 & 0 & 1 \\ 
    0 & -1 & 0 & -1 
    \end{array} \right] x[k] + 
    \left[ \begin{array}{c} -1 \\ 
    1 \\ 
    1 \\ 
    1
    \end{array} \right] u
    \\
    y[k] &= \left[ \begin{array}{cccc} 0 & 1 & 1 & 2 \end{array} \right] x
\end{align*}
%
Obtain Kalman decomposition and extract  reachable and observable (minimal) sub-system, reachable only sub-system, and observable only system,
and non-reachable and non-observable sub-system.
%
Let's first find the reachability matrix and reachable subspace 
%
\begin{align*}
    \mathbf{R} &= \left[ \begin{array}{c|c|c|c} A^3 B & A^2 B & A B & B \end{array} \right] 
    =
    \left[ \begin{array}{cccc} 0 & 0 & 2 & -1 \\ 
    0 & 0 & 2 & 1 \\ 
    0 & 0 & 2 & 1 \\ 
    0 & 0 & -2 & 1 
    \end{array} \right] 
    \\
    \mathcal{R} &= \mathrm{Ra}[ \mathbf{R} ] = \mathrm{Span} \left\lbrace 
    \left[ \begin{array}{c} -1 \\  1 \\  1 \\  1  \end{array} \right] 
    \ , \
    \left[ \begin{array}{c} 1 \\  1 \\  1 \\  -1  \end{array} \right] 
    \right\rbrace
\end{align*}
%
Now let's find the observability matrix and non-observable subspace
%
%
\begin{align*}
    \mathbf{O} &= \left[ \begin{array}{c} C & C A & C A^2 & C A^3 \end{array} \right] 
    =
    \left[ \begin{array}{cccc} 0 & 1 & 1 & 2 \\ 
    2 & 0 & 0 & 2 \\ 
    0 & 0 & 0 & 0 \\ 
    0 & 0 & 0 & 0 
    \end{array} \right] 
    \\
    \mathbf{\mathcal{O}} &= \mathcal{N}[ \mathbf{O} ] = \mathrm{Span} \left\lbrace 
    \left[ \begin{array}{c} 0 \\  1 \\  -1 \\  0  \end{array} \right] 
    \ , \
    \left[ \begin{array}{c} 1 \\  2 \\  0 \\  -1  \end{array} \right] 
    \right\rbrace
\end{align*}
%
% **** This ENDS THE EXAMPLES. DON'T DELETE THE FOLLOWING LINE:
\end{document}
