% 


\documentclass[twoside]{article}
\setlength{\oddsidemargin}{0.25 in}
\setlength{\evensidemargin}{-0.25 in}
\setlength{\topmargin}{-0.6 in}
\setlength{\textwidth}{6.5 in}
\setlength{\textheight}{8.5 in}
\setlength{\headsep}{0.75 in}
\setlength{\parindent}{0 in}
\setlength{\parskip}{0.1 in}

%
% ADD PACKAGES here:
%

\usepackage{amsmath,amsfonts,graphicx,arydshln}


\newcounter{lecnum}
\renewcommand{\thepage}{\thelecnum-\arabic{page}}
\renewcommand{\thesection}{\thelecnum.\arabic{section}}
\renewcommand{\theequation}{\thelecnum.\arabic{equation}}
\renewcommand{\thefigure}{\thelecnum.\arabic{figure}}
\renewcommand{\thetable}{\thelecnum.\arabic{table}}

%
% The following macro is used to generate the header.
%
\newcommand{\lecture}[4]{
   \pagestyle{myheadings}
   \thispagestyle{plain}
   \newpage
   \setcounter{lecnum}{#1}
   \setcounter{page}{1}
   \noindent
   \begin{center}
   \framebox{
      \vbox{\vspace{2mm}
    \hbox to 6.28in { {\bf EE502 - Linear Systems Theory II
	\hfill Spring 2023} }
       \vspace{4mm}
       \hbox to 6.28in { {\Large \hfill Lecture #1 \hfill} }
       \vspace{2mm}
       \hbox to 6.28in { {\it Lecturer: #2 \hfill } }
      \vspace{2mm}}
   }
   \end{center}
   \markboth{Lecture #1}{Lecture #1}

   \vspace*{4mm}
}

\renewcommand{\cite}[1]{[#1]}
\def\beginrefs{\begin{list}%
        {[\arabic{equation}]}{\usecounter{equation}
         \setlength{\leftmargin}{2.0truecm}\setlength{\labelsep}{0.4truecm}%
         \setlength{\labelwidth}{1.6truecm}}}
\def\endrefs{\end{list}}
\def\bibentry#1{\item[\hbox{[#1]}]}


\newcommand{\fig}[3]{
			\vspace{#2}
			\begin{center}
			Figure \thelecnum.#1:~#3
			\end{center}
	}

% Use these for theorems, lemmas, proofs, etc.
\newtheorem{theorem}{Theorem}[lecnum]
\newtheorem{lemma}[theorem]{Lemma}
\newtheorem{proposition}[theorem]{Proposition}
\newtheorem{claim}[theorem]{Claim}
\newtheorem{corollary}[theorem]{Corollary}
\newtheorem{definition}[theorem]{Definition}
\newenvironment{proof}{{\bf Proof:}}{\hfill\rule{2mm}{2mm}}
\newtheorem{exmp}[theorem]{Ex}

% **** IF YOU WANT TO DEFINE ADDITIONAL MACROS FOR YOURSELF, PUT THEM HERE:

\begin{document}

% Lecture Details
\lecture{15}{Asst. Prof. M. Mert Ankarali}

%%%%%%%%%%%%%%%%%%%%%%%%%%

\section{Poles \& Zeros of MIMO Systems}

\subsection{Poles \& Zeros of SISO Systems}

Let $G(s)$ (or $G(z)$ in DT case) and $\left( \begin{array}{c|c} A & B \\ \hline C & D  \end{array} \right)$ are the transfer function and a \textit{minimal} state-space representation of a SISO LTI system.

$p_0$ is a pole of the system if 
%
\begin{itemize}
 \item $\lim_{s \to p_0}G(s) = \infty$
 \item $p_0$ is an eigenvalue of $A$
\end{itemize}
%
whereas
$z_0$ is a pole of the system if 
%
\begin{itemize}
 \item $\lim_{s \to z_0}G(s) = 0$
 \item steady-state part of the zero state response to $u(t) = e^{z_0 t}$
 \begin{align*}
 y_{ss}(t) = C (z_0 I - A)^{-1} B e^{z_0 t} = 0 
 \end{align*} 
\end{itemize}
%

\subsection{Poles of MISO Systems}

Unlike MIMO zeros, definition and derivation of MIMO poles is much more straightforward 

Let $G(s)$ (or $G(z)$ in DT case) and $\left( \begin{array}{c|c} A & B \\ \hline C & D  \end{array} \right)$ are the transfer function matrix and a \textit{minimal} state-space representation of a MIMO LTI system.

$p_0$ is a pole of the system if 
%
\begin{itemize}
 \item $\exists (i,j)$ s.t. $\lim_{s \to p_0}G_{ij}(s) = \infty$
 \item $\lim_{s \to p_0} || G(s) || = \infty$
 \item $p_0$ is an eigenvalue of $A$
\end{itemize}
%
in other words in the context of transfer function matrix $p_0$ is a pole of the system if it is a pole of any entry of
$G(s)$. Understanding and derivation of the multiplicities of a pole based on transfer function matrix is a little bit tricky 
and not very intuitive for the context and scope of the class. Thus, we can simply state that we can find the algebraic and
geometric multiplicity of a pole based on Jordan decomposition of $A$ provided that state-space representation is minimal. 
%
\begin{exmp}
	\begin{align*}
	G(s) = \left[ \begin{array}{ccc} \frac{s+1}{(s+2)^2} & 0 & 0 \\  
	0 & \frac{s}{(s+1)(s+2)} & 0 \\ 0 & 0 & 1  \end{array} \right]
	\end{align*}
\end{exmp}
%
We can clearly see that $p_1 = -2$ and $p_2 = -2$ are the poles of the system. Note that $z = -1$ also a zero of the 
firs entry of the transfer function matrix (and indeed it is a zero of the system). This states that a MIMO system can have 
a pole and a zero at the same location. 

\subsection{Zeros of MISO Systems}

If we generalised the definition of zero in SISO systems to the MIMO systems using the same logic with MIMO poles, we can see that
resultant definition would not be useful and somewhat unsatisfactory. Let's consider the transfer function matrix in the previous example. 
In this example all non-diagonal elements are zero and hence if we define a zero such that $\exists (i,j)$ s.t. $\lim_{s \to z_0}G_{ij}(s) = 0$,
then any $z_0 \in \mathbb{C}$ would be a zero of the system. In that context, we need more useful and satisfactory definition(s) for MIMO zeros.

\textbf{Definition:} Let $z_0 \neq p_i , \, \forall p_i \in \mathcal{P}$ , where $\mathcal{P}$ is teh set of all poles of the MIMO system $G(s)$ with $m$ inputs and
$q$ outputs. $z_0$ is a zero of the system if $H(s)$ drops rank at $s = z_0$.

Naturally let's assume that $G(s)$ is full rank, then we can also observe that
\begin{itemize}
\item if $m \leq p$, $G(s)$ drops rank at $s = z_0$ $\iff$ $\exists u_0 \in \mathbb{R}^{m} \, \& \, u_0 \neq 0$ such that $H(z_0) u_0 = 0$
\item if $q \leq m$, $G(s)$ drops rank at $s = z_0$ $\iff$ $\exists u_0^T \in \mathbb{R}^{q} \, \& \, u_0 \neq 0$ such that $u_0^T H(z_0) = 0$
\end{itemize}

These definitions provide us the zeros that are different from poles of the system. 

\begin{exmp}
	\begin{align*}
	G(s) = \left[ \begin{array}{ccc} \frac{s+1}{(s+2)^2} & \frac{s+3}{s+4}  \\  
	0 & \frac{s}{s+2}  \end{array} \right]
	\end{align*}
\end{exmp}

In this example we can see that $G(s)$ is full rank for \textit{most} $s \in \mathbb{C}$. The matrix drops rank at locations
$z_0 \in \lbrace 0 , -1 \rbrace$, however rank is preserved for $-3$ even if $G_{12}(-3) = 0$. Thus zeros of the system 
are $\lbrace 0 , -1 \rbrace$.







% **** This ENDS THE EXAMPLES. DON'T DELETE THE FOLLOWING LINE:
\end{document}
