% 


\documentclass[twoside]{article}
\setlength{\oddsidemargin}{0.25 in}
\setlength{\evensidemargin}{-0.25 in}
\setlength{\topmargin}{-0.6 in}
\setlength{\textwidth}{6.5 in}
\setlength{\textheight}{8.5 in}
\setlength{\headsep}{0.75 in}
\setlength{\parindent}{0 in}
\setlength{\parskip}{0.1 in}

%
% ADD PACKAGES here:
%

\usepackage{amsmath,amsfonts,graphicx,arydshln}


\newcounter{lecnum}
\renewcommand{\thepage}{\thelecnum-\arabic{page}}
\renewcommand{\thesection}{\thelecnum.\arabic{section}}
\renewcommand{\theequation}{\thelecnum.\arabic{equation}}
\renewcommand{\thefigure}{\thelecnum.\arabic{figure}}
\renewcommand{\thetable}{\thelecnum.\arabic{table}}

%
% The following macro is used to generate the header.
%
\newcommand{\lecture}[4]{
   \pagestyle{myheadings}
   \thispagestyle{plain}
   \newpage
   \setcounter{lecnum}{#1}
   \setcounter{page}{1}
   \noindent
   \begin{center}
   \framebox{
      \vbox{\vspace{2mm}
    \hbox to 6.28in { {\bf EE502 - Linear Systems Theory II
	\hfill Spring 2023} }
       \vspace{4mm}
       \hbox to 6.28in { {\Large \hfill Lecture #1 \hfill} }
       \vspace{2mm}
       \hbox to 6.28in { {\it Lecturer: #2 \hfill } }
      \vspace{2mm}}
   }
   \end{center}
   \markboth{Lecture #1}{Lecture #1}

   \vspace*{4mm}
}

\renewcommand{\cite}[1]{[#1]}
\def\beginrefs{\begin{list}%
        {[\arabic{equation}]}{\usecounter{equation}
         \setlength{\leftmargin}{2.0truecm}\setlength{\labelsep}{0.4truecm}%
         \setlength{\labelwidth}{1.6truecm}}}
\def\endrefs{\end{list}}
\def\bibentry#1{\item[\hbox{[#1]}]}


\newcommand{\fig}[3]{
			\vspace{#2}
			\begin{center}
			Figure \thelecnum.#1:~#3
			\end{center}
	}

% Use these for theorems, lemmas, proofs, etc.
\newtheorem{theorem}{Theorem}[lecnum]
\newtheorem{lemma}[theorem]{Lemma}
\newtheorem{proposition}[theorem]{Proposition}
\newtheorem{claim}[theorem]{Claim}
\newtheorem{corollary}[theorem]{Corollary}
\newtheorem{definition}[theorem]{Definition}
\newenvironment{proof}{{\bf Proof:}}{\hfill\rule{2mm}{2mm}}
\newtheorem{exmp}[theorem]{Ex}

% **** IF YOU WANT TO DEFINE ADDITIONAL MACROS FOR YOURSELF, PUT THEM HERE:

\begin{document}

% Lecture Details
\lecture{15}{Asst. Prof. M. Mert Ankarali}

%%%%%%%%%%%%%%%%%%%%%%%%%%

\section{Poles \& Zeros of MIMO Systems}

\subsection{Poles \& Zeros of SISO Systems}

Let $G(s)$ (or $G(z)$ in DT case) and $\left( \begin{array}{c|c} A & B \\ \hline C & D  \end{array} \right)$ are the transfer function and a \textit{minimal} state-space representation of a SISO LTI system.

$p_0$ is a pole of the system if 
%
\begin{itemize}
 \item $\lim_{s \to p_0}G(s) = \infty$
 \item $p_0$ is an eigenvalue of $A$
\end{itemize}
%
whereas
$z_0$ is a pole of the system if 
%
\begin{itemize}
 \item $\lim_{s \to z_0}G(s) = 0$
 \item $\exists x_0$ such that output response to $u(t) = e^{z_0 t}$ and $x(t) = x_0$ satisfies $y(t) = 0$
\end{itemize}
%
Let's remember the structure of the response for input types in the form of $u(t) = e^{z_0 t}$
%
\begin{align*}
  y(t) &= \underbrace{C e^{A t} \left[ \bar{x}_0 + x_0 \right] }_{transient}  \, + \, \underbrace{G(s_0) e^{s_0 t}}_{steady-state} \, , \, \mathrm{where}
   \\
   \bar{x}_0 &=  \left[ - (z_o I - A)^{-1} B \right] \ , \ 
   G(z_0) = C (z_0 I - A)^{-1} B + D 
\end{align*}
%
Thus $z_0$ is a zero if 
\begin{align*}
   x_0 &= -\bar{x}_0 = (z_o I - A)^{-1} B  \\ 
   G(z_0) &= C (z_0 I - A)^{-1} B + D = 0
\end{align*}

Note that these definitions and conditions directly hold for DT SISO LTI systems.

\newpage

\subsection{Poles of MIMO Systems}

Unlike MIMO zeros, the definition and derivation of MIMO poles are much more straightforward. 

Let $G(s)$ (or $G(z)$ in DT case) and $\left( \begin{array}{c|c} A & B \\ \hline C & D  \end{array} \right)$ are the transfer function matrix and a \textit{minimal} state-space representation of a MIMO LTI system.

$p_0$ is a pole of the system if 
%
\begin{itemize}
 \item $\exists (i,j)$ s.t. $\lim_{s \to p_0}G_{ij}(s) = \infty$
 \item $\lim_{s \to p_0} || G(s) || = \infty$
 \item $p_0$ is an eigenvalue of $A$
\end{itemize}
%
in other words, in the context of the transfer function matrix, $p_0$ is a pole of the system if it is a pole of any entry of
$G(s)$. Understanding and derivation of the multiplicities of a pole based on the transfer function matrix is a little bit tricky 
and not very intuitive for the context and scope of the class. Thus, we can simply state that we can find a pole's algebraic and
geometric multiplicity based on Jordan decomposition of $A$ provided that state-space representation is minimal. 
%
\begin{exmp}
	\begin{align*}
	G(s) = \left[ \begin{array}{ccc} \frac{s+1}{(s+2)^2} & 0 & 0 \\  
	0 & \frac{s}{(s+1)(s+2)} & 0 \\ 0 & 0 & 1  \end{array} \right]
	\end{align*}
\end{exmp}
%
We can clearly see that $p_1 = -2$ and $p_2 = -2$ are the system's poles. Note that $z = -1$ is also a zero of the 
first entry of the transfer function matrix (and indeed, it is a zero of the system). This states that a MIMO system can have 
a pole and a zero at the exact location. 

\subsection{Zeros of MIMO Systems}

Suppose we generalize the definition of a zero in SISO systems to the MIMO systems using the same logic with MIMO poles. In that case, we can see that the resultant definition would not be helpful and somewhat unsatisfactory. Let's consider the transfer function matrix in the previous example. In this example, all non-diagonal elements are zero, and hence if we define a zero such that $\exists (i,j)$ s.t. $\lim_{s \to z_0}G_{ij}(s) = 0$, then any $z_0 \in \mathbb{C}$ would be a zero of the system. In that context, we need a more useful and satisfactory definition(s) for MIMO zeros.

\textbf{Definition:} Let $z_0 \neq p_i , \, \forall p_i \in \mathcal{P}$ , where $\mathcal{P}$ is teh set of all poles of the MIMO system $G(s)$ with $m$ inputs and
$q$ outputs. $z_0$ is a zero of the system if $H(s)$ drops rank at $s = z_0$.

Naturally let's assume that $G(s)$ is full rank, then we can also observe that
\begin{itemize}
\item if $m \leq p$, $G(s)$ drops rank at $s = z_0$ $\iff$ $\exists u_0 \in \mathbb{R}^{m} \, \& \, u_0 \neq 0$ such that $G(z_0) u_0 = 0$
\item if $q \leq m$, $G(s)$ drops rank at $s = z_0$ $\iff$ $\exists w_0^T \in \mathbb{R}^{q} \, \& \, w_0 \neq 0$ such that $w_0^T G(z_0) = 0$
\end{itemize}

These definitions provide us the zeros different from the system's poles. $u_0$ and $w_0$ are called the direction of MIMO zeros in their respective cases. 
%
\begin{exmp}
	\begin{align*}
	G(s) = \left[ \begin{array}{ccc} \frac{s+1}{(s+2)^2} & \frac{s+3}{s+4}  \\  
	0 & \frac{s}{s+2}  \end{array} \right]
	\end{align*}
\end{exmp}
%
\textbf{Solution:} In this example, we can see that $G(s)$ is full rank for \textit{most} $s \in \mathbb{C}$. The matrix drops rank at locations
$z_0 \in \lbrace 0 , -1 \rbrace$, however rank is preserved for $-3$ even if $G_{12}(-3) = 0$. Thus zeros of the system 
are $\lbrace 0 , -1 \rbrace$.

Let's solve some non-square examples.
%
\begin{exmp}
	\begin{align*}
	G(s) = \left[ \begin{array}{ccc} \frac{s+1}{(s+2)^2} & \frac{s+3}{s+4}  \\  
	0 & \frac{s}{s+2}  
        \\
        \frac{s(s+3)}{(s+4)^2}  & 0
        \end{array} \right]
	\end{align*}
\end{exmp}
%
\textbf{Solution:} Let's test if $s = -1$ is a zero and if it is let's find $u_{-1}$ such that $G(-1) u_{-1} = 0$
%
\begin{align*}
	G(-1) = \left[ \begin{array}{ccc} 0 & \frac{2}{3}  \\  
	0 & -1  
        \\
        \frac{-2}{9}  & 0
        \end{array} \right] \, \Rightarrow \, \mathrm{rank}[G(-1)] = 2  
\end{align*}
%
Thus for this system, $s = -1$ is not a zero. Let's test if $s = -3$ is a zero and if it is let's find $u_{-3}$ such that $G(-3) u_{-3} = 0$
%
\begin{align*}
	G(-3) = \left[ \begin{array}{ccc} -2 & 0 \\  
	0 & 3  
        \\
        0  & 0
        \end{array} \right] \, \Rightarrow \, \mathrm{rank}[G(-3)] = 2  
\end{align*}
%
Thus, also for this system, $s = -3$ is not a zero. Let's test if $s = 0$ is a zero and if it is let's find $u_{0}$ such that $G(0) u_{0} = 0$
%
\begin{align*}
	G(0) = \left[ \begin{array}{ccc} \frac{1}{4} & \frac{3}{4}  \\  
	0 & 0
        \\
        0  & 0
        \end{array} \right] \, \Rightarrow \, \mathrm{rank}[G(0)] = 1  
        \\
        u_0 = \begin{bmatrix} -12 \\ 4 \end{bmatrix} \, \Rightarrow \, G(0) u_0 = \begin{bmatrix} 0 \\ 0 \end{bmatrix}
\end{align*}
%
Thus $s = 0$ is a zero of the system and \textit{direction} of the zero is $u_0 = \begin{bmatrix} -12 \\ 4 \end{bmatrix} $
%
\begin{exmp}
	\begin{align*}
	G(s) = \left[ \begin{array}{ccc} \frac{s+1}{(s+2)^2} & \frac{s+3}{s+4} & 0 \\  
	0 & \frac{s}{s+2} & \frac{(s+1)(s+3)}{(s-1)^2}
        \end{array} \right]
	\end{align*}
\end{exmp}
%
\textbf{Solution:} Let's test if $s = -1$ is a zero and if it is let's find $w_{-1}$ such that $w_{-1}^T G(-1) = 0$
%
\begin{align*}
	G(-1) &= \left[ \begin{array}{ccc} 0 & \frac{2}{3} & 0 \\  
	0 & -1 & 0
        \end{array} \right] \, \Rightarrow \, \mathrm{rank}[G(-1)] = 1
        \\
        w_{-1} &= \begin{bmatrix} 3 \\ 1 \end{bmatrix} \, \Rightarrow \, w_{-1}^T G(-1) = \begin{bmatrix} 0 & 0 \end{bmatrix}
\end{align*}
%
Thus $s = -1$ is a zero of the system and \textit{direction} of the zero is $w_{-1} = \begin{bmatrix} 3 \\ 1 \end{bmatrix} $.
Now let's test if $s = -3$ and $s=0$ are zeros f of the system.
%
\begin{align*}
	G(-3) &= \left[ \begin{array}{ccc} -2 & 0 & 0 \\  
	0 & 3 & 0
        \end{array} \right] \, \Rightarrow \, \mathrm{rank}[G(-3)] = 2
        \\
        G(0) &= \left[ \begin{array}{ccc} \frac{1}{4} & \frac{3}{4} & 0 \\  
	0 & 0 & 3
        \end{array} \right] \, \Rightarrow \, \mathrm{rank}[G(0)] = 0
\end{align*}
%
Thus neither $s=-3$ nor $s=0$ is a zero of the system.

\vspace{12pt}

The major limitation of this definition is that it is limited to zeros that are different from the poles of the system. So we 
need a more general definition of zeros 

\vspace{12pt}

\textbf{Definition} Let's assume that $G(s)$ is a full rank (for \textit{most} $s \in \mathbb{C}$) transfer function matrix with $m$
inputs and $q$ outputs. $z_0 \in \mathbb{C}$ is a zero of the system if
\begin{itemize}
\item if $m \leq p$, $\exists u(s) \neq 0$ such that $\lim_{s \to z_0} [ G(s) u(s) ] = 0$
\item if $p \leq m$, $\exists w(s) \neq 0$ such that $\lim_{s \to z_0} [ w(s)^T G(s) ] = 0$
\end{itemize}
%
\begin{exmp}
	\begin{align*}
	G(s) = \left[ \begin{array}{ccc} 1 & \frac{s+1}{s+2}  \\  
	0 & 1  \end{array} \right]
	\end{align*}
\end{exmp}

Let $u(s) = \begin{bmatrix} 1 \\ s+2 \end{bmatrix}$, then
%
\begin{align*}
\lim_{s \to -2} [ G(s) u(s) ] = \lim_{s \to -2} \begin{bmatrix} s+2 \\ s+2 \end{bmatrix} = \begin{bmatrix} 0 \\ 0 \end{bmatrix}
\end{align*}
%
thus $s = -2$ is both a pole and zero location for $G(s)$. If you bother with the fact that selected $u(s)$ is non-causal you can simple multiple $u(s)$ with $\frac{1}{s+b}$ where $b$ is not a zero or pole of the system, then we will still have 
%
%
\begin{align*}
u_{causal}(s) &= \frac{u(s)}{s+b} = \begin{bmatrix} \frac{1}{s+b} \\ \frac{1}{(s+b)(s+2)} \end{bmatrix}
\\
\lim_{s \to -2} [ G(s) u_{causal}(s) ] = \lim_{s \to -2} \begin{bmatrix} \frac{s+2}{s+b} \\ \frac{s+2}{s+b} \end{bmatrix} = \begin{bmatrix} 0 \\ 0 \end{bmatrix}
\end{align*}
%


\textbf{Corollary:} Let $G(s)$ be square transfer function matrix and it is invertible for \textit{most} $s \in \mathbb{C}$, then zeros of the systems
are the poles of $G(s)^{-1}$. 

Let's analyze the previous example 

\begin{align*}
G(s)^{-1} &= \left[ \begin{array}{ccc} 1 & -\frac{s+1}{s+2}  \\  
	0 & 1  \end{array} \right]
	\\
\lim_{s \to -2} || G(s)^{-1} || &= \infty \, \Rightarrow \, z_0 = - 2 \ \mathrm{is} \, \mathrm{zero} \, \mathrm{of} \, G(s)
\end{align*}

\subsubsection{Determining Zeros from a Minimal Realization}

Given a minimal state realization of a MIMO system with quadruple $(A,B,C,D)$, $z_0$ is a zero
of the system if it drops rank of the system matrix given below
%
\begin{align*}
\mathbf{S} = \left[ \begin{array}{cc} s I - A & -B \\ C & D \end{array} \right]
\end{align*}
%
Notice that for a observable and reachable system $\mathbf{S}$ is full rank for \textit{most} $s \in \mathbb{C}$. 
Let's first verify that this condition correctly gives the zeros of a SISO system. Let's assume that 
$\mathbf{S}$ drops rank at $z_0$, then we know that $\exists v_0 = [x_0 \, u_0]$, where $x_0 \in \mathbb{R}^{n}$
and $u_0 \in \mathbb{R}$ such that 
%
\begin{align*}
\mathbf{S} v_0 &= 0
\, \Rightarrow \,
\left[ \begin{array}{cc} z_0 I - A & -B \\ C & D \end{array} \right] \left[ \begin{array}{c} x_0 \\ u_0 \end{array} \right] = 0
\, \Rightarrow \,
 \begin{array}{cc} \left[ z_0 I - A \right] x_0 - B u_0 = 0 \\ C x_0 + D u_0 = 0 \end{array}
\end{align*}
%
Nota that for SISO systems zeros can not be located at the pole locations, hence $ \left[ z_0 I - A \right] x_0 \neq 0$, which implies that $u_0 \neq 0$. Let $u_0 = 1$, then 
%
\begin{align*}
&x_0 =  \left[ z_0 I - A \right]^{-1} B  \\
&C \left[ z_0 I - A \right]^{-1} B  + D = G(z_0) =  0 
\end{align*}
%
Note that these are minimal state-space representation based the conditions that provide the zeros of a SISO system. In other words if $x(t) = x_0 = \left[ z_0 I - A \right]^{-1} B$ and $u(t) = e^{z_0 t}$, then $y(t) = 0 \, , \, \forall t \geq 0$. 

Now, let's verify that this condition correctly gives the zeros of a MISO system that are different from the poles. 

Let's first focus on the case $m \leq q$, where the number of the inputs is less then or equal to the number of the outputs. 
Let's assume that  $\mathbf{S}$ drops rank at $z_0$, then we know that $\exists v_0 = [x_0 \, u_0]$, where $x_0 \in \mathbb{R}^{n}$ and $u_0 \in \mathbb{R}^m$ such that 
%
%
\begin{align*}
\mathbf{S} v_0 &= 0
\, \Rightarrow \,
\left[ \begin{array}{cc} z_0 I - A & -B \\ C & D \end{array} \right] \left[ \begin{array}{c} x_0 \\ u_0 \end{array} \right] = 0
\, \Rightarrow \,
 \begin{array}{cc} \left[ z_0 I - A \right] x_0 - B u_0 = 0 \\ C x_0 + D u_0 = 0 \end{array}
 \\
 &x_0 =  \left[ z_0 I - A \right]^{-1} B u_0 \\
&\left[ C \left[ z_0 I - A \right]^{-1} B  + D \right] u_0 = G(z_0) u_0 =  0 
\end{align*}
%
Note that for a MIMO system with $m \leq q$, $z_0$ is a zero of the system that are different from the poles, 
if $\exists u_0 \in \mathbb{R}^m$ such that $G(z_0) u_0 $ which the condition that we derived above. Moreover, if we 
assume $x(0) = x_0$ and $u(t) = u_0 e^{z_0 t}$ we will see that $y(t) = 0 \, \forall t \geq 0$. This is indeed another more general definition of a MIMO zero (which also holds for zeros located at the poles of the system).

Let's focus on the case $q \leq m$, where the number of the outputs is less then or equal to the number of the inputs. 
Let's assume that  $\mathbf{S}$ drops rank at $z_0$, then we know that $\exists v_0^T = [x_0^T \, w_0^T]$, where $x_0 \in \mathbb{R}^{n}$ and $w_0 \in \mathbb{R}^q$ such that 
%
\begin{align*}
&v_0 ^T \mathbf{S} = 0
\, \Rightarrow \,
\left[ \begin{array}{cc} x_0^T & w_0^T \end{array} \right] \left[ \begin{array}{cc} z_0 I - A & -B \\ C & D \end{array} \right] = 0
\, \Rightarrow \,
 \begin{array}{cc} x_0^T \left[ z_0 I - A \right] + w_0^T C = 0 \\ x_0^T B + w_0^T D = 0 \end{array}
 \\
 &\left[ z_0 I - A \right]^T x_0 + C^T w_0 = 0 \, \Rightarrow \, x_0 =  \left( \left[ z_0 I - A \right]^T \right)^{-1} C^T w_0
\\
&x_0^T B + w_0^T D = 0 \, \Rightarrow \, B^T x_0 + D^T w_0 = 0  \, \Rightarrow \, 
\left[ B^T \left( \left[ z_0 I - A \right]^T \right)^{-1} C^T + D^T \right] w_0 = 0
\\
&G(z_0)^T w_0 =  0 \iff w_0^T G(z_0) = 0
\end{align*}
%
Note that for a MIMO system with $q \leq m$, $z_0$ is a zero of the system that are different from the poles, 
if $\exists w_0 \in \mathbb{R}^q$ such that $w_0^T G(z_0) $ which is the condition that we derived above. Moreover, if we 
assume $x(0) = x_0$ and $w(t) = w_0 e^{z_0 t}$ for a system where the state-space matrices has the form $(A^T,C^T,B^T,D^T)$, we will see that $y(t) = 0 \, \forall t \geq 0$. This is indeed another more general definition of a MIMO zero (which also holds for zeros located at the poles of the system).

\begin{exmp}
	\begin{align*}
	G(s) = \left[ \begin{array}{ccc} 1 & \frac{s+1}{s+2}  \\  
	0 & 1  \end{array} \right]
	\end{align*}
\end{exmp}

Previously we found that, $G(s)$ has a pole and a zero at $s = -2$. Let's verify this based on the
state-space condition. We can find a minimal realization using Gilbert's method 
\begin{align*}
     G(s) &= \left[ \begin{array}{cc} 1 & 1 - \frac{1}{s+2}  \\  
	0 & 1  \end{array} \right]
	= \left[ \begin{array}{cc} 1 & 1   \\  
	0 & 1  \end{array} \right] +  \left[ \begin{array}{cc} 0 & -1 \\  
	0 & 0  \end{array} \right] \frac{1}{s+1}
	= \left[ \begin{array}{cc} 1 & 1   \\  
	0 & 1  \end{array} \right] +  \left[ \begin{array}{c} -1 \\ 0 \end{array} \right]   
       \left[ \begin{array}{cc} 0 & 1 \end{array} \right]    \frac{1}{s+1}
       \\
       A &= -1 \, , \, B = \left[ \begin{array}{cc} 0 & 1 \end{array} \right] \ , \ C = \left[ \begin{array}{c} -1 \\ 0 \end{array} \right]
       \, , \, D = \left[ \begin{array}{cc} 1 & 1   \\  
	0 & 1  \end{array} \right]
\end{align*}
%
Let's construct the system matrix 
%
\begin{align*}
\mathbf{S} = \left[ \begin{array}{cc} s I - A & -B \\ C & D \end{array} \right]
= \left[ \begin{array}{ccc} s + 1 & 0 & - 1 \\ -1 & 1 & 1 \\ 0 & 0 & 1 \end{array} \right]
\, \Rightarrow \, \mathrm{rank}S(s)_{s-1} = 2 
\end{align*}
%
which verifies that system has a single zero at $s = -2$. Now let's find the direction of the zero 
%
\begin{align*}
&\mathbf{S}(-2) \left[ \begin{array}{c} x_0 \\ u_0 \end{array} \right] = 0
\\
&\left[ \begin{array}{ccc} 0 & 0 & - 1 \\ -1 & 1 & 1 \\ 0 & 0 & 1 \end{array} \right] \left[ \begin{array}{c} x_0 \\ \hline u_0 \end{array} \right] = 0 \, \Rightarrow \, x_0 = 1 \, , \, u_0 
 \left[ \begin{array}{c} 1 \\  0 \end{array} \right] 
\end{align*}
%

% **** This ENDS THE EXAMPLES. DON'T DELETE THE FOLLOWING LINE:
\end{document}
