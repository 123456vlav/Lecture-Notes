% 


\documentclass[twoside]{article}
\setlength{\oddsidemargin}{0.25 in}
\setlength{\evensidemargin}{-0.25 in}
\setlength{\topmargin}{-0.6 in}
\setlength{\textwidth}{6.5 in}
\setlength{\textheight}{8.5 in}
\setlength{\headsep}{0.75 in}
\setlength{\parindent}{0 in}
\setlength{\parskip}{0.1 in}

%
% ADD PACKAGES here:
%

\usepackage{amsmath,amsfonts,graphicx}


\newcounter{lecnum}
\renewcommand{\thepage}{\thelecnum-\arabic{page}}
\renewcommand{\thesection}{\thelecnum.\arabic{section}}
\renewcommand{\theequation}{\thelecnum.\arabic{equation}}
\renewcommand{\thefigure}{\thelecnum.\arabic{figure}}
\renewcommand{\thetable}{\thelecnum.\arabic{table}}

%
% The following macro is used to generate the header.
%
\newcommand{\lecture}[4]{
   \pagestyle{myheadings}
   \thispagestyle{plain}
   \newpage
   \setcounter{lecnum}{#1}
   \setcounter{page}{1}
   \noindent
   \begin{center}
   \framebox{
      \vbox{\vspace{2mm}
    \hbox to 6.28in { {\bf EE402 - Discrete Time Systems
	\hfill Spring 2018} }
       \vspace{4mm}
       \hbox to 6.28in { {\Large \hfill Lecture #1 \hfill} }
       \vspace{2mm}
       \hbox to 6.28in { {\it Lecturer: #2 \hfill } }
      \vspace{2mm}}
   }
   \end{center}
   \markboth{Lecture #1}{Lecture #1}

   \vspace*{4mm}
}

\renewcommand{\cite}[1]{[#1]}
\def\beginrefs{\begin{list}%
        {[\arabic{equation}]}{\usecounter{equation}
         \setlength{\leftmargin}{2.0truecm}\setlength{\labelsep}{0.4truecm}%
         \setlength{\labelwidth}{1.6truecm}}}
\def\endrefs{\end{list}}
\def\bibentry#1{\item[\hbox{[#1]}]}


\newcommand{\fig}[3]{
			\vspace{#2}
			\begin{center}
			Figure \thelecnum.#1:~#3
			\end{center}
	}

% Use these for theorems, lemmas, proofs, etc.
\newtheorem{theorem}{Theorem}[lecnum]
\newtheorem{lemma}[theorem]{Lemma}
\newtheorem{proposition}[theorem]{Proposition}
\newtheorem{claim}[theorem]{Claim}
\newtheorem{corollary}[theorem]{Corollary}
\newtheorem{definition}[theorem]{Definition}
\newenvironment{proof}{{\bf Proof:}}{\hfill\rule{2mm}{2mm}}
\newtheorem{exmp}[theorem]{Ex}

% **** IF YOU WANT TO DEFINE ADDITIONAL MACROS FOR YOURSELF, PUT THEM HERE:

\begin{document}

% Lecture Details
\lecture{10}{Asst. Prof. M. Mert Ankarali}


%%%%%%%%%%%%%%%%%%%%%%%%%%

\section{Reachability \& Controllability of DT-LTI Systems}

For LTI a discrete time state-space representation
%
\begin{align*}
  x[k+1] &= A x[k] + B u[k]
\\
  y[k] &= C x[k] + D u[k]
\end{align*}
%
\begin{itemize}
  \item A state $x_r$ is said to be $m$-step \textbf{reachable}, if there exist
  an input sequence, $u[k] , k \in \lbrace 0 , 1, \cdots m-1 \rbrace$, that transfers the state vector 
  $x[k]$ from the origin (i.e. $x[0] = 0$) to the state $x_r$ in m 
  number of steps, i.e. $x[m] = x_r$.
 
  \item A state $x_d$ is said to be $m$-step \textbf{controllable},
  if there exist an input sequence, $u[k] , k \in \lbrace 0 , 1, \cdots m-1 \rbrace$, that transfers the state vector 
  $x[k]$ from the initial state $x_c$ (i.e. $x[0] = x_c$) to the origin
  in $m$ number of steps, i.e. $x[m] = 0$.
\end{itemize}

Note that
\begin{itemize}
  \item the set $\mathcal{R}_m$ of all m-step reachable states is a linear
(sub)space: $\mathcal{R}_m \subset \mathbb{R}^n$
  \item the set $\mathcal{C}_m$ of all m-step controllable states is a linear
(sub)space: $\mathcal{C}_m \subset \mathbb{R}^n$
\end{itemize}

Let's characterize $\mathcal{R}_m$ and then try to generalize the reachability concept. 
When $x[0] = 0$, the solution of $x[m]$ is given by

\begin{align*}
  x[m] &= \left[ \begin{array}{c|c|c|c|c} A^{m-1} B & A^{m-2} B &
         \cdots & A B & B \end{array} \right]
         \left[ \begin{array}{c}
                  u[0] \\ u[1] \\ \vdots \\ u[m-2] \\ u[m-1]
         \end{array} \right]
\end{align*}

Let 
\begin{align*}
 \mathbf{R}_m &= 
        \left[ \begin{array}{c|c|c|c|c} A^{m-1} B & A^{m-2} B &
         \cdots & A B & B \end{array} \right]
\\
 \mathbf{U}_m &=
         \left[ \begin{array}{c}
                  u[0] \\ u[1] \\ \vdots \\ u[m-2] \\ u[m-1]
         \end{array} \right]
\end{align*}
%
then if a state $x_r$ is reachable at $k$ steps, it should 
satisfy the following equation for some $\mathbf{U}_m$.
%
\begin{align*}
 \mathbf{M}_m \mathbf{U}_m &= x_m
\end{align*}
%
In order this matrix equation to have a solution $x_r$
should be in the range space of $\mathbf{M}_m$.
%
\begin{align*}
  x_r \in \mathrm{Ra} ( \mathbf{M}_m ) 
\end{align*}
%
Thus $m$-step reachable sub-space is simply equal to range space of $\mathcal{R}_k$
\begin{align*}
  \mathrm{Ra} ( \mathbf{R}_m ) = \mathcal{R}_m
\end{align*}

\textbf{Theorem:} For $k < n < l$
%
\begin{align*}
 \mathcal{R}_k \subset \mathcal{R}_{n} &= \mathcal{R}_{l} 
  \\
  \mathrm{Ra} ( \mathbf{R}_k ) \subset \mathrm{Ra} ( \mathbf{R}_{n} ) &= \mathrm{Ra} ( \mathbf{R}_{l} )
\end{align*}
%
\textbf{Proof:} It is fairly easy to observe that
%
\begin{align*}
\mathcal{R}_i & \subset \mathcal{R}_{i+1} 
\\
\mathrm{Ra} ( \mathbf{R}_i ) & \subset \mathrm{Ra} ( \mathbf{R}_{i+1} )
\end{align*}
%
since we add a new column (or columns for multi-input systems) to $ \mathbf{R}_i$, thus it can only increase the dimension of the range-space. Thus we can conclude that 
%
%
\begin{align*}
 \mathcal{R}_k &\subset \mathcal{R}_{n} \subset \mathcal{R}_{l} 
  \\
  \mathrm{Ra} ( \mathbf{R}_k ) &\subset \mathrm{Ra} ( \mathbf{R}_{n} ) \subset \mathrm{Ra} ( \mathbf{R}_{l} )
\end{align*}
%
In order prove $\mathcal{R}_{n} = \mathcal{R}_{l} $, we simply use the Cayley-Hamilton theorem. 
Based on Cayley-Hamilton theorem 
%
\begin{align*}
 A^n &= -a_1 A^{n-1} - \cdots - a_{n-1} A - a_n I
 \\
 A^n B &= -a_1 A^{n-1} B - \cdots - a_{n-1} A B - a_n B
\end{align*}
%
which shows that $ A^n B$ is linearly dependent to previous columns and thus 
%
%
\begin{align*}
\mathcal{R}_{n} =& \mathcal{R}_{l} 
  \\
\mathrm{Ra} ( \mathbf{R}_{n} ) &= \mathrm{Ra} ( \mathbf{R}_{l} )
\end{align*}
%
This theorem shows that if $x_r$ is reachable in $n$ steps then it is reachable for $l > n$ steps, similarly, if it is not reachable in 
$n$ steps then it is reachable for $l > n$ steps. In this context, the sub-space of states reachable in $n$-steps, $\mathcal{R}_n$
is referred as the reachable subspace of $(A,N)$, and will be denoted simply by $\mathcal{R}$ and $\mathbf{R} = \mathbf{R}_k$ will be system 
wide the reachability matrix. The system is termed a (fully) reachable system if 
%
\begin{align*}
 \mathrm{rank} (\mathbf{R}) &= n 
 \\
 \mathrm{Ra} (\mathbf{R}) &= \mathcal{R} = \mathbb{R}^n
\end{align*}

\begin{exmp}
Solve the following problems regarding controllable sub-space
\end{exmp}
\begin{itemize}
    \item Show that $\mathcal{R} \subset \mathcal{C}$, $\forall (A,B)$, however $\mathcal{C} \subset \mathcal{R}$ not necessarily true $\forall (A,B)$.
    \item Similar to the reachable subspace, characterize the controllable subspace
    \item Derive conditions such that $\mathcal{R} = \mathcal{C}$
\end{itemize}

\subsection{Reachability Gramian}

An alternative characterization of $\mathbf{R}$ is using reachability Gramian (which is more critical for CT systems).
$m$-step reachability Gramian, $\mathbf{P}_m$, is defined as
%
\begin{align}
 \mathbf{P}_m &= \mathbf{R}_m {\mathbf{R}_m}^T =\sum\limits_{i=0}^{k-1} A^i B B^T \begin{pmatrix} A^T \end{pmatrix}^i
\end{align}
%
Note that $\mathcal{P}_m$ is a symmetric positive semi-definite matrix.

\textbf{Lemma:} $\mathcal{R}_m = \mathrm{Ra}( \mathbf{R}_m ) = \mathrm{Ra} ( \mathbf{P}_m )$

\textbf{Proof:} Let's fits show that $\mathrm{Ra}( \mathbf{P}_m ) \subset \mathrm{Ra}( \mathbf{R}_m )$. 
If $x \in \mathrm{Ra} (\mathbf{P}_m)$, then $\exists v \in \mathbb{R}^n$ s.t. $x = \mathbf{P}_m v$ then
%
\begin{align*}
 x = \mathbf{P}_m v &= \mathbf{R}_m {\mathbf{R}_m}^T v = \mathbf{R}_m y \ \Rightarrow \ x \in \mathrm{Ra}( \mathbf{R}_m ) \ \Rightarrow \ \mathrm{Ra}( \mathbf{P}_m ) \subset \mathrm{Ra}( \mathbf{R}_m )
\end{align*}

Now let's show that $\mathrm{Ra}( \mathbf{R}_m ) \subset \mathrm{Ra}( \mathbf{P}_m )$. We know that
%
\begin{align*}
    \mathrm{Ra}( \mathbf{R}_m ) \subset \mathrm{Ra}( \mathbf{P}_m ) 
    \ \iff \ \mathrm{Ra}^\perp( \mathbf{P}_m ) \subset \mathrm{Ra}^\perp( \mathbf{R}_m )
\end{align*}
%
So we can equivalently show that $\mathrm{Ra}^\perp( \mathbf{P}_m ) \subset \mathrm{Ra}^\perp( \mathbf{R}_m )$.
Let $q \in \mathrm{Ra}^\perp( \mathbf{P}_m )$, then
%
%
\begin{align*}
    q^T \mathbf{P}_m = \mathbf{0} \ \Rightarrow \ q^T \mathbf{P}_m q = 0 \ &\iff \ q^T \mathbf{R}_m {\mathbf{R}_m}^T q = 0
    \ \iff \ ( {\mathbf{R}_m}^T q )^T ( {\mathbf{R}_m}^T q ) = 0
    \\ \ &\iff \ {\mathbf{R}_m}^T q = \mathbf{0} \ \iff \ q^T {\mathbf{R}_m} = \mathbf{0}^T = \mathbf{0}  
    \\ \ &\Rightarrow \ q \in \mathrm{Ra}^\perp( \mathbf{R}_m ) \ \Rightarrow \ \mathrm{Ra}^\perp( \mathbf{P}_m ) \subset \mathrm{Ra}^\perp( \mathbf{R}_m )
\end{align*}
%
This completes the proof. As a result of this lemma, full reachable subspace $\mathcal{R} = \mathrm{Ra}( \mathbf{P}_l )$ for any $l \geq 0$.
As a result we can make the following conclusions 
\begin{itemize}
\item $(A,B)$ pair is fully reachable $\iff$ $\mathrm{dim} \left [\mathrm{Ra}( \mathbf{P}_l ) \right] = n$ for any $l \geq n$
\item $(A,B)$ pair is fully reachable $\iff$ $\mathrm{det} \left [\mathbf{P}_l\right] \neq 0$ for any $l \geq n$
\end{itemize}
% 
If $x[k+1] = A x[k]$ is asymptotically stable, then $\textbf{P}_{\infty} = \lim_{k \to \infty} \sum\limits_{i=0}^{k-1} A^i B B^T \begin{pmatrix} A^T \end{pmatrix}^i \stackrel{\triangle}{=} P$ is well defined and $P$ satisfies the following Lyapunov equation
%
\begin{align*}
 A P A^T - P = -B B^T
\end{align*}
%
To understand this derivation, refer to the Quadratic Lyapunov Functions for LTI systems section in Lecture Notes 8. 
%
\begin{exmp}
 Let $x[k+1] = Ax [k]$ be asymptotically stable. Then, show that  
\end{exmp}
%
\begin{align*}
 A P A^T - P = -B B^T
\end{align*}
%
has a unique positive definite solution of $P$, if and only if, $(A,B)$ pair is fully reachable.

\subsection{Modal Aspects and Modal Reachability Tests}

\textbf{Lemma:} The reachable sub-space, $\mathcal{R}$ is $A$ invariant, i.e. $x\in \mathcal{R} \ \Rightarrow \ A x \in \mathcal{R}$. We 
write this as $A \mathcal{R} \subset \mathcal{R}$.

\textbf{Proof:} Let $x \in \mathcal{R}$ then $\exists \textbf{U}_n \in \mathbb{R}^{n p}$, s.t. $\ x = \mathbf{R} \textbf{U}_n 
 \ \mathrm{where} \ B \in \mathbb{R}^{n \times p}$, then
%
\begin{align*}
 x = \left[ \begin{array}{c|c|c|c|c} A^{n-1} B & A^{n-2} B &
         \cdots & A B & B \end{array} \right] \textbf{U}_n 
\end{align*}
%
Now let's expand $A x$
%
\begin{align*}
 A x = \left[ \begin{array}{c|c|c|c|c} A^{n} B & A^{n-1} B &
         \cdots & A^2 B & A B \end{array} \right] \textbf{U}_n 
\end{align*}
%
Using Cayley-Hamilton theorem we reach that
%
%
\begin{align*}
 A^n &= -a_{n-1} A^{n-1} - \cdots - a_1 A - a_0 I
 \\
 A^n B &= -a_{n-1} A^{n-1} B - \cdots - a_1 A B - a_0 B
 \\
 A x &= \left[ \begin{array}{c|c|c|c|c} \sum\limits_{i=0}^{n-1} A^{i} B & A^{n-2} B &
         \cdots & A^2 B & A B \end{array} \right] \textbf{U}_n 
 \\
 A x &\in \mathrm{Span} \left\lbrace \begin{array}{ccccc} A^{n-1} B , & A^{n-2} B , &
         \cdots & A B , & B \end{array} \right\rbrace = \mathcal{R}
\end{align*}
%
\textbf{Theorem:} $(A,B)$ pair is unreachable (or not fully reachable) if and only if $w^T B = 0$ for some left eigenvector of $A$, i.e. 
$w^T A = \lambda w^T$.

\textbf{Proof:} Let $w^T A = \lambda w^T$ and $w^T B = 0$, then
%
\begin{align*}
 w^T \mathbf{R} &= w^T \left[ \begin{array}{c|c|c|c|c} A^{n-1} B & A^{n-2} B &
         \cdots & A B & B \end{array} \right] 
         \\
  &= \left[ \begin{array}{c|c|c|c|c} w^T A^{n-1} B & w^T A^{n-2} B &
         \cdots & w^T A B & w^T B \end{array} \right] 
         \\
&= \left[ \begin{array}{c|c|c|c|c} \lambda^{n-1} w^T B & \lambda^{n-2} w^T B &
         \cdots & \lambda w^T  B & w^T B \end{array} \right] = 0 \ \rightarrow \ (A,B) \ \mathrm{unreachable} 
\end{align*}
%
Now let's show that if $(A,B)$ is not reachable, then $\exists w$ s.t. $w^T A = \lambda w^T$ and $w^T B = 0$.  if $(A,B)$ is not reachable, then there exist $q \in \mathbb{R}^n$ such that
%
\begin{align*}
q^T \mathbf{R} &= 0 \ \Rightarrow \ \mathbf{R}^T q = 0 
\ \Rightarrow \ \begin{array}{c} B^T (A^T)^{n-1} q = 0 & B^T (A^T)^{n-2} q = 0 &
         \cdots & B^T A^T q = 0 & B^T q = 0 \end{array} 
\end{align*}
%
Define $\mathcal{S} = \mathrm{Span} \left\lbrace q \ , \ A^T q \ , \ \cdots \ , \  (A^T)^{n-1} q \right\rbrace$, note that 
$\mathcal{S} \subset \mathcal{N}(B^T)$ and $\mathcal{S}$ is invariant under $A^T$ (see the similar proof above), i.e. 
If $v \in \mathcal{S}$, then $A^T v \in \mathcal{S}$. Let $\mathrm{dim} (\mathcal{S}) = j$ and $V \in \mathbb{R}^{n \times j}$ 
be a matrix whose columns form a basis for $\mathcal{S}$. Since $\mathcal{S}$ is invariant under $A^T$, we can find a transformation
matrix, $\Gamma \in \mathbb{R}^{j \times j}$ such that 
%
\begin{align*}
A^T V = V \Gamma
\end{align*}
%
Let $\nu$ be an eigenvector of $V$, i.e. $\Gamma \nu = \alpha \nu , \, \alpha \in \mathcal{C}$, then 
%
\begin{align*}
A^T V \nu = V \Gamma \nu = \alpha V \nu \ \rightarrow \ z = V \nu \in \mathbb{C}^n
\end{align*}
%
where $z$ is an eigenvector of $A^T$, moreover $z \in \mathcal{S} = \mathcal{N}(B^T)$. This completes the proof. 

\textbf{Theorem:} \textit{PBH Rechability Test} - $(A,B)$ is reachable if and only if 
%
\begin{align*}
    \mathrm{rank} \left[ \begin{array}{c|c} \lambda I - A & B \end{array} \right] = n , \ \forall \lambda \in \mathcal{C}
\end{align*}
%
\textbf{Proof:} Technically, it is the same argument as the eigenvector/modal reachability test proposed above. The same proof can be adapted for
the PBH test.

\textbf{Definition:} If $\lambda, w^T$ (eigenvalue, left-eigenvector) pair fails in one of the modal reachability tests, we call this 
pair an unreachable mode of the system. 

\subsection{Reachability \& Similarity Transformation}

\textbf{Theorem:} Reachability is invariant under state/similarity transformation $\bar{x} = P^{-1} x$ where $\mathrm{det}(P) \neq 0$. 
We know that system and input matrices under such a transformation take the form
\begin{align*}
    \bar{A} = P A P^{-1} \ , \ \bar{B} = P B
\end{align*}
%
Reachability matrix for the $(\bar{A},\bar{B)}$ can be written as
%
\begin{align*}
   \bar{\mathbf{R}} &= \left[ \begin{array}{c|c|c|c|c} \bar{A}^{n-1} \bar{B} & \bar{A}^{n-2} \bar{B} & \cdots & \bar{A} \bar{B} & \bar{B} \end{array} \right] 
   \\
    =& \left[ \begin{array}{c|c|c|c|c} P A^{n-1} B & P A^{n-2} B & \cdots & P A P^{-1} P B & P B \end{array} \right] 
    \\
    =& P \left[ \begin{array}{c|c|c|c|c} A^{n-1} B & A^{n-2} B & \cdots & A P^{-1} B & B \end{array} \right] = P \mathbf{R}
    \\
    &\Rightarrow \mathrm{rank} [ \mathbf{R} ] = \mathrm{rank} [ \mathbf{\bar{R}} ]
\end{align*}
%
Let $\dot{x} = A x + B u$ be an unreachable system; then it can be convenient and practical to choose coordinates (via similarity transformation)
to highlight reachable and reachable ``spaces''. Let 
%
\begin{align*}
   \mathrm{dim}({\mathbf{R}}) &= r 
   \ \& \
   P^{-1} = T = \left[ \begin{array}{c|c} T_1 & T_2  \end{array} \right] \ \mathrm{where} \ T_1 \in \mathbb{R}^{n \times r} \ , 
   T_2 \in \mathbb{R}^{n \times (n-r)} \ , 
\end{align*}
%
Let's choose a $T_1$ such that $\mathrm{Ra}(T_1) = \mathrm{Ra}(\mathbf{R}) = \mathcal{R}$ and let's choose $T_2$ be a matrix such that 
columns of $T$ are linearly independent. Let's analyze the similarity transformation 
%
%
\begin{align*}
    A T &=  A \left[ \begin{array}{c|c} T_1 & T_2  \end{array} \right] = T \bar{A} = \left[ \begin{array}{c|c} T_1 & T_2  \end{array} \right]
    \left[ \begin{array}{c|c} A_{11} & A_{22} \\ \hline A_{21} & A_{22}  \end{array} \right]
    \\
    A T_1 &= T_1 A_{11} + T_2 A_{22}
\end{align*}
%
Note that the reachable sub-space is $A$ invariant $A T_1$ must remain in $\mathrm{Ra}(T_1) = \mathrm{Ra}(\mathbf{R}) = \mathcal{R}$,
since $T_2$ is composed of linearly independent columns, in order $T_1 A_{11} + T_2 A_{22}$ remain in $\mathrm{Ra}(T_1)$, $A_{12} = \mathbf{0}$.
Now let's analyze the effect of transformation on the input matrix
%
\begin{align*}
    B &= T \bar{B} = \left[ \begin{array}{c|c} T_1 & T_2  \end{array} \right] \left[ \begin{array}{c} B_1 & B_2  \end{array} \right]
    \\
    B &= T_1 B_1 + T_2 B_2
\end{align*}
%
Similarly since $\mathrm{Ra}(B) \subset \mathrm{Ra}(T_1) = \mathrm{Ra}(\mathbf{R}) = \mathcal{R} \ \Rightarrow \ B_2 = 0$, thus a
standard unreachable form takes the form
%
\begin{align*}
\bar{x}[k+1] &= \left[ \begin{array}{c|c} A_{11} & A_{22} \\ \hline 0 & A_{22}  \end{array} \right] \bar{x}[k]
+ \left[ \begin{array}{c} B_1 & 0  \end{array} \right] u[k]
\end{align*}

% **** This ENDS THE EXAMPLES. DON'T DELETE THE FOLLOWING LINE:
\end{document}