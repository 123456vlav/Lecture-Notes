% 


\documentclass[twoside]{article}
\setlength{\oddsidemargin}{0.25 in}
\setlength{\evensidemargin}{-0.25 in}
\setlength{\topmargin}{-0.6 in}
\setlength{\textwidth}{6.5 in}
\setlength{\textheight}{8.5 in}
\setlength{\headsep}{0.75 in}
\setlength{\parindent}{0 in}
\setlength{\parskip}{0.1 in}

%
% ADD PACKAGES here:
%

\usepackage{amsmath,amsfonts,graphicx}


\newcounter{lecnum}
\renewcommand{\thepage}{\thelecnum-\arabic{page}}
\renewcommand{\thesection}{\thelecnum.\arabic{section}}
\renewcommand{\theequation}{\thelecnum.\arabic{equation}}
\renewcommand{\thefigure}{\thelecnum.\arabic{figure}}
\renewcommand{\thetable}{\thelecnum.\arabic{table}}

%
% The following macro is used to generate the header.
%
\newcommand{\lecture}[4]{
   \pagestyle{myheadings}
   \thispagestyle{plain}
   \newpage
   \setcounter{lecnum}{#1}
   \setcounter{page}{1}
   \noindent
   \begin{center}
   \framebox{
      \vbox{\vspace{2mm}
    \hbox to 6.28in { {\bf EE402 - Discrete Time Systems
	\hfill Spring 2018} }
       \vspace{4mm}
       \hbox to 6.28in { {\Large \hfill Lecture #1 \hfill} }
       \vspace{2mm}
       \hbox to 6.28in { {\it Lecturer: #2 \hfill } }
      \vspace{2mm}}
   }
   \end{center}
   \markboth{Lecture #1}{Lecture #1}

   \vspace*{4mm}
}

\renewcommand{\cite}[1]{[#1]}
\def\beginrefs{\begin{list}%
        {[\arabic{equation}]}{\usecounter{equation}
         \setlength{\leftmargin}{2.0truecm}\setlength{\labelsep}{0.4truecm}%
         \setlength{\labelwidth}{1.6truecm}}}
\def\endrefs{\end{list}}
\def\bibentry#1{\item[\hbox{[#1]}]}


\newcommand{\fig}[3]{
			\vspace{#2}
			\begin{center}
			Figure \thelecnum.#1:~#3
			\end{center}
	}

% Use these for theorems, lemmas, proofs, etc.
\newtheorem{theorem}{Theorem}[lecnum]
\newtheorem{lemma}[theorem]{Lemma}
\newtheorem{proposition}[theorem]{Proposition}
\newtheorem{claim}[theorem]{Claim}
\newtheorem{corollary}[theorem]{Corollary}
\newtheorem{definition}[theorem]{Definition}
\newenvironment{proof}{{\bf Proof:}}{\hfill\rule{2mm}{2mm}}
\newtheorem{exmp}[theorem]{Ex}

% **** IF YOU WANT TO DEFINE ADDITIONAL MACROS FOR YOURSELF, PUT THEM HERE:

\begin{document}

% Lecture Details
\lecture{10}{Asst. Prof. M. Mert Ankarali}


%%%%%%%%%%%%%%%%%%%%%%%%%%

\section{Reachability \& Controllability of DT-LTI Systems}

For LTI a discrete-time state-space representation
%
\begin{align*}
  x[k+1] &= A x[k] + B u[k]
\\
  y[k] &= C x[k] + D u[k]
\end{align*}
%
\begin{itemize}
  \item A state $x_r$ is said to be $m$-step \textbf{reachable}, if there exist
  an input sequence, $u[k] , k \in \lbrace 0 , 1, \cdots m-1 \rbrace$, that transfers the state vector 
  $x[k]$ from the origin (i.e. $x[0] = 0$) to the state $x_r$ in m 
  number of steps, i.e. $x[m] = x_r$.
 
  \item A state $x_d$ is said to be $m$-step \textbf{controllable},
  if there exist an input sequence, $u[k] , k \in \lbrace 0 , 1, \cdots m-1 \rbrace$, that transfers the state vector 
  $x[k]$ from the initial state $x_c$ (i.e. $x[0] = x_c$) to the origin
  in $m$ number of steps, i.e. $x[m] = 0$.
\end{itemize}

Note that
\begin{itemize}
  \item the set $\mathcal{R}_m$ of all m-step reachable states is a linear
(sub)space: $\mathcal{R}_m \subset \mathbb{R}^n$
  \item the set $\mathcal{C}_m$ of all m-step controllable states is a linear
(sub)space: $\mathcal{C}_m \subset \mathbb{R}^n$
\end{itemize}

Let's characterize $\mathcal{R}_m$ and then try to generalize the reachability concept. 
When $x[0] = 0$, the solution of $x[m]$ is given by

\begin{align*}
  x[m] &= \left[ \begin{array}{c|c|c|c|c} A^{m-1} B & A^{m-2} B &
         \cdots & A B & B \end{array} \right]
         \left[ \begin{array}{c}
                  u[0] \\ u[1] \\ \vdots \\ u[m-2] \\ u[m-1]
         \end{array} \right]
\end{align*}

Let 
\begin{align*}
 \mathbf{R}_m &= 
        \left[ \begin{array}{c|c|c|c|c} A^{m-1} B & A^{m-2} B &
         \cdots & A B & B \end{array} \right]
\\
 \mathbf{U}_m &=
         \left[ \begin{array}{c}
                  u[0] \\ u[1] \\ \vdots \\ u[m-2] \\ u[m-1]
         \end{array} \right]
\end{align*}
%
then if a state $x_r$ is reachable at $k$ steps, it should 
satisfy the following equation for some $\mathbf{U}_m$.
%
\begin{align*}
 \mathbf{M}_m \mathbf{U}_m &= x_m
\end{align*}
%
In order this matrix equation to have a solution $x_r$
should be in the range space of $\mathbf{M}_m$.
%
\begin{align*}
  x_r \in \mathrm{Ra} ( \mathbf{M}_m ) 
\end{align*}
%
Thus $m$-step reachable sub-space is simply equal to range space of $\mathcal{R}_k$
\begin{align*}
  \mathrm{Ra} ( \mathbf{R}_m ) = \mathcal{R}_m
\end{align*}

\textbf{Theorem:} For $k < n < l$
%
\begin{align*}
 \mathcal{R}_k \subset \mathcal{R}_{n} &= \mathcal{R}_{l} 
  \\
  \mathrm{Ra} ( \mathbf{R}_k ) \subset \mathrm{Ra} ( \mathbf{R}_{n} ) &= \mathrm{Ra} ( \mathbf{R}_{l} )
\end{align*}
%
\textbf{Proof:} It is fairly easy to observe that
%
\begin{align*}
\mathcal{R}_i & \subset \mathcal{R}_{i+1} 
\\
\mathrm{Ra} ( \mathbf{R}_i ) & \subset \mathrm{Ra} ( \mathbf{R}_{i+1} )
\end{align*}
%
since we add a new column (or columns for multi-input systems) to $ \mathbf{R}_i$, thus it can only increase the dimension of the range-space. Thus we can conclude that 
%
%
\begin{align*}
 \mathcal{R}_k &\subset \mathcal{R}_{n} \subset \mathcal{R}_{l} 
  \\
  \mathrm{Ra} ( \mathbf{R}_k ) &\subset \mathrm{Ra} ( \mathbf{R}_{n} ) \subset \mathrm{Ra} ( \mathbf{R}_{l} )
\end{align*}
%
In order prove $\mathcal{R}_{n} = \mathcal{R}_{l} $, we simply use the Cayley-Hamilton theorem. 
Based on Cayley-Hamilton theorem 
%
\begin{align*}
 A^n &= -a_1 A^{n-1} - \cdots - a_{n-1} A - a_n I
 \\
 A^n B &= -a_1 A^{n-1} B - \cdots - a_{n-1} A B - a_n B
\end{align*}
%
which shows that $ A^n B$ is linearly dependent to previous columns and thus 
%
%
\begin{align*}
\mathcal{R}_{n} =& \mathcal{R}_{l} 
  \\
\mathrm{Ra} ( \mathbf{R}_{n} ) &= \mathrm{Ra} ( \mathbf{R}_{l} )
\end{align*}
%
This theorem shows that if $x_r$ is reachable in $n$ steps then it is reachable for $l > n$ steps, similarly, if it is not reachable in 
$n$ steps then it is reachable for $l > n$ steps. In this context, the sub-space of states reachable in $n$-steps, $\mathcal{R}_n$
is referred as the reachable subspace of $(A,N)$, and will be denoted simply by $\mathcal{R}$ and $\mathbf{R} = \mathbf{R}_k$ will be system 
wide the reachability matrix. The system is termed a (fully) reachable system if 
%
\begin{align*}
 \mathrm{rank} (\mathbf{R}) &= n 
 \\
 \mathrm{Ra} (\mathbf{R}) &= \mathcal{R} = \mathbb{R}^n
\end{align*}

\begin{exmp}
Solve the following problems regarding controllable sub-space
\end{exmp}
\begin{itemize}
    \item Show that $\mathcal{R} \subset \mathcal{C}$, $\forall (A,B)$, however $\mathcal{C} \subset \mathcal{R}$ not necessarily true $\forall (A,B)$.
    \item Similar to the reachable subspace, characterize the controllable subspace
    \item Derive conditions such that $\mathcal{R} = \mathcal{C}$
\end{itemize}

\subsection{Reachability Gramian}

An alternative characterization of $\mathbf{R}$ is using reachability Gramian (which is more critical for CT systems).
$m$-step reachability Gramian, $\mathbf{P}_m$, is defined as
%
\begin{align}
 \mathbf{P}_m &= \mathbf{R}_m {\mathbf{R}_m}^T =\sum\limits_{i=0}^{k-1} A^i B B^T \begin{pmatrix} A^T \end{pmatrix}^i
\end{align}
%
Note that $\mathcal{P}_m$ is a symmetric positive semi-definite matrix.

\textbf{Lemma:} $\mathcal{R}_m = \mathrm{Ra}( \mathbf{R}_m ) = \mathrm{Ra} ( \mathbf{P}_m )$

\textbf{Proof:} Let's fits show that $\mathrm{Ra}( \mathbf{P}_m ) \subset \mathrm{Ra}( \mathbf{R}_m )$. 
If $x \in \mathrm{Ra} (\mathbf{P}_m)$, then $\exists v \in \mathbb{R}^n$ s.t. $x = \mathbf{P}_m v$ then
%
\begin{align*}
 x = \mathbf{P}_m v &= \mathbf{R}_m {\mathbf{R}_m}^T v = \mathbf{R}_m y \ \Rightarrow \ x \in \mathrm{Ra}( \mathbf{R}_m ) \ \Rightarrow \ \mathrm{Ra}( \mathbf{P}_m ) \subset \mathrm{Ra}( \mathbf{R}_m )
\end{align*}

Now let's show that $\mathrm{Ra}( \mathbf{R}_m ) \subset \mathrm{Ra}( \mathbf{P}_m )$. We know that
%
\begin{align*}
    \mathrm{Ra}( \mathbf{R}_m ) \subset \mathrm{Ra}( \mathbf{P}_m ) 
    \ \iff \ \mathrm{Ra}^\perp( \mathbf{P}_m ) \subset \mathrm{Ra}^\perp( \mathbf{R}_m )
\end{align*}
%
So we can equivalently show that $\mathrm{Ra}^\perp( \mathbf{P}_m ) \subset \mathrm{Ra}^\perp( \mathbf{R}_m )$.
Let $q \in \mathrm{Ra}^\perp( \mathbf{P}_m )$, then
%
%
\begin{align*}
    q^T \mathbf{P}_m = \mathbf{0} \ \Rightarrow \ q^T \mathbf{P}_m q = 0 \ &\iff \ q^T \mathbf{R}_m {\mathbf{R}_m}^T q = 0
    \ \iff \ ( {\mathbf{R}_m}^T q )^T ( {\mathbf{R}_m}^T q ) = 0
    \\ \ &\iff \ {\mathbf{R}_m}^T q = \mathbf{0} \ \iff \ q^T {\mathbf{R}_m} = \mathbf{0}^T = \mathbf{0}  
    \\ \ &\Rightarrow \ q \in \mathrm{Ra}^\perp( \mathbf{R}_m ) \ \Rightarrow \ \mathrm{Ra}^\perp( \mathbf{P}_m ) \subset \mathrm{Ra}^\perp( \mathbf{R}_m )
\end{align*}
%
This completes the proof. As a result of this lemma, full reachable subspace $\mathcal{R} = \mathrm{Ra}( \mathbf{P}_l )$ for any $l \geq 0$.
As a result we can make the following conclusions 
\begin{itemize}
\item $(A,B)$ pair is fully reachable $\iff$ $\mathrm{dim} \left [\mathrm{Ra}( \mathbf{P}_l ) \right] = n$ for any $l \geq n$
\item $(A,B)$ pair is fully reachable $\iff$ $\mathrm{det} \left [\mathbf{P}_l\right] \neq 0$ for any $l \geq n$
\end{itemize}
% 
If $x[k+1] = A x[k]$ is asymptotically stable, then $\textbf{P}_{\infty} = \lim_{k \to \infty} \sum\limits_{i=0}^{k-1} A^i B B^T \begin{pmatrix} A^T \end{pmatrix}^i \stackrel{\triangle}{=} P$ is well defined and $P$ satisfies the following Lyapunov equation
%
\begin{align*}
 A P A^T - P = -B B^T
\end{align*}
%
To understand this derivation, refer to the Quadratic Lyapunov Functions for LTI systems section in Lecture Notes 8. 
%
\begin{exmp}
 Let $x[k+1] = Ax [k]$ be asymptotically stable. Then, show that  
\end{exmp}
%
\begin{align*}
 A P A^T - P = -B B^T
\end{align*}
%
has a unique positive definite solution of $P$, if and only if, $(A,B)$ pair is fully reachable.

\subsection{Modal Aspects and Modal Reachability Tests}

\textbf{Lemma:} The reachable sub-space, $\mathcal{R}$ is $A$ invariant, i.e. $x\in \mathcal{R} \ \Rightarrow \ A x \in \mathcal{R}$. We 
write this as $A \mathcal{R} \subset \mathcal{R}$.

\textbf{Proof:} Let $x \in \mathcal{R}$ then $\exists \textbf{U}_n \in \mathbb{R}^{n p}$, s.t. $\ x = \mathbf{R} \textbf{U}_n 
 \ \mathrm{where} \ B \in \mathbb{R}^{n \times p}$, then
%
\begin{align*}
 x = \left[ \begin{array}{c|c|c|c|c} A^{n-1} B & A^{n-2} B &
         \cdots & A B & B \end{array} \right] \textbf{U}_n 
\end{align*}
%
Now let's expand $A x$
%
\begin{align*}
 A x = \left[ \begin{array}{c|c|c|c|c} A^{n} B & A^{n-1} B &
         \cdots & A^2 B & A B \end{array} \right] \textbf{U}_n 
\end{align*}
%
Using Cayley-Hamilton theorem, we reach that
%
%
\begin{align*}
 A^n &= -a_{n-1} A^{n-1} - \cdots - a_1 A - a_0 I
 \\
 A^n B &= -a_{n-1} A^{n-1} B - \cdots - a_1 A B - a_0 B
 \\
 A x &= \left[ \begin{array}{c|c|c|c|c} \sum\limits_{i=0}^{n-1} A^{i} B & A^{n-2} B &
         \cdots & A^2 B & A B \end{array} \right] \textbf{U}_n 
 \\
 A x &\in \mathrm{Span} \left\lbrace \begin{array}{ccccc} A^{n-1} B , & A^{n-2} B , &
         \cdots & A B , & B \end{array} \right\rbrace = \mathcal{R}
\end{align*}
%
\textbf{Theorem:} $(A,B)$ pair is unreachable (or not fully reachable) if and only if $w^T B = 0$ for some left eigenvector of $A$, i.e. 
$w^T A = \lambda w^T$.

\textbf{Proof:} Let $w^T A = \lambda w^T$ and $w^T B = 0$, then
%
\begin{align*}
 w^T \mathbf{R} &= w^T \left[ \begin{array}{c|c|c|c|c} A^{n-1} B & A^{n-2} B &
         \cdots & A B & B \end{array} \right] 
         \\
  &= \left[ \begin{array}{c|c|c|c|c} w^T A^{n-1} B & w^T A^{n-2} B &
         \cdots & w^T A B & w^T B \end{array} \right] 
         \\
&= \left[ \begin{array}{c|c|c|c|c} \lambda^{n-1} w^T B & \lambda^{n-2} w^T B &
         \cdots & \lambda w^T  B & w^T B \end{array} \right] = 0 \ \rightarrow \ (A,B) \ \mathrm{unreachable} 
\end{align*}
%
Now let's show that if $(A,B)$ is not reachable, then $\exists w$ s.t. $w^T A = \lambda w^T$ and $w^T B = 0$.  if $(A,B)$ is not reachable, then there exist $q \in \mathbb{R}^n$ such that
%
\begin{align*}
q^T \mathbf{R} &= 0 \ \Rightarrow \ \mathbf{R}^T q = 0 
\ \Rightarrow \ \begin{array}{c} B^T (A^T)^{n-1} q = 0 \\ B^T (A^T)^{n-2} q = 0 \\
         \vdots \\ B^T A^T q = 0 \\ B^T q = 0 \end{array} 
\end{align*}
%
Define $\mathcal{S} = \mathrm{Span} \left\lbrace q \ , \ A^T q \ , \ \cdots \ , \  (A^T)^{n-1} q \right\rbrace$, note that 
$\mathcal{S} \subset \mathcal{N}(B^T)$ and $\mathcal{S}$ is invariant under $A^T$ (see the similar proof above), i.e. 
If $v \in \mathcal{S}$, then $A^T v \in \mathcal{S}$. Let $\mathrm{dim} (\mathcal{S}) = j$ and $V \in \mathbb{R}^{n \times j}$ 
be a matrix whose columns form a basis for $\mathcal{S}$. Since $\mathcal{S}$ is invariant under $A^T$, we can find a transformation
matrix, $\Gamma \in \mathbb{R}^{j \times j}$ such that 
%
\begin{align*}
A^T V = V \Gamma
\end{align*}
%
Let $\nu$ be an eigenvector of $V$, i.e. $\Gamma \nu = \alpha \nu , \, \alpha \in \mathcal{C}$, then 
%
\begin{align*}
A^T V \nu = V \Gamma \nu = \alpha V \nu \ \rightarrow \ z = V \nu \in \mathbb{C}^n
\end{align*}
%
where $z$ is an eigenvector of $A^T$, moreover $z \in \mathcal{S} = \mathcal{N}(B^T)$. This completes the proof. 

\textbf{Theorem:} \textit{PBH Rechability Test} - $(A,B)$ is reachable if and only if 
%
\begin{align*}
    \mathrm{rank} \left[ \begin{array}{c|c} \lambda I - A & B \end{array} \right] = n , \ \forall \lambda \in \mathcal{C}
\end{align*}
%
\textbf{Proof:} Technically, it is the same argument as the eigenvector/modal reachability test proposed above. The same proof can be adapted for
the PBH test.

\textbf{Definition:} If $\lambda, w^T$ (eigenvalue, left-eigenvector) pair fails in one of the modal reachability tests, we call this 
pair an unreachable mode of the system. 

\subsection{Reachability \& Similarity Transformation}

\textbf{Theorem:} Reachability is invariant under state/similarity transformation $\bar{x} = P^{-1} x$ where $\mathrm{det}(P) \neq 0$. 
We know that system and input matrices under such a transformation take the form
\begin{align*}
    \bar{A} = P A P^{-1} \ , \ \bar{B} = P B
\end{align*}
%
Reachability matrix for the $(\bar{A},\bar{B)}$ can be written as
%
\begin{align*}
   \bar{\mathbf{R}} &= \left[ \begin{array}{c|c|c|c|c} \bar{A}^{n-1} \bar{B} & \bar{A}^{n-2} \bar{B} & \cdots & \bar{A} \bar{B} & \bar{B} \end{array} \right] 
   \\
    =& \left[ \begin{array}{c|c|c|c|c} P A^{n-1} B & P A^{n-2} B & \cdots & P A P^{-1} P B & P B \end{array} \right] 
    \\
    =& P \left[ \begin{array}{c|c|c|c|c} A^{n-1} B & A^{n-2} B & \cdots & A P^{-1} B & B \end{array} \right] = P \mathbf{R}
    \\
    &\Rightarrow \mathrm{rank} [ \mathbf{R} ] = \mathrm{rank} [ \mathbf{\bar{R}} ]
\end{align*}
%
Let $x[k+1] = A x[k] + B u[k]$ be an unreachable system; then it can be convenient and practical to choose coordinates (via similarity transformation)
to highlight reachable and reachable ``spaces''. Let 
%
\begin{align*}
   \mathrm{dim}({\mathbf{R}}) &= r 
   \ \& \
   P^{-1} = T = \left[ \begin{array}{c|c} T_1 & T_2  \end{array} \right] \ \mathrm{where} \ T_1 \in \mathbb{R}^{n \times r} \ , 
   T_2 \in \mathbb{R}^{n \times (n-r)} \ , 
\end{align*}
%
Let's choose a $T_1$ such that $\mathrm{Ra}(T_1) = \mathrm{Ra}(\mathbf{R}) = \mathcal{R}$ and let's choose $T_2$ be a matrix such that 
columns of $T$ are linearly independent. Let's analyze the similarity transformation. 
%
%
\begin{align*}
    A T &=  A \left[ \begin{array}{c|c} T_1 & T_2  \end{array} \right] = T \bar{A} = \left[ \begin{array}{c|c} T_1 & T_2  \end{array} \right]
    \left[ \begin{array}{c|c} A_{11} & A_{22} \\ \hline A_{21} & A_{22}  \end{array} \right]
    \\
    A T_1 &= T_1 A_{11} + T_2 A_{22}
\end{align*}
%
Note that the reachable sub-space is $A$ invariant $A T_1$ must remain in $\mathrm{Ra}(T_1) = \mathrm{Ra}(\mathbf{R}) = \mathcal{R}$,
since $T_2$ is composed of linearly independent columns, in order $T_1 A_{11} + T_2 A_{22}$ remain in $\mathrm{Ra}(T_1)$, $A_{12} = \mathbf{0}$.
Now let's analyze the effect of transformation on the input matrix.
%
\begin{align*}
    B &= T \bar{B} = \left[ \begin{array}{c|c} T_1 & T_2  \end{array} \right] \left[ \begin{array}{c} B_1 \\ B_2  \end{array} \right]
    \\
    B &= T_1 B_1 + T_2 B_2
\end{align*}
%
Similarly since $\mathrm{Ra}(B) \subset \mathrm{Ra}(T_1) = \mathrm{Ra}(\mathbf{R}) = \mathcal{R} \ \Rightarrow \ B_2 = 0$, thus a standard unreachable form takes the form
%
\begin{align*}
\bar{x}[k+1] &= \left[ \begin{array}{c|c} A_{11} & A_{22} \\ \hline 0 & A_{22}  \end{array} \right] \bar{x}[k]
+ \left[ \begin{array}{c} B_1 \\ 0 \end{array} \right] u[k]
\end{align*}

\newpage

\section{Reachability \& Controllability of CT-LTI Systems}

For LTI a continuous-time state-space representation
%
\begin{align*}
  \dot{x} &= A x + B u
\\
  y &= C x + D u
\end{align*}
%
\begin{itemize}
  \item A state $x_r$ is said to be \textbf{reachable} in time $t_f > 0$, if there exists
  an input signal, $u(t) , t \in [ 0 , t_f ]$, that transfers the state vector 
  $x(t)$ from the origin (i.e. $x(0) = 0$) to the state $x_r$ at the given time, 
   i.e. $x(t) = x_r$.
 
  \item A state $x_c$ is said to be \textbf{controllable} in time $t_f > 0$,
  if there exist an input signal, $u(t) , t \in [0 , t_f] $, that transfers the state vector 
  $x(t)$ from the initial state $x_c$ (i.e. $x(0) = x_c$) to the origin
  in given time frame, i.e. $x(t) = 0$.
\end{itemize}

Note that
\begin{itemize}
  \item the set $\mathcal{R}_{t_f}$ of all states that reachable in time $t_f$ is a linear
(sub)space: $\mathcal{R}_{t_f} \subset \mathbb{R}^n$
  \item the set $\mathcal{R}_{t_f}$ of all states that controllable in time $t_f$ is a linear
(sub)space: $\mathcal{R}_{t_f} \subset \mathbb{R}^n$
\end{itemize}

Let's characterize $\mathcal{R}_{t_f}$. In CT derivation of the reachability 
matrix is not as intuitive as the case in DT systems. Instead, we construct the
main concepts using the reachability Gramian for CT systems. Note that 
for a CT-LTI system $x(t_f)$ for a zero-state response written as
%
\begin{align*}
  x(t_f) = \int\limits_{0}^{t_f} e^{A (t_f - \tau)} B u(\tau) d \tau
\end{align*}
%
Obviously if a $x_r \in \mathbb{R}$ is reachable in time $t_f$, 
then $\exists u(t), \ t \in [0 , t_f]$ such that $x(t_f) = x_r$. In this context,
the reachability Grammian is defined as
%
\begin{align}
    \textbf{P}(t) = \int\limits_{0}^{t} e^{A (t - \tau)} B B^T e^{A^T (t - \tau)}  d \tau 
\end{align}
%
\textbf{Theorem:} $\mathcal{R}_{t_f} = \mathrm{Ra}[ \textbf{P}(t_f) ]$, i.e. reachable in time $t_f$ sub-space is characterized by the range-space of the reachability Gramian.

\textbf{Proof:} Let's first show that $\mathrm{Ra}[ \textbf{P}(t_f) ] \subset \mathcal{R}_{t_f}$.
Let $x^* \in \mathrm{Ra}[ \textbf{P}(t_f) ]$, then $\exists v \in \mathbb{R}^n$ such that 
%
\begin{align}
    x^* = \textbf{P}(t_f) v
\end{align}
%
Let $u(t) = B^T e^{A (t_f - t)} v$, then
%
\begin{align*}
  x(t_f) &= \int\limits_{0}^{t_f} e^{A (t_f - \tau)} B u(\tau) d \tau
  \\
  &= \int\limits_{0}^{t_f} e^{A (t_f - \tau)} B B^T e^{A^T (t_f - \tau)} v d \tau
  \\
  &= \textbf{P}(t_f) v = x^* \ \Rightarrow \ x^* \in  \mathcal{R}_{t_f} 
  \ \Rightarrow \ \mathrm{Ra}[ \textbf{P}(t_f) ] \subset \mathcal{R}_{t_f}
\end{align*}

Now let's show that $\mathcal{R}_{t_f} \subset \mathrm{Ra}[ \textbf{P}(t_f) ] $. Equivalently we can show that 
$\mathrm{Ra}^\perp[ \textbf{P}(t_f) ] \subset \mathcal{R}_{t_f}^\perp$. Let $q \in \mathrm{Ra}^\perp[ \textbf{P}(t_f) ]$, then 
%
\begin{align}
\notag
q^T \textbf{P}(t_f) = 0 \, &\Rightarrow \, q^T \textbf{P}(t_f) q = 0 
\, \iff \,
\int\limits_{0}^{t_f} q^T e^{A (t_f - \tau)} B B^T e^{A^T (t_f - \tau)} q d \tau = 0
\, \iff \,
\int\limits_{0}^{t_f} || B^T e^{A^T (t_f - \tau)} q ||_2^2 d \tau = 0
\\
&\iff \ B^T e^{A^T (t_f - \tau)} q = 0 \, \forall \, \tau \in [0 , t_f]
\label{eq:orthGram}
\end{align}
%
Let $x(t_f) \in \mathcal{R}_{t_f}$ (any vector in the reachable space), then we know that
%
\begin{align*}
  x(t_f) &= \int\limits_{0}^{t_f} e^{A (t_f - \tau)} B u(\tau) d \tau
\end{align*}
%
Let's analyze $x(t_f)^T q$
%
\begin{align*}
  x(t_f)^T q &= \int\limits_{0}^{t_f} u(\tau)^T \left[ B^T e^{A^T (t_f - \tau)} q \right] d \tau
  = \int\limits_{0}^{t_f} u(\tau)^T \left[ 0 \right] d \tau = 0
  \, \Rightarrow \, q \in \mathcal{R}_{t_f}^\perp
  \, \Rightarrow \, \mathrm{Ra}^\perp[ \textbf{P}(t_f) ] \subset \mathcal{R}_{t_f}^\perp
\end{align*}
%
This completes the proof.

\textbf{Theorem:} $\mathcal{R}_{t_f} = \mathrm{Ra}[ \textbf{P}(t_f) ] = \mathrm{Ra}[ \mathbf{R} ] \ \forall t_f > 0$. This theorem claims that also, for CT-LTI systems, the reachability matrix characterizes the reachable in time $t_f$ sub-space; moreover, reachable sub-space is indeed independent of $t_f$.

Let's concentrate the relation between the reachability grammian, $\textbf{P}(t_f) = \int\limits_{0}^{t_f} e^{A (t_f - \tau)} B B^T e^{A^T (t_f - \tau)}  d \tau $ 
and the reachability matrix, $\mathbf{R} = \left[ \begin{array}{c|c|c|c|c} A^{n-1} B & A^{n-2} B & \cdots & A B & B \end{array} \right] $. Note that 
%
\begin{align*}
  \mathrm{Ra}[ \textbf{P}(t_f) ] = \mathrm{Ra}[ \mathbf{R} ]
  \, \iff \, \mathrm{Ra}^\perp [ \textbf{P}(t_f) ] = \mathrm{Ra}^\perp [ \mathbf{R} ]
\end{align*}
%
Firs show that if $\mathrm{Ra}^\perp [ \mathbf{R} ] \subset \mathrm{Ra}^\perp [ \textbf{P}(t_f) ]$.
Let $q \in \mathrm{Ra}^\perp [ \mathbf{R} ] $, then 
%
\begin{align*}
  q^T \mathbf{R} = 0 \, &\iff \, q^T \left[ \begin{array}{c|c|c|c|c} A^{n-1} B & A^{n-2} B & \cdots & A B & B \end{array} \right] = 0 \ \iff \, q^T A^i B = 0 \forall i \in \mathbb{Z^+}
\end{align*}
%
Note that for $i \geq n$, we referred to the Cayley-Hamilton theorem. Now let's check if 
$q \in \mathrm{Ra}^\perp [ \mathbf{P}_{t_f} ] $
%
\begin{align*}
  q^T \mathbf{P}_{t_f} &= \int\limits_{0}^{t_f} q^T e^{A (t_f - \tau)} B B^T e^{A^T (t_f - \tau)}  d \tau 
  \\
  q^T A^i B &= 0 \forall i \in \mathbb{Z^+} \, \Rightarrow \, q^T e^{A (t_f - \tau)} = 0 \, \forall t_f \geq 0
  \\
  &\Rightarrow \, q^T \mathbf{P}_{t_f} = 0 \, \Rightarrow \, q \in \mathrm{Ra}^\perp [ \mathbf{P}_{t_f} ]
  \, \Rightarrow \, \mathrm{Ra}^\perp [ \mathbf{R} ] \subset \mathrm{Ra}^\perp [ \textbf{P}(t_f) ]
\end{align*}
%
Now let's show that if $\mathrm{Ra}^\perp [ \textbf{P}(t_f) ] \subset \mathrm{Ra}^\perp [ \mathbf{R} ]$.
Let $q \in \mathrm{Ra}^\perp [ \mathbf{R} ] $, from (\ref{eq:orthGram}) we know that 
%
\begin{align*}
  q e^{A (t_f - \tau)} B = 0 \, \forall \, \tau \in [0 , t_f]
\end{align*}
%
Then we know that 
%
\begin{align*}
  [ q e^{A \eta} B]_{\eta = 0} = 0 \, &\Rightarrow \, q^T B = 0
  \\
  \frac{d}{dt}[ q e^{A \eta} B]_{\eta = 0} = 0 \, &\rightarrow \, q^T A B = 0
  \\
  \frac{d^2}{dt^2}[ q e^{A \eta} B]_{\eta = 0} = 0 \, &\Rightarrow \, q^T A^2 B = 0
  \\
  &\vdots
    \\
    \frac{d^{n-1}}{dt^{n-1}}[ q e^{A \eta} B]_{\eta = 0} = 0 \, &\Rightarrow \, q^T A^{n-1} B = 0
    \\
    &\Rightarrow \, q^T \mathbf{R} = 0 \, \Rightarrow \, q \in \mathrm{Ra}^\perp [ \mathbf{R} ]
    \, \Rightarrow \, \mathrm{Ra}^\perp [ \textbf{P}(t_f) ] \subset \mathrm{Ra}^\perp [ \mathbf{R} ]
\end{align*}
%
This completes the proof. This theorem indicates that similar to the DT systems range space of the
reachability matrix, $\mathbf{R}$, characterizes the whole reachable space, and this space independent
of the time interval. In that respect 

\textbf{Corollary:} $(A,B)$ is fully reachable if and only if $\mathrm{dim} \left( \mathrm{Ra} [ \mathbf{R} ] \right) = \mathrm{dim} \left( \mathrm{Ra} [ \textbf{P}(t_f) ] \right)$

\textbf{Remark:} This theorem shows that the condition for CT reachability is expressed in the same way 
as DT reachability. Hence all our DT results on standard forms for unreachable systems, modal tests, and so on; remain unchanged. We, therefore, do not repeat any of these DT results for the CT case but count on you to explicitly note the CT versions of our earlier DT results.

\begin{exmp}
Show that
\end{exmp}
if $\dot{x} = A x$ is asymptotically stable, then reachability Grammian at $t_f \to \infty$, $P := \mathbf{P}_{\infty}$,
satisfies the following Lyapunov equation
%
\begin{align*}
    A P + P A^T = - B B^T
\end{align*}
% 
, and this Lyapunov equation has a (unique) positive-definite solution for $P$,
if and only if, $(A,B)$ is fully reachable.

\begin{exmp}
Show that for a CT-LTI system $\mathcal{C} = \mathcal{R}$
\end{exmp}
i.e., show that reachability is equivalent to controllability for CT systems. 

\begin{exmp}
Let 
\end{exmp}
%
%
\begin{align*}
    \dot{x} = \begin{bmatrix} \lambda_1 & & 0  \\
    & \ddots &
    \\
    0 & &  \lambda_n 
    \end{bmatrix} x + \begin{bmatrix} b_1 \\ \vdots \\ b_n \end{bmatrix} u
\end{align*}
% 
Analyze the reachability of the $(A,B)$ pair and
find the necessary and sufficient conditions such that 
$(A,B)$ fully reachable.

\textbf{Solution:} Since $A$ is already in diagonal Canonical 
form, the most convenient way to analyze reachability is the PBH's modal
test 
%
\begin{align*}
  \left[ \begin{array}{c|c} A - \lambda I & B \end{array} \right]
  &= \left[ \begin{array}{ccc|c} 
  \lambda_1 & & 0 & b_1 \\
    & \ddots &
    \\
    0 & &  \lambda_n & b_n \end{array} \right]
\end{align*}
%
\textit{Necessary Condition:} Let $\lambda = \lambda_i$, if $b_i = 0$,
then $\left[ \begin{array}{c|c} A - \lambda I & B \end{array} \right]$
drops rank. Thus, if $\exists \lambda_i$ such that $b_i = 0$, then the system 
is unreachable. 

\textit{Sufficient Condition:} Not that if $\lambda_i \neq \lambda_j$ for $i \neq j$,
the condition above also becomes sufficient. Now analyze the general case, without loss 
of generality let's assume $\lambda_1 = \lambda_2 = \cdots = \lambda_p$ and all other 
eigenvalues are different. Let $\lambda = \lambda_1$ in the PBH test, then first $p$-rows of the
PBH matrix takes the form
%
\begin{align*}
  \left[ \begin{array}{c|c} A - \lambda_1 I & B \end{array} \right]
  &= \left[ \begin{array}{c|c} 
  0_{1 \times n} & b_1 \\
  0_{1 \times n}  & b_2 \\
  \\
  \vdots
  \\
  0_{1 \times n}  & b_p \\
     \end{array} \right]
\end{align*}
%
PBH matrix does not drop rank, if and only if, $\lbrace b_1^T , b_2^T , \cdots , b_p^T \rbrace$ 
is a linearly independent set. We can see that number of inputs of the system has to be larger than 
or equal to $p$ (indeed $\mathrm{max}(p)$ for system-wide reachability). Moreover, the rank of the 
following sub-matrix matrix has to be equal to the algebraic multiplicity of the eigenvalue, i.e.
%
\begin{align*}
  B_{\lambda_1}
  &= \left[ \begin{array}{c} 
  b_1 \\
   b_2 \\
  \\
  \vdots
  \\
   b_p \\
     \end{array} \right] \ , \ \mathrm{rank}(B_{\lambda_1}) = p  
\end{align*}
%

\begin{exmp}
Let 
\end{exmp}
%
%
\begin{align*}
    \dot{x} = \left[
 \begin{array}{ccc|cc} \lambda_1 & 1 & 0 & 0 & 0 \\
     0 & \lambda_1 & 1 & 0 & 0 
    \\
    0 & 0 & \lambda_1 &  0 & 0 \\
    \hline
    0 & 0 & 0 & \lambda_2 & 1 \\
    0 & 0 & 0 & 0 & \lambda_2 \\
    \end{array} \right]
    x + \left[ \begin{array}{c} b_{1,1} \\ b_{1,2} 
    \\ b_{1,3} \\ b_{2,1} \\ b_{2,2}  
  \end{array} \right] u
\end{align*}
% 
Analyze the reachability of the $(A,B)$ pair for $\lambda_1 
neq \lambda_2$ and $\lambda_1 = \lambda_2$
and find the necessary and sufficient conditions such that 
$(A,B)$ fully reachable.

\textbf{Solution:} Let's first assume that $\lambda_1 
neq \lambda_2$, we can analyze each mode separately using the BPH test. Let's start with $\lambda = \lambda_1$
%
\begin{align*}
  \left[ \begin{array}{c|c} A - \lambda_1 I & B \end{array} \right]
  &= \left[  \begin{array}{ccc|cc|c} 0 & 1 & 0 & 0 & 0 & b_{1,1} \\
     0 & 0 & 1 & 0 & 0 & b_{1,2}
    \\
    0 & 0 & 0 &  0 & 0 & b_{1,3 }\\
    \hline
    0 & 0 & 0 & \lambda_2 - \lambda_1 & 1 & b_{2,1} \\
    0 & 0 & 0 & 0 & \lambda_2 - \lambda_2 & b_{2,2}   \\
    \end{array} \right]
\end{align*}
%
The matrix above is full rank if and only if $b_{1,3}  
\neq 0$,
this implies that $b_{1,1}$ and $b_{1,2}$ do not affect
the reachability condition. Similarly for $\lambda = \lambda_2$
we can find that $b_{2,2} \neq 0$ is a necessary condition for 
reachability. So for $\lambda_1 neq \lambda_2$, $(A,B)$ pair
is reachable if and only if $b_{2,2} \neq 0$ and $b_{1,3} \neq 0$.

Now let's analyze the case $\lambda_2 = \lambda_1$. It is fairly easy 
to observe that $b_{2,2} \neq 0$ and $b_{1,3} \neq 0$ is a necessary
but not sufficient condition for full reachability. Let's re-analyze the
PBH matrix
%
\begin{align*}
  \left[ \begin{array}{c|c} A - \lambda_1 I & B \end{array} \right]
  &= \left[  \begin{array}{ccc|cc|c} 0 & 1 & 0 & 0 & 0 & b_{1,1} \\
     0 & 0 & 1 & 0 & 0 & b_{1,2}
    \\
    0 & 0 & 0 &  0 & 0 & b_{1,3 }\\
    \hline
    0 & 0 & 0 & 0 & 1 & b_{2,1} \\
    0 & 0 & 0 & 0 & 0 & b_{2,2}   \\
    \end{array} \right]
\end{align*}
%
The matrix has full rank if and only if 
%
\begin{align*}
 \mathrm{Ra}\left\lbrace 
 \begin{bmatrix}
     b_{1,3}
     \\
     b_{2,2}
 \end{bmatrix}
 \right\rbrace = \mathbb{R}^2
\end{align*}
%

% **** This ENDS THE EXAMPLES. DON'T DELETE THE FOLLOWING LINE:
\end{document}
