% 


\documentclass[twoside]{article}
\setlength{\oddsidemargin}{0.25 in}
\setlength{\evensidemargin}{-0.25 in}
\setlength{\topmargin}{-0.6 in}
\setlength{\textwidth}{6.5 in}
\setlength{\textheight}{8.5 in}
\setlength{\headsep}{0.75 in}
\setlength{\parindent}{0 in}
\setlength{\parskip}{0.1 in}

%
% ADD PACKAGES here:
%

\usepackage{amsmath,amsfonts,graphicx,arydshln}


\newcounter{lecnum}
\renewcommand{\thepage}{\thelecnum-\arabic{page}}
\renewcommand{\thesection}{\thelecnum.\arabic{section}}
\renewcommand{\theequation}{\thelecnum.\arabic{equation}}
\renewcommand{\thefigure}{\thelecnum.\arabic{figure}}
\renewcommand{\thetable}{\thelecnum.\arabic{table}}

%
% The following macro is used to generate the header.
%
\newcommand{\lecture}[4]{
   \pagestyle{myheadings}
   \thispagestyle{plain}
   \newpage
   \setcounter{lecnum}{#1}
   \setcounter{page}{1}
   \noindent
   \begin{center}
   \framebox{
      \vbox{\vspace{2mm}
    \hbox to 6.28in { {\bf EE502 - Linear Systems Theory II
	\hfill Spring 2023} }
       \vspace{4mm}
       \hbox to 6.28in { {\Large \hfill Lecture #1 \hfill} }
       \vspace{2mm}
       \hbox to 6.28in { {\it Lecturer: #2 \hfill } }
      \vspace{2mm}}
   }
   \end{center}
   \markboth{Lecture #1}{Lecture #1}

   \vspace*{4mm}
}

\renewcommand{\cite}[1]{[#1]}
\def\beginrefs{\begin{list}%
        {[\arabic{equation}]}{\usecounter{equation}
         \setlength{\leftmargin}{2.0truecm}\setlength{\labelsep}{0.4truecm}%
         \setlength{\labelwidth}{1.6truecm}}}
\def\endrefs{\end{list}}
\def\bibentry#1{\item[\hbox{[#1]}]}


\newcommand{\fig}[3]{
			\vspace{#2}
			\begin{center}
			Figure \thelecnum.#1:~#3
			\end{center}
	}

% Use these for theorems, lemmas, proofs, etc.
\newtheorem{theorem}{Theorem}[lecnum]
\newtheorem{lemma}[theorem]{Lemma}
\newtheorem{proposition}[theorem]{Proposition}
\newtheorem{claim}[theorem]{Claim}
\newtheorem{corollary}[theorem]{Corollary}
\newtheorem{definition}[theorem]{Definition}
\newenvironment{proof}{{\bf Proof:}}{\hfill\rule{2mm}{2mm}}
\newtheorem{exmp}[theorem]{Ex}

% **** IF YOU WANT TO DEFINE ADDITIONAL MACROS FOR YOURSELF, PUT THEM HERE:

\begin{document}

% Lecture Details
\lecture{11}{Asst. Prof. M. Mert Ankarali}


%%%%%%%%%%%%%%%%%%%%%%%%%%

\section{Observability in CT-LTI Systems}

In the context of observability of dynamical systems, it 
turns out that it is more convenient to think in terms of
``un-observabile states'' and then connect it to the 
concept of observability and fully observable systems.
as reflected in the following definitions.

\textbf{Definition:} For an LTI contınıous-time state-space representation
%
\begin{align*}
  \dot{x} &= A x + B u
\\
  y &= C x + D u
\end{align*}
%
A state $x_u$ is said to be unobservable over $t \in [0 , T )$,
if with $x(0) = x_u$ and $\forall \, u(t)$ we get the same $y(t)$ as we would with $x(0) = 0$. 

The set, $\bar{\mathcal{O}_T}$, of all unobservable states over $t \in [0 , T )$ forms a vector space, $\bar{\mathcal{O}_T} \subset 
\mathbb{R}^n$, and the system is called fully observable over
$t \in [0 , T )$, if $\mathrm{dim}[\bar{\mathcal{O}_T}] = 0$.

Note for linear dynamical systems observability of a state and
system is independent from $u(t)$, in that respect we will 
only analyze zero-input response of the system in our derivations.

\textbf{Theorem:} $x_u \in \bar{\mathcal{O}_T} \, \iff \, 
x_u \in \bar{\mathcal{O}_{\tau}} , \, \forall \tau > 0 \, \iff \, 
x_u \in \mathcal{N}[ \mathbf{O} ]$, where 
%
\begin{align*}
  \mathbf{O} = \left[ 
  \begin{array}{c}
    C 
    \\ \hdashline
    C A 
    \\ \hdashline
    C A^2
    \\ \hdashline
    \vdots
    \\ \hdashline
    C A^{n-1}
  \end{array}
  \right]
\end{align*}
%
Let's first show that $x_u \in \mathcal{N}[ \mathbf{O} ] \, \iff \, x_u \in \bar{\mathcal{O}_{\tau}} , \, \forall \tau > 0$. Let $x_u \in \mathcal{N}[ \mathbf{O} ]$, then  
%
\begin{align*}
  \mathbf{O} x_u &= 0
  \, 
    \rightarrow
  \,
  \begin{array}{c}
    C x_u = 0
    \\ \hdashline
    C A x_u = 0
    \\ \hdashline
    C A^2 x_u = 0
    \\ \hdashline
    \vdots
    \\ \hdashline
    C A^{n-1} x_u = 0
  \end{array}
\end{align*}
%
Moreover, by Cayley-Hamilton theorem, we can also conclude that $C A^{l} x_u = 0, \ \forall l \in \mathbf{Z}^+$. Now let's analyze the zero-input response of the system with $x(0) = x_u$
%
\begin{align*}
  x(\tau) = C e^{A \tau} x_u = 0 \, \Rightarrow \, x_u \in \bar{\mathcal{O}_{\tau}}
\end{align*}
%
and indeed it is true for all $\tau \in \mathbb{R}$. Now let's 
show that $x_u \in \bar{\mathcal{O}_T} \, \Rightarrow \, x_u \in \mathcal{N}[ \mathbf{O} ]$
%
\begin{align*}
  x(t) = C e^{A t} x_u = 0 , \, 
  \forall t \in [0,\tau] , \, \forall 
  \tau \in \mathbb{R} \, \Rightarrow \, x_u \in \bar{\mathcal{O}_{\tau}}
\end{align*}
%
Now let's show that $x_u \in \bar{\mathcal{O}_T} \, \Rightarrow \, 
x_u \in \mathcal{N}[ \mathbf{O} ]$. If $x_u \in \bar{\mathcal{O}_T}$, then 
%
\begin{align*}
  x(t) = C e^{A t} x_u = 0 , \, \forall t \in [0,T] \, 
  \Rightarrow \, \begin{array}{c} 
  x(0) = 0 \, \Rightarrow \, C x_u = 0
  \\
  \left[\frac{d}{dt}x(t) \right]_{t=0} = 0 \, \Rightarrow \, C A x_u = 0
  \\
  \left[\frac{d^2}{dt^2}x(t) \right]_{t=0} = 0 \, \Rightarrow \, C A^2 x_u = 0
  \\
  \vdots
  \\
  \left[\frac{d^{n-1}}{dt^{n-1}}x(t) \right]_{t=0} = 0 \, \Rightarrow \, C A^{n-1} x_u = 0
  \end{array}
  \, \Rightarrow \, \mathbf{O} x_u = 0 
  \, \iff \, x_u \in \mathcal{N}[ \mathbf{O} ]
\end{align*}
%
Similar to the reachability, we show that for CT-LTI systems
observability and unobservable (and observable) subspace
are independent of time. 

\subsection{CT Observability Grammian}

For a CT-LTI system, observability Grammian is defined as
%
\begin{align*}
    \textbf{Q}(t) = \int\limits_{0}^{t} e^{A^T (t - \tau)} C^T C e^{A (t - \tau)}  d \tau 
\end{align*}
%
\textbf{Theorem:} $\mathcal{N}[ \textbf{Q}(t) ] = \mathcal{N}[ \mathbf{O} ] \ \forall t > 0$. 

\textbf{Proof:} Let's first show that $\mathcal{N}[ \mathbf{O} ] \subset \mathcal{N}[ \textbf{Q}(t) ] \ \forall t > 0$. 
If $x_u \in \mathcal{N}[ \mathbf{O} ]$, then we know that $C A^{l} x_u = 0, \ \forall l \in \mathbb{Z}^{\geq 0}$. Let's anlayze 
if $x_u$ is in the null-space of $\textbf{Q}(t)$
%
\begin{align*}
    \textbf{Q}(t) x_u = \int\limits_{0}^{t} e^{A^T (t - \tau)} C^T C e^{A (t - \tau)} x_u d \tau
    = 0 \, \Rightarrow \, x_u \in \mathcal{N}[ \textbf{Q}(t) ] \ \forall t > 0
\end{align*}
%
Now let's show that $\mathcal{N}[ \textbf{Q}(t) ] \subset \mathcal{N}[ \mathbf{O} ], \, \forall t > 0$.
Let $x_u \in \mathcal{N}[ \textbf{Q}(t) ]$, then
\begin{align*}
    \textbf{Q}(t) x_u = 0 \, \Rightarrow \, x_u^T \textbf{Q}(t) x_u \, \iff \, \int\limits_{0}^{t} x_u^T e^{A^T (t - \tau)} C^T C e^{A (t - \tau)} x_u d \tau
    = 0 \, \iff \, C e^{A (t - \tau)} x_u = 0 \, \forall \tau \in [0 , t]
\end{align*}
%
Then we know that 
%
\begin{align*}
  [ C e^{A \eta} x_u]_{\eta = 0} = 0 \, &\Rightarrow \, C x_u = 0
  \\
  \frac{d}{d\eta}[ C e^{A \eta} x_u ]_{\eta = 0} = 0 \, &\rightarrow \, C A x_u = 0
  \\
  \frac{d^2}{d\eta^2}[ C e^{A \eta} x_u ]_{\eta = 0} = 0 \, &\Rightarrow \, C A^2 x_u = 0
  \\
  &\vdots
    \\
    \frac{d^{n-1}}{d\eta^{n-1}}[ C e^{A \eta} x_u ]_{\eta = 0} = 0 \, &\Rightarrow \, C A^{n-1} x_u = 0
    \\
    &\Rightarrow \, \mathbf{O} x_u = 0 \, \Rightarrow \, x_u \in \mathcal{N}[ \mathbf{O} ]
    \, \Rightarrow \, \mathcal{N}[ \textbf{Q}(t) ] \subset \mathcal{N}[ \mathbf{O} ] \, \forall t > 0
\end{align*}

\begin{exmp}
Show that
\end{exmp}
if $\dot{x} = A x$ is asymptotically stable, then observability Grammian at $t \to \infty$, $Q := \mathbf{Q}_{\infty}$,
satisfies the following Lyapunov equation
%
\begin{align*}
    A Q + Q A^T = - C^T C
\end{align*}
% 
, and this Lyapunov equation has a (unique) positive-definite solution for $Q$,
if and only if, $(A,C)$ is fully observable.

\subsection{Futher Results in CT-LTI Observability}

In view of duality, we can use our reachability results to immediately derive various conclusions, tests,
standard and canonical forms, etc., for observable and unobservable systems. The reader is highly encouraged to 
derive the following results, theorems, and claims by referring to the dual results in the reachability lecture.

\textbf{Result 1:} The unobservable sub-space, $\mathcal{N} [\mathbf{O}] $ is $A$ invariant, i.e. $x\in \mathcal{N} [\mathbf{O}]  \ \Rightarrow \ A x \in \mathcal{N} [\mathbf{O}] $. 

\textbf{Result 2:} $(A,C)$ pair is unobservable $\iff$ $C v = 0$ for some right eigenvector of $A$, i.e. $A v = \lambda v$ $\iff$
\begin{align*}
\mathrm{rank} \left[ \begin{array}{c} \lambda I - A \\ \hline C \end{array} \right] = n , \ \forall \lambda \in \mathcal{C}
\end{align*}

\textbf{Result 3:} Observability is invariant under state/similarity transformation. 

\subsection{Standard Form of Unobservable Systems}

Let $\dot{x} = A x + B u \ , \ y = C x + D u$ be an unobservable system; then it can be convenient and practical to choose coordinates (via similarity transformation)
to highlight unobservable and observable ``spaces''. Let 
%
\begin{align*}
   \mathrm{dim}({\mathcal{\mathbf{O}}}) &= \bar{o} 
   \ \& \
   T = \left[ \begin{array}{c|c} T_2 & T_{\bar{o}}  \end{array} \right] \ \mathrm{where} \ T_{\bar{o}} \in \mathbb{R}^{n \times \bar{o}} \ , 
   T_2 \in \mathbb{R}^{n \times (n-\bar{o})} \ , 
   \\
   T^{-1} A T &= \bar{A} \ , \ C T = \bar{C}
\end{align*}
%
Let's choose a $T_{\bar{o}}$ such that $\mathrm{Ra}(T_{\bar{o}}) = \mathcal{N}(\mathbf{O}) = \bar{\mathcal{O}}$, in other words columns of $T_{\bar{o}}$
span the null-space of the observability matrix (i.e. non-observable subspace).
Similarly, let's choose $T_2$ such that columns of $T$ are linearly independent. Let's analyze the similarity transformation. 
%
\begin{align*}
    A T &=  A \left[ \begin{array}{c|c} T_2 & T_{\bar{o}} \end{array} \right] = T \bar{A} = \left[ \begin{array}{c|c} T_2 & T_{\bar{o}} \end{array} \right]
    \left[ \begin{array}{c|c} A_{11} & A_{12} \\ \hline A_{21} & A_{22}  \end{array} \right]
    \\
    A T_{\bar{o}} &= T_2 A_{12} + T_{\bar{o}} A_{22}
\end{align*}
%
Note that the unobservable sub-space is $A$ invariant $A T_{\bar{o}}$ must remain in $\mathrm{Ra}(T_{\bar{o}}) = \mathcal{N}(\mathbf{O}) = \bar{\mathcal{O}}$,
since $T_2$ is composed of linearly independent columns, in order $T_2 A_{12} + T_{\bar{o}} A_{22}$ remain in $\mathrm{Ra}(T_{\bar{o}})$, $A_{12} = \mathbf{0}$.
Now let's analyze the effect of transformation on the output matrix.
%
\begin{align*}
    \bar{C} &= C T = C \left[ \begin{array}{c|c} T_2 & T_{\bar{o}} \end{array} 
    \right] = \left[ \begin{array}{c|c} C_1 & C_2  \end{array} \right]
    \\
    C_2 &= C T_{\bar{o}} = 0
\end{align*}
%
thus a standard unobservable form takes the form
%
\begin{align*}
\dot{\bar{x}} &= \left[ \begin{array}{c|c} A_{11} & 0 \\ \hline A_{21} & A_{22}  \end{array} \right] \bar{x} + \left[ \begin{array}{c} B_1 \\ B_2 \end{array} \right] u
\\
y &= \left[ \begin{array}{c|c} C_1 & 0 \end{array} \right] x + D u
\end{align*}

\section{Observability in DT-LTI Systems}

\textbf{Definition:} For LTI a discrete-time state-space representation
%
\begin{align*}
  x[k+1] &= A[k] x + B[k] u
\\
  y[k] &= C[k] x + D[k] u
\end{align*}
%
A state $x_u$ is said to be $m$-step unobservable 
if with $x[0] = x_u$ and $\forall \, u[k] \, k \in [0,m)$ we get the same $y[k] \, k \in [0,m)$ as we would with $x[0] = 0$. 

The set, $\bar{\mathcal{O}_m}$, of all $m$-step unobservale states forms a vecotor space, $\bar{\mathcal{O}_m} \subset 
\mathbb{R}^n$.

Similar to the CT-LTI case, we will only analyze zero-input response of the system in our 
observability-related derivations. 

If $x_u \in \bar{\mathcal{O}_m}$, then 
%
\begin{align*}
y[k] = C A^k x_u = 0 \ , \ \forall k \in [0,m)
\, , \, \Rightarrow \, \begin{array}{c} C x_u = 0 
\\
C A x_u = 0
\\
\vdots
\\
C A^{m-1} x_u = 0
\end{array}
\, \Rightarrow \, 
\left[ \begin{array}{c} C 
\\
C A x_u 
\\
\vdots
\\
C A^{m-1} x_u 
\end{array} \right] x_u = \mathbf{O}_m x_u = 0
\end{align*}
%
This implies that $x_u$ is $m$-step unobservable if and only if
$x_u \in \mathcal{N}[ \mathbf{O}_m ]$

\textbf{Theorem:} $\mathcal{N}[ \mathbf{O}_m ] \subset \mathcal{N}[ \mathbf{O}_n ] = \mathcal{N}[ \mathbf{O}_i ]$ for $m < n < l$.

\textbf{Corollary:} If $x_u$ is $n$-step observable, then it is $i$-step observable $\forall k \in \mathbb{Z}^+$.

In this context for DT systems, unobservable space is defined as $\bar{O} = \mathcal{N}[ \mathbf{O}_n ]$
and the system is fully observable if and only if 
\begin{align*}
 \bar{O} = {0} \, \iff \, \mathrm{dim}( \mathcal{N}[ \mathbf{O}_n ] ) = 0 \, \iff \,
 \mathrm{rank}[ \mathbf{O}_n ] = n
\end{align*}

\subsection{DT Observability Grammian}

$m$-step observability Grammian is defined as
\begin{align*}
 Q_m = \mathbf{O}_m^T \mathbf{O} = \sum\limits_{i=0}^{m-1} (A^i)^T C^T C A^i 
\end{align*}

\textbf{Theorem:} $\mathcal{N}[ \mathbf{O}_m ] = \mathcal{N}[ \mathbf{Q}_m ]$.

Let $x[k+1] = A x[k]$ be asymptotically stable, then $Q = \mathbf{Q}_{\infty}$ is well defined 
and satisfies the Lyapunov equation below
%
\begin{align*}
 A^T Q + Q A = - C^T C    
\end{align*}
%
and this Lyapunov equation has a (unique) positive-definite solution for $Q$,
if and only if, $(A,C)$ is fully observable.

\vspace{6pt}

\textbf{Corollary:} Total observability theorem
\begin{itemize}
 \item System is fully observable $\iff$ $(A,C)$ pair is observable $\iff$
 \item $\mathrm{rank}[\mathbf{O}] = n$, where $\mathbf{O} = \mathbf{O}_n$ $\iff$ 
 \item $C v \neq 0 \, , \, \forall v \in \lbrace v | v \neq 0 \, , \, A v = \lambda v , \, \lambda \in \mathcal{C} \rbrace$ $\iff$ 
 \item $\mathrm{rank} \left[ \begin{array}{c} \lambda I - A \\ \hline C \end{array} \right] = n , \ \forall \lambda \in \mathcal{C}$
\end{itemize}



% **** This ENDS THE EXAMPLES. DON'T DELETE THE FOLLOWING LINE:
\end{document}
