% 


\documentclass[twoside]{article}
\setlength{\oddsidemargin}{0.25 in}
\setlength{\evensidemargin}{-0.25 in}
\setlength{\topmargin}{-0.6 in}
\setlength{\textwidth}{6.5 in}
\setlength{\textheight}{8.5 in}
\setlength{\headsep}{0.75 in}
\setlength{\parindent}{0 in}
\setlength{\parskip}{0.1 in}

%
% ADD PACKAGES here:
%

\usepackage{amsmath,amsfonts,graphicx,arydshln}


\newcounter{lecnum}
\renewcommand{\thepage}{\thelecnum-\arabic{page}}
\renewcommand{\thesection}{\thelecnum.\arabic{section}}
\renewcommand{\theequation}{\thelecnum.\arabic{equation}}
\renewcommand{\thefigure}{\thelecnum.\arabic{figure}}
\renewcommand{\thetable}{\thelecnum.\arabic{table}}

%
% The following macro is used to generate the header.
%
\newcommand{\lecture}[4]{
   \pagestyle{myheadings}
   \thispagestyle{plain}
   \newpage
   \setcounter{lecnum}{#1}
   \setcounter{page}{1}
   \noindent
   \begin{center}
   \framebox{
      \vbox{\vspace{2mm}
    \hbox to 6.28in { {\bf EE402 - Discrete Time Systems
	\hfill Spring 2018} }
       \vspace{4mm}
       \hbox to 6.28in { {\Large \hfill Lecture #1 \hfill} }
       \vspace{2mm}
       \hbox to 6.28in { {\it Lecturer: #2 \hfill } }
      \vspace{2mm}}
   }
   \end{center}
   \markboth{Lecture #1}{Lecture #1}

   \vspace*{4mm}
}

\renewcommand{\cite}[1]{[#1]}
\def\beginrefs{\begin{list}%
        {[\arabic{equation}]}{\usecounter{equation}
         \setlength{\leftmargin}{2.0truecm}\setlength{\labelsep}{0.4truecm}%
         \setlength{\labelwidth}{1.6truecm}}}
\def\endrefs{\end{list}}
\def\bibentry#1{\item[\hbox{[#1]}]}


\newcommand{\fig}[3]{
			\vspace{#2}
			\begin{center}
			Figure \thelecnum.#1:~#3
			\end{center}
	}

% Use these for theorems, lemmas, proofs, etc.
\newtheorem{theorem}{Theorem}[lecnum]
\newtheorem{lemma}[theorem]{Lemma}
\newtheorem{proposition}[theorem]{Proposition}
\newtheorem{claim}[theorem]{Claim}
\newtheorem{corollary}[theorem]{Corollary}
\newtheorem{definition}[theorem]{Definition}
\newenvironment{proof}{{\bf Proof:}}{\hfill\rule{2mm}{2mm}}
\newtheorem{exmp}[theorem]{Ex}

% **** IF YOU WANT TO DEFINE ADDITIONAL MACROS FOR YOURSELF, PUT THEM HERE:

\begin{document}

% Lecture Details
\lecture{11}{Asst. Prof. M. Mert Ankarali}


%%%%%%%%%%%%%%%%%%%%%%%%%%

\section{Observability in CT-LTI Systems}

In the context of observability of dynamical systems, it 
turns out that it is more convenient to think in terms of
``un-observabile states'' and then connect it to the 
concept of observability and fully observable systems.
as reflected in the following definitions.

\textbf{Definition:} For LTI a contınıous-time state-space representation
%
\begin{align*}
  \dot{x} &= A x + B u
\\
  y &= C x + D u
\end{align*}
%
A state $x_u$ is said to be unobservable over $t \in [0 , T )$,
if with $x(0) = x_u$ and $\forall \, u(t)$ we get the same $y(t)$ as we would with $x(0) = 0$. 

The set, $\bar{\mathcal{O}_T}$, of all unobservale states over $t \in [0 , T )$ forms a vecotor space, $\bar{\mathcal{O}_T} \subset 
\mathbb{R}^n$, and the system is called fully observable over
$t \in [0 , T )$, if $\mathrm{dim}[\bar{\mathcal{O}_T}] = 0$.

Note for linear dynamical systems observability of a state and
system is independent from $u(t)$, in that respect we will 
only analuze zero-input response of the system in our derivations.

\textbf{Theorem:} $x_u \in \bar{\mathcal{O}_T} \, \iff \, 
x_u \in \bar{\mathcal{O}_{\tau}} , \, \forall \tau > 0 \, \iff \, 
x_u \in \mathcal{N}[ \mathbf{O} ]$, where 
%
\begin{align*}
  \mathbf{O} = \left[ 
  \begin{array}{c}
    C 
    \\ \hdashline
    C A 
    \\ \hdashline
    C A^2
    \\ \hdashline
    \vdots
    \\ \hdashline
    C A^{n-1}
  \end{array}
  \right]
\end{align*}
%
Let's first show that $x_u \in \mathcal{N}[ \mathbf{O} ] \, \iff \, x_u \in \bar{\mathcal{O}_{\tau}} , \, \forall \tau > 0$. Let $x_u \in \mathcal{N}[ \mathbf{O} ]$, then  
%
\begin{align*}
  \mathbf{O} x_u &= 0
  \, 
    \rightarrow
  \,
  \begin{array}{c}
    C x_u = 0
    \\ \hdashline
    C A x_u = 0
    \\ \hdashline
    C A^2 x_u = 0
    \\ \hdashline
    \vdots
    \\ \hdashline
    C A^{n-1} x_u = 0
  \end{array}
\end{align*}
%
Moreover, by Cayley-Hamilton theorem, we can also conclude that $C A^{l} x_u = 0, \ \forall l \in \mathbf{Z}^+$. Now let's analyze the zero-input response of the system with $x(0) = x_u$
%
\begin{align*}
  x(\tau) = C e^{A \tau} x_u = 0 \, \Rightarrow \, x_u \in \bar{\mathcal{O}_{\tau}}
\end{align*}
%
and indeed it is true for all $\tau \in \mathbb{R}$. Now let's 
show that $x_u \in \bar{\mathcal{O}_T} \, \Rightarrow \, x_u \in \mathcal{N}[ \mathbf{O} ]$
%
\begin{align*}
  x(t) = C e^{A t} x_u = 0 , \, 
  \forall t \in [0,\tau] , \, \forall 
  \tau \in \mathbb{R} \, \Rightarrow \, x_u \in \bar{\mathcal{O}_{\tau}}
\end{align*}
%
Now let's show that $x_u \in \bar{\mathcal{O}_T} \, \Rightarrow \, 
x_u \in \mathcal{N}[ \mathbf{O} ]$. If $x_u \in \bar{\mathcal{O}_T}$, then 
%
\begin{align*}
  x(t) = C e^{A t} x_u = 0 , \, \forall t \in [0,T] \, 
  \Rightarrow \, \begin{array}{c} 
  x(0) = 0 \, \Rightarrow \, C x_u = 0
  \\
  \left[\frac{d}{dt}x(t) \right]_{t=0} = 0 \, \Rightarrow \, C A x_u = 0
  \\
  \left[\frac{d^2}{dt^2}x(t) \right]_{t=0} = 0 \, \Rightarrow \, C A^2 x_u = 0
  \\
  \vdots
  \\
  \left[\frac{d^{n-1}}{dt^{n-1}}x(t) \right]_{t=0} = 0 \, \Rightarrow \, C A^{n-1} x_u = 0
  \end{array}
  \, \Rightarrow \, \mathbf{O} x_u = 0 
  \, \iff \, x_u \in \mathcal{N}[ \mathbf{O} ]
\end{align*}
%
Similar to the reachability, we show that for CT-LTI systems
observability and unobservable (and observable) subspace
are independent of time. 

\subsection{Observability Grammian}

For a CT-LTI system, observability Grammian is defined as
%
\begin{align*}
    \textbf{Q}(t) = \int\limits_{0}^{t} e^{A^T (t - \tau)} C^T C e^{A (t - \tau)}  d \tau 
\end{align*}
%
\textbf{Theorem:} $\mathcal{N}[ \textbf{Q}(t) ] = \mathcal{N}[ \mathbf{O} ] \ \forall t > 0$. 

\textbf{Proof:} Let's first show that $\mathcal{N}[ \mathbf{O} ] \subset \mathcal{N}[ \textbf{Q}(t) ] \ \forall t > 0$. 
If $x_u \in \mathcal{N}[ \mathbf{O} ]$, then we know that $C A^{l} x_u = 0, \ \forall l \in \mathbb{Z}^{\geq 0}$. Let's anlayze 
if $x_u$ is in the null-space of $\textbf{Q}(t)$
%
\begin{align*}
    \textbf{Q}(t) x_u = \int\limits_{0}^{t} e^{A^T (t - \tau)} C^T C e^{A (t - \tau)} x_u d \tau
    = 0 \, \Rightarrow \, x_u \in \mathcal{N}[ \textbf{Q}(t) ] \ \forall t > 0
\end{align*}
%
Now let's show that $\mathcal{N}[ \textbf{Q}(t) ] \subset \mathcal{N}[ \mathbf{O} ], \, \forall t > 0$.
Let $x_u \in \mathcal{N}[ \textbf{Q}(t) ]$, then
\begin{align*}
    \textbf{Q}(t) x_u = 0 \, \Rightarrow \, x_u^T \textbf{Q}(t) x_u \, \iff \, \int\limits_{0}^{t} x_u^T e^{A^T (t - \tau)} C^T C e^{A (t - \tau)} x_u d \tau
    = 0 \, \iff \, C e^{A (t - \tau)} x_u = 0 \, \forall \tau \in [0 , t]
\end{align*}
%
Then we know that 
%
\begin{align*}
  [ C e^{A \eta} x_u]_{\eta = 0} = 0 \, &\Rightarrow \, C x_u = 0
  \\
  \frac{d}{d\eta}[ C e^{A \eta} x_u ]_{\eta = 0} = 0 \, &\rightarrow \, C A x_u = 0
  \\
  \frac{d^2}{d\eta^2}[ C e^{A \eta} x_u ]_{\eta = 0} = 0 \, &\Rightarrow \, C A^2 x_u = 0
  \\
  &\vdots
    \\
    \frac{d^{n-1}}{d\eta^{n-1}}[ C e^{A \eta} x_u ]_{\eta = 0} = 0 \, &\Rightarrow \, C A^{n-1} x_u = 0
    \\
    &\Rightarrow \, \mathbf{O} x_u = 0 \, \Rightarrow \, x_u \in \mathcal{N}[ \mathbf{O} ]
    \, \Rightarrow \, \mathcal{N}[ \textbf{Q}(t) ] \subset \mathcal{N}[ \mathbf{O} ] \, \forall t > 0
\end{align*}

\begin{exmp}
Show that
\end{exmp}
if $\dot{x} = A x$ is asymptotically stable, then observability Grammian at $t \to \infty$, $Q := \mathbf{Q}_{\infty}$,
satisfies the following Lyapunov equation
%
\begin{align*}
    A Q + Q A^T = - C^T C
\end{align*}
% 
, and this Lyapunov equation has a (unique) positive-definite solution for $Q$,
if and only if, $(A,C)$ is fully observable.

\subsection{Futher Results in CT-LTI Observability}

In view of duality, we can use our reachability results to immediately derive various conclusions, tests,
standard and canonical forms, etc., for observable and unobservable systems. The reader is highly encouraged to 
derive the following results, theorems, and claims by referring to the dual results in the reachability lecture.

\textbf{Result 1:} The unobservable sub-space, $\mathcal{N} [\mathbf{O}] $ is $A$ invariant, i.e. $x\in \mathcal{N} [\mathbf{O}]  \ \Rightarrow \ A x \in \mathcal{N} [\mathbf{O}] $. 

\textbf{Result 2:} $(A,C)$ pair is unobservable $\iff$ $C v = 0$ for some right eigenvector of $A$, i.e. $A v = \lambda v$ $\iff$
\begin{align*}
\mathrm{rank} \left[ \begin{array}{c} \lambda I - A \\ \hline C \end{array} \right] = n , \ \forall \lambda \in \mathcal{C}
\end{align*}

\textbf{Result 3:} Observability is invariant under state/similarity transformation. 

% **** This ENDS THE EXAMPLES. DON'T DELETE THE FOLLOWING LINE:
\end{document}
