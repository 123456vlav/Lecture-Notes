\documentclass[twoside]{article}
\setlength{\oddsidemargin}{0.25 in}
\setlength{\evensidemargin}{-0.25 in}
\setlength{\topmargin}{-0.6 in}
\setlength{\textwidth}{6.5 in}
\setlength{\textheight}{8.5 in}
\setlength{\headsep}{0.75 in}
\setlength{\parindent}{0 in}
\setlength{\parskip}{0.1 in}

%
% ADD PACKAGES here:
%

\usepackage{amsmath,amsfonts,graphicx,mathtools}

%
\newcounter{lecnum}
\renewcommand{\thepage}{\thelecnum-\arabic{page}}
\renewcommand{\thesection}{\thelecnum.\arabic{section}}
\renewcommand{\theequation}{\thelecnum.\arabic{equation}}
\renewcommand{\thefigure}{\thelecnum.\arabic{figure}}
\renewcommand{\thetable}{\thelecnum.\arabic{table}}

%
% The following macro is used to generate the header.
%
\newcommand{\lecture}[4]{
   \pagestyle{myheadings}
   \thispagestyle{plain}
   \newpage
   \setcounter{lecnum}{#1}
   \setcounter{page}{1}
   \noindent
   \begin{center}
   \framebox{
      \vbox{\vspace{2mm}
    \hbox to 6.28in { {\bf EE502 - Linear Systems Theory II
	\hfill Spring 2032} }
       \vspace{4mm}
       \hbox to 6.28in { {\Large \hfill Lecture #1 \hfill} }
       \vspace{2mm}
       \hbox to 6.28in { {\it Lecturer: #2 \hfill } }
      \vspace{2mm}}
   }
   \end{center}
   \markboth{Lecture #1}{Lecture #1}

   \vspace*{4mm}
}


\renewcommand{\cite}[1]{[#1]}
\def\beginrefs{\begin{list}%
        {[\arabic{equation}]}{\usecounter{equation}
         \setlength{\leftmargin}{2.0truecm}\setlength{\labelsep}{0.4truecm}%
         \setlength{\labelwidth}{1.6truecm}}}
\def\endrefs{\end{list}}
\def\bibentry#1{\item[\hbox{[#1]}]}

%Use this command for a figure; it puts a figure in wherever you want it.
%usage: \fig{NUMBER}{SPACE-IN-INCHES}{CAPTION}
\newcommand{\fig}[3]{
			\vspace{#2}
			\begin{center}
			Figure \thelecnum.#1:~#3
			\end{center}
	}
% Use these for theorems, lemmas, proofs, etc.
\newtheorem{theorem}{Theorem}[lecnum]
\newtheorem{lemma}[theorem]{Lemma}
\newtheorem{proposition}[theorem]{Proposition}
\newtheorem{claim}[theorem]{Claim}
\newtheorem{corollary}[theorem]{Corollary}
\newtheorem{definition}[theorem]{Definition}
\newenvironment{proof}{{\bf Proof:}}{\hfill\rule{2mm}{2mm}}
\newtheorem{exmp}[theorem]{Ex}

% **** IF YOU WANT TO DEFINE ADDITIONAL MACROS FOR YOURSELF, PUT THEM HERE:

\begin{document}

% Lecture Details
\lecture{8}{Assoc. Prof. M. Mert Ankarali}



\section{Internal (Lyapunov) Stability} 

In internal stability, we are interested in un-driven (zero-input response) part of the dynamical system and solely 
focus on state evolution dynamics, i.e. autonomous part of the dynamical system. CT and DT non-linear autonomous 
systems can simply be expressed by 
%
\begin{align*}
 x &\in \mathbb{D} \subset \mathbb{R}^n
 \\
 \dot{x} &= F(x , t) \\
  x[k+1] &= F(x[k]], k) 
\end{align*}
%
For non-linear systems, in order to define and analyze the stability of a dynamical system, we need to 
define equilibrium points (or nominal solutions), since we will technically analyze the stability around 
such points. An equilibrium point for CT and DT non-linear systems are defined as
%
\begin{align*}
  &x_e \in \mathbb{D}
  \\
  CT:& \ 0 = F(x_e , t) \ \forall t > t_0
  \\
  DT:& \ x_e = F(x_e , k) \ \forall k > k_0
\end{align*}
%
Obviously if a dynamical system at time $t_0$ (or $k_0)$ starts from an an equilibrium point, $x(t_0) = x_e$ 
(or $x[k_o] = x_e$), it will remain on the equilibrium point $\forall \ t \geq t_0$ (or $\forall \ k \geq k_0$). 
A non-linear system can have a single equilibrium point, $x_e \in \mathcal{E}, \, \mathrm{card}(\mathcal{E}) = 1$, have multiple finite 
number of equilibria, $x_e \in \mathcal{E}, \, \mathrm{card}(\mathcal{E}) = n_e < \infty$, or infinite number of equilibrium points, 
$x_e \in \mathcal{E}, \, \mathrm{card}(\mathcal{E}) = \infty$.

\begin{exmp}
	Show that for an LTI dynamical system, set of equilibrium points define a vector space. Then characterize this vector space. 
\end{exmp}

\textbf{Definition:} Without loss of generality, let's assume that the equilibrium point that is point of interest is located at the origin $x_e = 0$.
%
\begin{enumerate}
%
\item The system is called \textit{stable in the sense of Lyapunov (s.i.s.L)} around $x_e = 0$ if it satisfies
%
\begin{align*}
	\forall \epsilon > 0 , \ \exists \delta_L(\epsilon) \ s.t. \ if \ || x(t_0) || < \delta_L \ \rightarrow || x(t) || < \epsilon \ \forall t \geq t_0
\end{align*}
%
\item The system is called \textit{asymptotically stable} around  around $x_e = 0$ if it is 
\textit{stable in the sense of Lyapunov (s.i.s.L)} around $x_e = 0$ and \textit{locally attractive}, i.e.
%
\begin{align*}
	\exists \delta_a \ s.t. \ if \ || x(t_0) || < \delta_a \ \rightarrow \lim_{t \to \infty} || x(t) || = 0
\end{align*}
%
\item The system is called \textit{exponentially stable} around  around $x_e = 0$ if it is 
\textit{asymptotically stable} around $x_e = 0$ and satisfies 
%
\begin{align*}
	\exists \delta_e > 0 , \, \alpha > 0 , \, \sigma >0 \ s.t. \ if \ || x(t_0) || < \delta_e \ \rightarrow || x(t) || \leq \alpha || x(t_0) || e^{- \sigma t} \ \forall t \geq t_0
\end{align*}
%
\end{enumerate}

\textbf{Remark:} If above stability conditions are satisfied $\forall \, t_0 \in \mathbb{R}$, then we call the system around the equilibrium \textit{uniformly s.i.s.L}, 
\textit{uniformly asymptotically stable}, and \textit{uniformly exponentially stable} respectively. The difference between uniform and non-uniform
stability is (slightly) important for only time-varying non-linear systems. Thus we will not use uniform stability definition in this course. 

\textbf{Remark:} Note that as you can see the internal stability
definitions, \textit{s.i.s.L}, \textit{asymptotic stability}, and
\textit{exponentially stability}, are all local stability definitions
defined in the neighborhood of $x_e$. If asymptotic and exponential stability definition holds
for all initial conditions, i.e. $x(t_0) \in \mathbb{D}$, then we use the terms
\textit{globally asymptotically stable}, and \textit{globally exponentially stable}.

\begin{exmp}
    Consider the pendulum dynamics
    \begin{align*}
    \dot{x} = \begin{bmatrix} x_2 \\ -sin(x_1)\end{bmatrix} \ , \ \mathrm{where} \ \begin{bmatrix} x_1 \\ x_2 \end{bmatrix} = \begin{bmatrix}  \theta \\ \dot{\theta} \end{bmatrix}
\end{align*}
\end{exmp}
Analyze the stability of the dynamics around $x_e = \begin{bmatrix} 0 \\ 0 \end{bmatrix}$ using Lyapunov's stability definitions.

\textbf{Solution:} We know that a mechanical pendulum is an
energy-conserving system since there is no dissipative or active element. In that respect, at any time instant, we can write total energy as 
%
\begin{align*}
E(x) &= \frac{1}{2} x_2^2 + 1 - \cos{x_2}  \ , \ \mathrm{note} \ E(0) = 0
\end{align*}
%
Let $x(0) = \begin{bmatrix} x_1(0) \\ x_2(0) \end{bmatrix}$ and $|| x(0) ||_2 < \epsilon , \, \epsilon \in \mathbb{R}^+ $, then we know that $ (x_1(0)^2 + x_2(0)^2) < \epsilon^2 $. Due to the conservation of energy, we also know that
%
\begin{align*}
&E(x(t)) = E(x(0)) \\
&\frac{1}{2} x_2(t)^2 + 1 - \cos{x_2(t)} 
= 
\frac{1}{2} x_2(0)^2 + 1 - \cos{x_2(0)} < \frac{1}{2} x_2(0)^2 + \frac{1}{2} x_1(0)^2 < \frac{\epsilon^2}{2}
\\
&\frac{1}{2} x_2(0)^2 + 1 - \cos{x_2(0)} < \frac{\epsilon^2}{2} 
\ \rightarrow \ x_2(0)^2 < \epsilon^2 + 4
\end{align*}
%
We already know that $x_1 = \theta$ and it is bounded since $\theta \in \mathbb{S}$, so 
%
\begin{align*}
x_1(t)^2 \leq 1 \ \rightarrow \ ( x_1(t)^2 + x_2(t)^2 ) < \epsilon^2 + 5 \ \rightarrow \ || x(t) ||_2 < \sqrt{\epsilon^2 + 5} = \delta(\epsilon)
\end{align*}
%
This shows that the dynamics around the equilibrium is \textit{stable in the sense of Lyapunov}. Note that 
since $E(t) = E(0) \, \forall t >0$, $|| x(t) ||_2 \neq 0 \forall t >0$, thus the system around the equilibrium is not asymptotically stable (neither local nor global), and hence not exponentially stable. 

\subsection{Internal Stability of LTI Systems}

Lyapunov's stability definitions may not be very useful for analyzing
the stability of non-linear systems, however we can easily derive
necessary and sufficient conditions for stability. One should also note the 
in LTI systems (and linear systems in general) we are interested in the
stability of the origin $x_e = 0$. The following matrix differential and difference equations express autonomous CT and DT LTI systems
%
\begin{align*}
\dot{x} &= A x 
\\
x[k+1] =& A x[k]
\end{align*}
%
and we know the analytical solutions to the zero-input responses have the following forms
%
\begin{align*}
x(t) = e^{A t} x_0
\\
x[k] = A^k x_0
\end{align*}
%
It is easier to analyze the internal stability using Jordan decomposition of the system matrix $A$, 
%
\begin{align*}
x(t) = G e^{J t} G^{-1} x_0 \ \rightarrow \  \left[ G^{-1} x(t) \right] = e^{J t} \left[ G^{-1} x_0 \right] \ \rightarrow \ \alpha(t) = e^{J t} \alpha_0
\\
x[k] = G J^k G^{-1} x_0  \ \rightarrow \ \alpha[k] = J^k \alpha_0
\end{align*}
%
Note that since $G$ and $G^{-1}$ finite and invertible matrices, we know that 
%
\begin{align*}
|| x || = 0 &\iff || G^{-1} x || = 0 \iff x = 0 
\\
|| x || < M_1 < \infty &\iff || G^{-1} x || < M_2 < \infty \ \mathrm{where} M_1 , M_2 \in \mathbb{R}
\\
|| x || \to \infty &\iff || G^{-1} x || \to \infty 
\end{align*}
%
We know that $e^{J t}$ and $J^k$ has the following block diagonal form
%
\begin{align*}
e^{J t} &= \left[ \begin{array}{ccccc} e^{J_1 t} &  & & &  \\  & e^{J_2 t}  &  & 0 &  \\ &  & \ddots & \\ & 0 & & \ddots & \\ & &  & &  e^{J_n t}  \end{array} \right]
\\
J^k &= \left[ \begin{array}{ccccc} J_1^k &  & & &  \\  & J_2^k  &  & 0 &  \\ &  & \ddots & \\ & 0 & & \ddots & \\ & &  & &  J_n^k  \end{array} \right]
\end{align*}
%
where an individual block associated with a Jordan block of $A$ has the following form
%
\begin{align*}
e^{J_i t} &= \left[  \begin{array}{cccccc} e^{\lambda_i t} & t e^{\lambda t} & \frac{t^2}{2 !} e^{\lambda_i t} 
& \cdots & \frac{t^{n-2}}{(n-2) !} e^{\lambda_i t}  & \frac{t^{n-1}}{(n-1) !} e^{\lambda_i t}
\\ 0 & e^{\lambda_i t} & t e^{\lambda_i t} & \frac{t^2}{2 !} e^{\lambda_i t} & \cdots  & \frac{t^{n-2}}{(n-2) !} e^{\lambda_i t}
\\  \vdots &  & \ddots &  &  & \vdots \\ 
& & & e^{\lambda_i t} & t e^{\lambda_i t} & \frac{t^2}{2 !} e^{\lambda_i t}
\\ 0 &  & \cdots  &  & e^{\lambda_i t} & t e^{\lambda_i t} \\
0 &  & \cdots &  & 0 & e^{\lambda_i t} \end{array} \right] 
\\
J_i^k &= \left[  \begin{array}{cccccc} \lambda_i^k & \begin{pmatrix} k \\ 1 \end{pmatrix} \lambda_i^{k-1} & \begin{pmatrix} k \\ 2 \end{pmatrix}  \lambda_i^{k-2} 
& \cdots & \begin{pmatrix} k \\ n\mathrm{-}2 \end{pmatrix} \lambda_i^{k-n+2} & \begin{pmatrix} k \\ n\mathrm{-}1 \end{pmatrix} \lambda_i^{k-n+1}
\\ 0 & \lambda_i^k & \begin{pmatrix} k \\ 1 \end{pmatrix}  \lambda_i^{k-1} & \begin{pmatrix} k \\ 2 \end{pmatrix} \lambda_i^{k-2} & \cdots  & \begin{pmatrix} k \\ n\mathrm{-}2 \end{pmatrix} \lambda_i^{k-n+2}
\\ 
\\  \vdots &  & \ddots &  &  & \vdots \\ 
\\ 
& & & \lambda_i^k & \begin{pmatrix} k \\ 1 \end{pmatrix}  \lambda_i^{k-1} & \begin{pmatrix} k \\ 2 \end{pmatrix}  \lambda_i^{k-2} 
\\ 0 &  & \cdots  &  & \lambda_i^k & \begin{pmatrix} k \\ 1 \end{pmatrix}  \lambda_i^{k-1} \\
0 &  & \cdots &  & 0 & \lambda_i^k \end{array} \right] 
\end{align*}
%
Based on this decomposition, we can derive the stability conditions.
%
\begin{enumerate}
\item 
\begin{itemize}
\item A CT LTI system is \textit{asympotitically \& exponentially stable} 

$\iff \ \lim_{t \to \infty } e^{J_i t} = 0 \ \forall i \ \iff \ \mathrm{Re}\lbrace \lambda_i \rbrace < 0 \ \forall i$  
\item A DT LTI system is \textit{asympotitically \& exponentially stable} 

$\iff \ \lim_{k \to \infty } J_i^k = 0 \ \forall i \ \iff \ | \lambda_i | < 1 \ \forall i$  
\end{itemize}
%
\item 
\begin{itemize}
\item A CT LTI system is \textit{s.i.s.L} 

$\iff \ \lim_{t \to \infty } e^{J_i t} = M_i < \infty \ \forall i \ \iff $ 
$ \lbrace \mathrm{Re}\lbrace \lambda_i \rbrace < 0 \ \mathrm{or} \ \left\lbrace \mathrm{Re}\lbrace \lambda_i \rbrace = 0 \ \mathrm{and} \ J_i \in \mathbb{R} \rbrace \right\rbrace \ \forall i$  
%
\item A DT LTI system is \textit{s.i.s.L} 

$\iff \ \lim_{k \to \infty } J_i^k = M_i < \infty \ \forall i \ \iff $ 
$ \lbrace | \lambda_i | < 1 \ \mathrm{or} \ \left\lbrace | \lambda_i | = 1 \ \mathrm{and} \ J_i \in \mathbb{R} \rbrace \right\rbrace \ \forall i$  
\end{itemize}

\item 
\begin{itemize}
\item A CT LTI system is \textit{unstable}

$\iff \ \exists i \ \ \lim_{t \to \infty } e^{J_i t} = \infty \ \iff $ 
$ \lbrace \mathrm{Re}\lbrace \lambda_i \rbrace > 0 \ \mathrm{or} \ \left\lbrace \mathrm{Re}\lbrace \lambda_i \rbrace = 0 \ \mathrm{and} \ J_i \in \mathbb{R}^{n \times n} , n>1 \rbrace \right\rbrace \ \forall i$  
%
\item A DT LTI system is \textit{unstable} 

$\iff \ \exists i \ \lim_{k \to \infty } J_i^k = \infty \ \iff $ 
$ \exists i \lbrace | \lambda_i | > 1 \ \mathrm{or} \ \left\lbrace | \lambda_i | = 1 \ \mathrm{and} \ J_i \in \mathbb{R}^{n \times n} , n>1 \rbrace \right\rbrace $  
\end{itemize}

\end{enumerate}

\subsection{Internal Stability of LTV Systems}

Indeed, for LTI systems we derived the conditions of stability using the state-transition matrix, 
which leads us to some conditions based on the eigenvalues of the system matrix. For general LTV systems,
if we follow Lyapunov's stability definitions, we can construct the stability conditions.

\textbf{Theroem:} The LTV system with the following state equation and state-transition matrix
%
\begin{align*}
 \dot{x}(t) = A(t) x(t) \ , \ x(t) = \Phi(t,t_0) x_0
\end{align*}
%
\begin{enumerate}
\item is \textit{s.i.s.L.} $\iff \ \exists M(t_0) \ s.t. \ || \Phi(t,t_0) || \leq M(t_0) < \infty \ , \ \forall t \geq t_0$
\item is \textit{asymptotically stable} $\iff$ if it is \textit{s.i.s.L.} and $ \lim_{t \to \infty} \Phi(t,t_0) = 0$
\item is \textit{exponentially stable} $\iff$ if it is \textit{asymptotically stable} and 
$ \exists \sigma(t_0) , C(t_0) > 0 $ s.t. $ || \Phi(t,t_0) || \leq C(t_0) e^{-\sigma(t_0) (t - t_0} \ \forall t \geq t_0$
\end{enumerate}
% 
If these stability definitions hold $\forall t_0 \in \mathbb{R}$, then they become \textit{uniformly stable} in their respective 
categories. Note that same conditions are valid for the stability of DT LTV systems if we replace time variables. 

\begin{exmp}
 Let 
 \begin{align*}
  x[k+1] = \begin{bmatrix} 0 & \rho^k  \\ 0 & (1/2)^k \end{bmatrix}
 \end{align*}
\end{exmp}
Analyze the stability of the origin based on the value of $\rho$

\textbf{Solution:} In LTI systems, we know that eigenvalues of the system matrix provide necessary and sufficient conditions on
Lyapunov, asymptotic, and exponential stability. We can see that for this LTV system, eigenvalues are independent of $k$, and
they satisfy LTI stability conditions, i.e.
%
\begin{align*}
  \lambda(A[k]) \in \lbrace 0 \ , \ 1/2 \rbrace \& | \lambda_{1,2} | < 1
\end{align*}
%
Let's try to find a solution to the initial value problem for the given LTV system and see if LTI intuition is valid or not for this LTV system. Let $x_{0,1} = \begin{bmatrix} 1 \\ 0 \end{bmatrix}$, then
%
\begin{align*}
   x[k]_1 = \begin{bmatrix} 0 \\ 0 \end{bmatrix}
\end{align*}
%
now let $x_{0,2} = \begin{bmatrix} 0 \\ 1 \end{bmatrix}$
%
%
\begin{align*}
   x[k]_2 = \begin{bmatrix} (\rho/2)^{k-1} \\ (1/2)^k \end{bmatrix} \ k>0
\end{align*}
%
thus we can write $\Phi[k,0]$ as
%
\begin{align*}
   \Phi[k,0] = \begin{bmatrix} x[k]_1 & x[k]_2 \end{bmatrix} = 
   \begin{bmatrix} 0 & (\rho/2)^k \\ 0 & (1/2)^{k-1} \end{bmatrix} \ k > 0 
\end{align*}
%
Based on this result, we can conclude that.
%
\begin{itemize}
    \item LTV system around the origin is asymptotically \& exponentially stable if $|\rho| < 2$
    %
    \item LTV system around the origin is stable i.s.L if $|\rho| \leq 2$
    %
    \item LTV system around the origin is unstable if $|\rho| > 2$
\end{itemize}
%
As you can see in this example (and in general) for LTV systems, eigenvalues of the system matrix $A[k]$ do not imply stability.

\begin{exmp}
 Let 
 \begin{align*}
  \dot{x} = \begin{bmatrix} 0 & e^{-t}  \\ 0 & 0 \end{bmatrix}
 \end{align*}
\end{exmp}
Analyze the stability of the origin 

\textbf{Solution:} Since 
\begin{align*}
  A(t) = \begin{bmatrix} 0 & e^{-t}  \\ 0 & 0 \end{bmatrix} =
  \begin{bmatrix} 0 & 1 \\ 0 & 0 \end{bmatrix} e^{-t} = \bar{A} f(t)
 \end{align*}
%
We can find the state transition matrix via
\begin{align*}
\Phi(t,t_0) = e^{\int\limits_{t_0}^t A(\tau) d\tau} = 
  \mathrm{exp} \begin{pmatrix}
      \int\limits_{t_0}^t \begin{bmatrix} 0 & e^{-\tau}  \\ 0 & 0 \end{bmatrix} d\tau
  \end{pmatrix}
  =  \mathrm{exp} \begin{pmatrix}  \begin{bmatrix} 0 & e^{-t_0} - e^{-t}  \\ 0 & 0 \end{bmatrix} 
  \end{pmatrix}
  = \begin{bmatrix} 1 & e^{-t_0} - e^{-t}  \\ 0 & 1 \end{bmatrix} 
\end{align*}
%
We can see that $ || \Phi(t,t_0) || < M < \infty \ \forall \ (t,t_0)$ but $\lim_{t\to\infty} \neq 0$
thus system around the origin is (only) stable in the sense of Lyapunov. However, if we had only 
checked the eigenvalues, we may have (wrongly) concluded that the system was unstable. 

\subsection{Lyapunov's Direct Method}

Consider an autonomous time-invariant CT or DT non-linear state-space mode;
%
\begin{align*}
    \dot{x} = F(x(t) \\
    x[k+1] = F(x[k]) \\
\end{align*}
%
with an equilibrium point $x_e$. Without loss of generality, let's assume that $x_e = 0$. Consider a continuous
scalar function $V : \mathbb{R}^{m} \mapsto \mathbb{R}$ (or  $V : D \mapsto F$ in general, where $D$ is the domain and $F$ is the field), where $V(0) = 0$ and $V(x) > 0$ for $x \neq 0$ inside some ``ball'' enclosing the equilibrium point. 
In that context, $V(x)$ can be considered as a function of ``energy'' (indeed, for a mechanical system, it can be a very good choice), or mathematically $V(x)$ can be a norm. Then $\dot{V}(x)$ for CT systems and $\triangle V(x) := V(x[k+1]) - V(x[k])$ 
DT systems can provide critical information regarding the internal stability of the dynamical system.

\subsubsection*{Lyapunov Functions, Lyapunov Theorem for Local \& Global Stability}

\textbf{Definition:} Let $V : \mathbb{R}^{m} \mapsto \mathbb{R}$. 
\begin{itemize}
\item We call $V(x)$ a \textit{locally positive definite (lpd)} function around $x = 0$ if 
\begin{enumerate}
    \item $V(0) = 0$
    \item $V(x) > 0, \ || x || \in (0 , r)$ for some $0 < r$
\end{enumerate}
%
\item We call $V(x)$ a \textit{locally negative definite (lnd)} function around $x = 0$ if 
\begin{enumerate}
    \item $V(0) = 0$
    \item $V(x) < 0, \ || x || \in (0 , r)$ for some $0 < r$
\end{enumerate}
%
%
\item We call $V(x)$ a \textit{locally positive semi definite (lpsd)} function around $x = 0$ if 
\begin{enumerate}
    \item $V(0) = 0$
    \item $V(x) \geq 0, \ || x || \in (0 , r)$ for some $0 < r$
\end{enumerate}
%
%
\item We call $V(x)$ a \textit{locally negative semi definite (lnsd)} function around $x = 0$ if 
\begin{enumerate}
    \item $V(0) = 0$
    \item $V(x) \leq 0, \ || x || \in (0 , r)$ for some $0 < r$
\end{enumerate}
%
\end{itemize}

\textbf{Definition:} Let $V(x)$ be a \textit{lpd} function, then we call it a ``candidate'' Lyapunov function. 
If $\dot{V}(x)$ (or $\triangle V(x)$) for the dynamical system of interest is a \textit{lnsd} function, then 
$V(x)$ is called a \textbf{Lyapunov function} of the system.

\textbf{Theorem:} If there exists a Lyapunov function of the system, $V(x)$, then
\begin{enumerate}
    \item Then $x_e = 0$ is a (locally) stable equilibrium point \textit{in the sense of Lyapunov}
    \item and if $\dot{V}(x) < 0$ (or $\triangle V(x) < 0$) $\ || x || \in (0 , r_a)$ (i.e. \textit{lsd}), then 
    $x_e = 0$ is a (locally) asymptotically stable equilibrium point.
\end{enumerate}

\textbf{Theorem:} If there exists a Lyapunov function of the system, $V(x)$, defined on the entire state-space (or domain in general), i.e. 
\begin{align*}
    &\exists V: \mathbb{R}^n \to \mathbb{R} \ s.t. \ V(0) = 0 \ 
    \ , \
    V(x) > 0, \ \forall x \in \mathbb{R}^n \ (i.e. nd) \ (i.e. \ pd) \ \mathrm{and}
    \\
    &\dot{V}(x) \leq 0, \ \forall x \in \mathbb{R}^n \ (i.e. \ nd)
\end{align*}
then 
\begin{enumerate}
    \item Then $x_e = 0$ is a stable equilibrium point \textit{in the sense of Lyapunov} (in a more global sense)
%    
    \item if $x_e = 0$ is a globally stable equilibrium point \textit{in the sense of Lyapunov} and if 
    \begin{align*}
    &\dot{V}(x) < 0, \ \forall x \in \mathbb{R}^n
    \\
    &|| x || \to \infty \implies V(x) \to \infty
\end{align*}
then $x_e = 0$ is a globally asymptotically stable equilibrium point.
\end{enumerate}

\begin{exmp}
    Let's go back to the pendulum dynamics
    \begin{align*}
    \dot{x} = \begin{bmatrix} x_2 \\ -sin(x_1)\end{bmatrix} \ , \ \mathrm{where} \ \begin{bmatrix} x_1 \\ x_2 \end{bmatrix} = \begin{bmatrix}  \theta \\ \dot{\theta} \end{bmatrix}
\end{align*}
\end{exmp}
Analyze the stability of the dynamics around $x_e = \begin{bmatrix} 0 \\ 0 \end{bmatrix}$ using Lyapunov's direct method.

\textbf{Solution:} We first need to find a candidate Lyapunov function. Let's test if
the total mechanical energy can be adopted as a Lyapunov candidate
%
\begin{align*}
V(x) &= E(x) = \frac{1}{2} x_2^2 + 1 - \cos{x_1}  \\ 
V(0) &= 0 \\
V(x) &> 0 \ , \ \forall x \neq 0
\end{align*}
%
Now let's test if $V(x)$ is a Lyapunov function.
%
\begin{align*}
\frac{d}{dt} V(x) &= - x_2 \sin x_1 + \sin(x_1) x_2 = 0
\\
\dot{V}(x) &\geq 0 \ \forall x 
\end{align*}
%
Thus we can conclude that $E(x)$ is a Lyapunov function and $x_e = 0$
is a stable equilibrium point in the sense of Lyapunov (globally). We know that since 
$x_{e,k} = \begin{bmatrix} k \pi \\ 0 \end{bmatrix} \forall k \in \mathbb{Z}$
is an equilibrium point, $x_e = 0$ is not globally asymptotically stable. 
Indeed, since $\dot{V}(x) = 0 \, \forall x $, we also know 
that $x_e = 0$ is not even locally asymptotically stable. However, one should
know the fact that Lyapunov's direct method, in general, does not imply 
un-stability.

\begin{exmp}
    Now let's add a linear damping term to the pendulum dynamics
    \begin{align*}
    \dot{x} = \begin{bmatrix} x_2 \\ -sin(x_1) - x_2 \end{bmatrix} \ , \ \mathrm{where} \ b > 0
\end{align*}
\end{exmp}
Analyze the stability of the dynamics around $x_e = \begin{bmatrix} 0 \\ 0 \end{bmatrix}$ using Lyapunov's direct method.

\textbf{Solution:} Total mechanical energy can again be a good candidate as
a Lyapunov function 
%
\begin{align*}
V(x) &= E(x) = \frac{1}{2} x_2^2 + 1 - \cos{x_1}  \\ 
V(0) &= 0 \ , \ V(x) > 0 \ , \ \forall x \neq 0
\\
\\
\frac{d}{dt} V(x) &= - x_2^2 - x_2 \sin x_1 + \sin(x_1) x_2 = - x_2^2 \geq 0
\\
\dot{V}(x) &= 0 ,\ \forall x \in \lbrace x | \begin{bmatrix}0 & 1\end{bmatrix} x = 0 \rbrace
\\
\dot{V}(x) &> 0 ,\ \forall x \in \lbrace x | \begin{bmatrix}0 & 1\end{bmatrix} x \neq 0 \rbrace
\end{align*}
%
In this case we can again conclude that $E(x)$ is a Lyapunov function and $x_e = 0$
is a stable equilibrium point in the sense of Lyapunov (globally). However, we can still not 
conclude that the system is asymptotically stable (even though intuitively, we know that it is 
at least locally asymptotically stable), since $\dot{V}(x)$ is a \textit{nsd} function 
(locally and globally). Let's try another candidate $V(x)$
%
\begin{align*}
V(x) &= \frac{1}{2} x_2^2 + a ( 1 - \cos{x_2} ) + \frac{b}{2} \begin{pmatrix} x_1 + x_2 \end{pmatrix}^2  ,
\, a > 0 , \,  b > 0
\\ 
V(0) &= 0 \ , \ V(x) > 0 \ , \ \forall x \neq 0 \ \rightarrow \ pd \ \checkmark
\\
\\
\frac{d}{dt} V(x) &= - x_2^2 - x_2 \sin(x_1) + a \sin(x_1) x_2 + b \begin{pmatrix} x_1 + x_2 \end{pmatrix} \dot{x_1} 
+ b \begin{pmatrix} x_1 + x_2 \end{pmatrix} \dot{x_2}
\\
&= (b - 1) x_2^2 + (a - 1) x_2 \sin(x_1) + b x_1 x_2 
+ b \begin{pmatrix} x_1 + x_2 \end{pmatrix} \dot{x_2}
\\
&= (b - 1) x_2^2 + (a - 1) \sin(x_1) x_2 + b x_1 x_2 
- b \sin(x_1) \begin{pmatrix} x_1 + x_2 \end{pmatrix} 
- b x_2\begin{pmatrix} x_1 + x_2 \end{pmatrix} 
\\
&= - x_2^2 + (a - b - 1) \sin(x_1) x_2 - b \sin(x_1) x_1 
\end{align*}
%
Let $(a,b) = (2,1)$, then 
\begin{align*}
\frac{d}{dt} V(x) &= - \left( x_2^2  + \sin(x_1) x_1 \right) \geq 0
\\
\frac{d}{dt} V(x) &= 0 \ \iff \ x_1 = k \pi \, k \in \mathbb{Z} \rightarrow \ l.n.d 
\end{align*}
%
We can now see that when $x_1 \in (\pi , \pi)$, $\dot{V}(x)$ is a locally negative 
definite function, the system around $x_e = 0$ is also locally asymptotically stable. 

\subsection{Lyapunov's Indirect Method}

Consider an autonomous time-invariant CT or DT non-linear state-space mode;
%
\begin{align*}
    \dot{x} = F(x(t) \\
    x[k+1] = F(x[k]) 
\end{align*}
%
with an equilibrium point $x_e$. Without loss of generality, let's assume that $x_e = 0$. Let the linearization of dynamics around the equilibrium yields
%
\begin{align*}
    \dot{\delta x} = A x(t) \\
    \delta x[k+1] = \delta x[k]) 
\end{align*}
%

\textbf{Theorem:} 
\begin{enumerate}
    \item If the linearized system matrix, $A$, is asymptotically/exponentially stable, then the non-linear system
    is also \textit{locally asymptotically stable} around the equilibrium point, $x_e$, inside a region locally defined around the equilibrium point
    %
    \item If the linearized system matrix, $A$, is asymptoticall/exponentially unstable, then the non-linear system
    is also \textit{locally unstable} around the equilibrium point
    %
    \item If the linearized system matrix, $A$, is stable i.s.L., we can not comment on the stability of the non-linear system around the equilibrium 
\end{enumerate}

\begin{exmp}
    Consider the following scalar DT dynamical system
    \begin{align*}
    x[k+1] = \frac{x[k]}{1 + (x[k])^2}
\end{align*}
\end{exmp}
Analyze the stability of the dynamics around $x_e = \begin{bmatrix} 0 \\ 0 \end{bmatrix}$ using both Lyapunov's direct and indirect methods.

\textbf{Solution:} Let's first start with the direct method and see if $V(x) = x^2$ is a Lyapunov function
%
\begin{align*}
V(x) &= x^2 \ , \ V(0) = 0 \ , \ V(x) > 0 \ , \ \forall x \neq 0 \ \rightarrow \ pd \ \checkmark
\\
\triangle V(x) &= x^2 \left( \frac{1}{(1 + x^2)^2} - 1 \right)  \ , \ \triangle V(0) = 0 \ , \ \triangle V(x) < 0 \ , \ \forall x \neq 0 \ \rightarrow \ nd \ \checkmark
\end{align*}
%
Thus we can conclude that the system around the origin is globally asymptotically stable. Now
let's adopt the indirect method
%
\begin{align*}
 \delta x[k+1] &= \frac{\partial 
 }{\partial} \left(
 \frac{x}{1 + (x)^2} \right)_{x=0} \delta x[k]
 \\
\delta x[k+1] &= \delta x[k]
\end{align*}
%
We can see that linearized dynamics has a single eigenvalue at $z = 1$, which implies that the linear dynamical system is stable i.s.L. However, based on only this result, we can not comment on the stability of the non-linear system. However, we know from the direct method that the system around the origin is, indeed, asymptotically stable.

\subsection{Quadratic Lyapunov Functions and Lyapunov Stability for LTI Systems}

Consider the function
%
\begin{align*}
V(x) = x^T P x , \ x \in \mathbb{R}^{n} , P \in \mathbb{R}^{n \times n} , \& \ P^T = P
\end{align*}
%
\textbf{Proposition:} $V(x)$ is a positive definite function if and only if all the eigenvalues of 
$P$ are positive

\textbf{Proof:} Since $P$ is a symmetric real matrix, we know that it can be diagonalized by an
\textit{orthagonal matirx}, i.e.
%
\begin{align*}
P = U^T \Lambda U , \ U^T U = I , \& \ \Lambda = 
\mathrm{diag}\begin{bmatrix} \lambda_1 , & \cdots , & \lambda_n  \end{bmatrix}
\end{align*}
%
Let $y = U x \ \rightarrow \ || y ||_2 = || x ||_2$, then
%
\begin{align*}
V(x) = x^T P x  = y^T \Lambda y = \sum\limits_{i=1}^n \lambda_i y_i^2 
\end{align*}
%
Thus
%
\begin{align*}
V(x) > 0 \ \forall x \neq 0 \ \iff \ \lambda_i > 0 \ \forall i 
\end{align*}
%
\textbf{Definition:} A symmetric matrix $P^T = P$ 
\begin{itemize}
    \item is called \textit{positive definite} if it satisfies 
%
\begin{align*}
x^T P x > 0 \ \forall x \neq 0
\end{align*}
%
and use the notation, $P > 0$ for such matrices
%
\item is called \textit{positive semi definite} if 
%
\begin{align*}
x^T P x \geq 0 \ \forall x \neq 0
\end{align*}
%
and use the notation, $P \geq 0$ for such matrices
\end{itemize}

\textbf{Corollaries:} 
\begin{itemize}
\item For a symmetric positive definite or symmetric positive semi-definite matrix 
%
\begin{align*}
 \lambda_{min}(P) || x ||_2^2 \leq x^T P x \leq \lambda_{max}(P) || x ||_2^2
\end{align*}
%
\item If $P \geq 0$ but $P \not> 0$, then $\lambda_{min} = 0$ 
\end{itemize}

\subsection{Quadratic Lyapunov Functions for LTI Systems}

Consider defining a Lyapunov function candidate of the form
%
\begin{align*}
V(x) = x^T P x  \ P > 0
\end{align*}
%
for an LTI CT or DT system. Then for a CT-LTI system, $\dot{x} = A x$, we have
%
\begin{align*}
\dot{V}(x) &= \dot{x}^T P x + x^T P x = x^T A^T P x + x^T P A x 
\\
&= x^T (A^T P + P A) x = - x^T Q x
\\
Q = - (A^T P + P A)
\end{align*}
%
note that $Q^T = Q$. Based on Lyapunov's direct method, we conclude that
\begin{itemize}
\item $\dot{x} = A x$ is stable i.s.L. if $-Q = (A^T P + P A) \geq 0$
%
\item $\dot{x} = A x$ is asymptotically stable if $-Q = (A^T P + P A) > 0$
\end{itemize}

% **** This ENDS THE EXAMPLES. DON'T DELETE THE FOLLOWING LINE:
\end{document}
