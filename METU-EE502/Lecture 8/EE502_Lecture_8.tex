

\documentclass[twoside]{article}
\setlength{\oddsidemargin}{0.25 in}
\setlength{\evensidemargin}{-0.25 in}
\setlength{\topmargin}{-0.6 in}
\setlength{\textwidth}{6.5 in}
\setlength{\textheight}{8.5 in}
\setlength{\headsep}{0.75 in}
\setlength{\parindent}{0 in}
\setlength{\parskip}{0.1 in}

%
% ADD PACKAGES here:
%

\usepackage{amsmath,amsfonts,graphicx,mathtools}

%
\newcounter{lecnum}
\renewcommand{\thepage}{\thelecnum-\arabic{page}}
\renewcommand{\thesection}{\thelecnum.\arabic{section}}
\renewcommand{\theequation}{\thelecnum.\arabic{equation}}
\renewcommand{\thefigure}{\thelecnum.\arabic{figure}}
\renewcommand{\thetable}{\thelecnum.\arabic{table}}

%
% The following macro is used to generate the header.
%
\newcommand{\lecture}[4]{
   \pagestyle{myheadings}
   \thispagestyle{plain}
   \newpage
   \setcounter{lecnum}{#1}
   \setcounter{page}{1}
   \noindent
   \begin{center}
   \framebox{
      \vbox{\vspace{2mm}
    \hbox to 6.28in { {\bf EE502 - Linear Systems Theory II
	\hfill Spring 2032} }
       \vspace{4mm}
       \hbox to 6.28in { {\Large \hfill Lecture #1 \hfill} }
       \vspace{2mm}
       \hbox to 6.28in { {\it Lecturer: #2 \hfill } }
      \vspace{2mm}}
   }
   \end{center}
   \markboth{Lecture #1}{Lecture #1}

   \vspace*{4mm}
}


\renewcommand{\cite}[1]{[#1]}
\def\beginrefs{\begin{list}%
        {[\arabic{equation}]}{\usecounter{equation}
         \setlength{\leftmargin}{2.0truecm}\setlength{\labelsep}{0.4truecm}%
         \setlength{\labelwidth}{1.6truecm}}}
\def\endrefs{\end{list}}
\def\bibentry#1{\item[\hbox{[#1]}]}

%Use this command for a figure; it puts a figure in wherever you want it.
%usage: \fig{NUMBER}{SPACE-IN-INCHES}{CAPTION}
\newcommand{\fig}[3]{
			\vspace{#2}
			\begin{center}
			Figure \thelecnum.#1:~#3
			\end{center}
	}
% Use these for theorems, lemmas, proofs, etc.
\newtheorem{theorem}{Theorem}[lecnum]
\newtheorem{lemma}[theorem]{Lemma}
\newtheorem{proposition}[theorem]{Proposition}
\newtheorem{claim}[theorem]{Claim}
\newtheorem{corollary}[theorem]{Corollary}
\newtheorem{definition}[theorem]{Definition}
\newenvironment{proof}{{\bf Proof:}}{\hfill\rule{2mm}{2mm}}
\newtheorem{exmp}[theorem]{Ex}

% **** IF YOU WANT TO DEFINE ADDITIONAL MACROS FOR YOURSELF, PUT THEM HERE:

\begin{document}

% Lecture Details
\lecture{8}{Assoc. Prof. M. Mert Ankarali}



\section{Internal (Lyapunov) Stability} 

In internal stability, we are interested in un-driven (zero-input response) part of the dynamical system and solely 
focus on state evolution dynamics, i.e. autonomous part of the dynamical system. CT and DT non-linear autonomous 
systems can simply be expressed by 
%
\begin{align*}
 x &\in \mathbb{D} \subset \mathbb{R}^n
 \\
 \dot{x} &= F(x , t) \\
  x[k+1] &= F(x[k]], k) 
\end{align*}
%
For non-linear systems, in order to define and analyze the stability of a dynamical system, we need to 
define equilibrium points (or nominal solutions), since we will technically analyze the stability around 
such points. An equilibrium point for CT and DT non-linear systems are defined as
%
\begin{align*}
  &x_e \in \mathbb{D}
  \\
  CT:& \ 0 = F(x_e , t) \ \forall t > t_0
  \\
  DT:& \ x_e = F(x_e , k) \ \forall k > k_0
\end{align*}
%
Obviously if a dynamical system at time $t_0$ (or $k_0)$ starts from an an equilibrium point, $x(t_0) = x_e$ 
(or $x[k_o] = x_e$), it will remain on the equilibrium point $\forall \ t \geq t_0$ (or $\forall \ k \geq k_0$). 
A non-linear system can have a single equilibrium point, $x_e \in \mathcal{E}, \, \mathrm{card}(\mathcal{E}) = 1$, have multiple finite 
number of equilibria, $x_e \in \mathcal{E}, \, \mathrm{card}(\mathcal{E}) = n_e < \infty$, or infinite number of equilibrium points, 
$x_e \in \mathcal{E}, \, \mathrm{card}(\mathcal{E}) = \infty$.

\begin{exmp}
	Show that for an LTI dynamical system, set of equilibrium points define a vector space. Then characterize this vector space. 
\end{exmp}

\textbf{Definition:} Without loss of generality, let's assume that the equilibrium point that is point of interest is located at the origin $x_e = 0$.
%
\begin{enumerate}
%
\item The system is called \textit{stable in the sense of Lyapunov (s.i.s.L)} around $x_e = 0$ if it satisfies
%
\begin{align*}
	\forall \epsilon > 0 , \ \exists \delta_L(\epsilon) \ s.t. \ if \ || x(t_0) || < \delta_L \ \rightarrow || x(t) || < \epsilon \ \forall t \geq t_0
\end{align*}
%
\item The system is called \textit{asymptotically stable} around  around $x_e = 0$ if it is 
\textit{stable in the sense of Lyapunov (s.i.s.L)} around $x_e = 0$ and \textit{locally attractive}, i.e.
%
\begin{align*}
	\exists \delta_a \ s.t. \ if \ || x(t_0) || < \delta_a \ \rightarrow \lim_{t \to \infty} || x(t) || = 0
\end{align*}
%
\item The system is called \textit{exponentially stable} around  around $x_e = 0$ if it is 
\textit{asymptotically stable} around $x_e = 0$ and satisfies 
%
\begin{align*}
	\exists \delta_e > 0 , \, \alpha > 0 , \, \sigma >0 \ s.t. \ if \ || x(t_0) || < \delta_e \ \rightarrow || x(t) || \leq \alpha || x(t_0) || e^{- \sigma t} \ \forall t \geq t_0
\end{align*}
%
\end{enumerate}

\textbf{Remark:} If above stability conditions are satisfied $\forall \, t_0 \in \mathbb{R}$, then we call the system around the equilibrium \textit{uniformly s.i.s.L}, 
\textit{uniformly asymptotically stable}, and \textit{uniformly exponentially stable} respectively. The difference between uniform and non-uniform
stability is (slightly) important for only time-varying non-linear systems. Thus we will not use uniform stability definition in this course. 

\textbf{Remark:} Note that as you can see the internal stability
definitions, \textit{s.i.s.L}, \textit{asymptotic stability}, and
\textit{exponentially stability}, are all local stability definitions
defined in the neighborhood of $x_e$. If a stability definition holds
for all initial conditions, i.e. $x(t_0) \in \mathbb{D}$, then we use the terms
\textit{globally s.i.s.L}, \textit{globally asymptotically stable}, and \textit{globally exponentially stable}.

\begin{exmp}
    Consider the pendulum dynamics
    \begin{align*}
    \dot{x} = \begin{bmatrix} x_2 \\ -sin(x_1)\end{bmatrix} \ , \ \mathrm{where} \ \begin{bmatrix}  \theta \\ \dot{\theta} \end{bmatrix}
\end{align*}
\end{exmp}
Analyze the stability of the dynamics around $x_e = \begin{bmatrix} 0 \\ 0 \end{bmatrix}$ using Lyapunov's stability definitions.

\textbf{Solution:} We know that a mechanical pendulum is an
energy-conserving system since there is no dissipative or active element. In that respect, at any time instant, we can write total energy as 
%
\begin{align*}
E(x) &= \frac{1}{2} x_2^2 + 1 - \cos{x_2}  \ , \ \mathrm{note} \ E(0) = 0
\end{align*}
%
Let $x(0) = \begin{bmatrix} x_1(0) \\ x_2(0) \end{bmatrix}$ and $|| x(0) ||_2 < \epsilon , \, \epsilon \in \mathbb{R}^+ $, then we know that $ (x_1(0)^2 + x_2(0)^2) < \epsilon^2 $. Due to the conservation of energy, we also know that
%
\begin{align*}
&E(x(t)) = E(x(0)) \\
&\frac{1}{2} x_2(t)^2 + 1 - \cos{x_2(t)} 
= 
\frac{1}{2} x_2(0)^2 + 1 - \cos{x_2(0)} < \frac{1}{2} x_2(0)^2 + \frac{1}{2} x_1(0)^2 < \frac{\epsilon^2}{2}
\\
&\frac{1}{2} x_2(0)^2 + 1 - \cos{x_2(0)} < \frac{\epsilon^2}{2} 
\ \rightarrow \ x_2(0)^2 < \epsilon^2 + 4
\end{align*}
%
We already know that $x_1 = \theta$ and it is bounded since $\theta \in \mathbb{S}$, so 
%
\begin{align*}
x_1(t)^2 \leq 1 \ \rightarrow \ ( x_1(t)^2 + x_2(t)^2 ) < \epsilon^2 + 5 \ \rightarrow \ || x(t) ||_2 < \sqrt{\epsilon^2 + 5} = \delta(\epsilon)
\end{align*}
%
This shows that the dynamics around the equilibrium is \textit{stable in the sense of Lyapunov} (globally). Note that 
since $E(t) = E(0) \, \forall t >0$, $|| x(t) ||_2 \neq 0 \forall t >0$, thus the system around the equilibrium is not asymptotically
stable (local or global), and hence not exponentially stable. 

\subsection{Internal Stability of LTI Systems}

Lyapunov's stability definitions may not be very useful for analyzing
the stability of non-linear systems, however we can easily derive
necessary and sufficient conditions for stability. One should also note the 
in LTI systems (and linear systems in general) we are interested in the
stability of the origin $x_e = 0$. Autonomous CT and DT LTI systems are expressed 
by the following matrix differential and difference equations
%
\begin{align*}
\dot{x} &= A x 
\\
x[k+1] =& A x[k]
\end{align*}
%
and we know the analytical solutions to the zero-input responses have the following forms
%
\begin{align*}
x(t) = e^{A t} x_0
\\
x[k] = A^k x_0
\end{align*}
%
It is easier to analyze the internal stability using Jordan decomposition of the system matrix $A$, 
%
\begin{align*}
x(t) = G e^{J t} G^{-1} x_0 \ \rightarrow \  \left[ G^{-1} x(t) \right] = e^{J t} \left[ G^{-1} x_0 \right] \ \rightarrow \ \alpha(t) = e^{J t} \alpha_0
\\
x[k] = G J^k G^{-1} x_0  \ \rightarrow \ \alpha[k] = J^k \alpha_0
\end{align*}
%
Note that since $G$ and $G^{-1}$ finite and invertible matrices, we know that 
%
\begin{align*}
|| x || = 0 &\iff || G^{-1} x || = 0 \iff x = 0 
\\
|| x || < M_1 < \infty &\iff || G^{-1} x || < M_2 < \infty \ \mathrm{where} M_1 , M_2 \in \mathbb{R}
\\
|| x || \to \infty &\iff || G^{-1} x || \to \infty 
\end{align*}
%
We know that $e^{J t}$ and $J^k$ has the following block diagonal form
%
\begin{align*}
e^{J t} &= \left[ \begin{array}{ccccc} e^{J_1 t} &  & & &  \\  & e^{J_2 t}  &  & 0 &  \\ &  & \ddots & \\ & 0 & & \ddots & \\ & &  & &  e^{J_n t}  \end{array} \right]
\\
J^k &= \left[ \begin{array}{ccccc} J_1^k &  & & &  \\  & J_2^k  &  & 0 &  \\ &  & \ddots & \\ & 0 & & \ddots & \\ & &  & &  J_n^k  \end{array} \right]
\end{align*}
%
where an individual block associated with a Jordan block of $A$ has the following form
%
\begin{align*}
e^{J_i t} &= \left[  \begin{array}{cccccc} e^{\lambda_i t} & t e^{\lambda t} & \frac{t^2}{2 !} e^{\lambda_i t} 
& \cdots & \frac{t^{n-2}}{(n-2) !} e^{\lambda_i t}  & \frac{t^{n-1}}{(n-1) !} e^{\lambda_i t}
\\ 0 & e^{\lambda_i t} & t e^{\lambda_i t} & \frac{t^2}{2 !} e^{\lambda_i t} & \cdots  & \frac{t^{n-2}}{(n-2) !} e^{\lambda_i t}
\\  \vdots &  & \ddots &  &  & \vdots \\ 
& & & e^{\lambda_i t} & t e^{\lambda_i t} & \frac{t^2}{2 !} e^{\lambda_i t}
\\ 0 &  & \cdots  &  & e^{\lambda_i t} & t e^{\lambda_i t} \\
0 &  & \cdots &  & 0 & e^{\lambda_i t} \end{array} \right] 
\\
J_i^k &= \left[  \begin{array}{cccccc} \lambda_i^k & \begin{pmatrix} k \\ 1 \end{pmatrix} \lambda_i^{k-1} & \begin{pmatrix} k \\ 2 \end{pmatrix}  \lambda_i^{k-2} 
& \cdots & \begin{pmatrix} k \\ n\mathrm{-}2 \end{pmatrix} \lambda_i^{k-n+2} & \begin{pmatrix} k \\ n\mathrm{-}1 \end{pmatrix} \lambda_i^{k-n+1}
\\ 0 & \lambda_i^k & \begin{pmatrix} k \\ 1 \end{pmatrix}  \lambda_i^{k-1} & \begin{pmatrix} k \\ 2 \end{pmatrix} \lambda_i^{k-2} & \cdots  & \begin{pmatrix} k \\ n\mathrm{-}2 \end{pmatrix} \lambda_i^{k-n+2}
\\ 
\\  \vdots &  & \ddots &  &  & \vdots \\ 
\\ 
& & & \lambda_i^k & \begin{pmatrix} k \\ 1 \end{pmatrix}  \lambda_i^{k-1} & \begin{pmatrix} k \\ 2 \end{pmatrix}  \lambda_i^{k-2} 
\\ 0 &  & \cdots  &  & \lambda_i^k & \begin{pmatrix} k \\ 1 \end{pmatrix}  \lambda_i^{k-1} \\
0 &  & \cdots &  & 0 & \lambda_i^k \end{array} \right] 
\end{align*}
%
Based on this decomposition we can derive the stability conditions
%
\begin{enumerate}
\item 
\begin{itemize}
\item A CT LTI system is \textit{asympotitically \& exponentially stable} 

$\iff \ \lim_{t \to \infty } e^{J_i t} = 0 \ \forall i \ \iff \ \mathrm{Re}\lbrace \lambda_i \rbrace < 0 \ \forall i$  
\item A DT LTI system is \textit{asympotitically \& exponentially stable} 

$\iff \ \lim_{k \to \infty } J_i^k = 0 \ \forall i \ \iff \ | \lambda_i | < 1 \ \forall i$  
\end{itemize}
%
\item 
\begin{itemize}
\item A CT LTI system is \textit{s.i.s.L} 

$\iff \ \lim_{t \to \infty } e^{J_i t} = M_i < \infty \ \forall i \ \iff $ 
$ \lbrace \mathrm{Re}\lbrace \lambda_i \rbrace < 0 \ \mathrm{or} \ \left\lbrace \mathrm{Re}\lbrace \lambda_i \rbrace = 0 \ \mathrm{and} \ J_i \in \mathbb{R} \rbrace \right\rbrace \ \forall i$  
%
\item A DT LTI system is \textit{s.i.s.L} 

$\iff \ \lim_{k \to \infty } J_i^k = M_i < \infty \ \forall i \ \iff $ 
$ \lbrace | \lambda_i | < 1 \ \mathrm{or} \ \left\lbrace | \lambda_i | = 1 \ \mathrm{and} \ J_i \in \mathbb{R} \rbrace \right\rbrace \ \forall i$  
\end{itemize}

\item 
\begin{itemize}
\item A CT LTI system is \textit{unstable}

$\iff \ \exists i \ \ \lim_{t \to \infty } e^{J_i t} = \infty \ \iff $ 
$ \lbrace \mathrm{Re}\lbrace \lambda_i \rbrace > 0 \ \mathrm{or} \ \left\lbrace \mathrm{Re}\lbrace \lambda_i \rbrace = 0 \ \mathrm{and} \ J_i \in \mathbb{R}^{n \times n} , n>1 \rbrace \right\rbrace \ \forall i$  
%
\item A DT LTI system is \textit{unstable} 

$\iff \ \exists i \ \lim_{k \to \infty } J_i^k = \infty \ \iff $ 
$ \exists i \lbrace | \lambda_i | > 1 \ \mathrm{or} \ \left\lbrace | \lambda_i | = 1 \ \mathrm{and} \ J_i \in \mathbb{R}^{n \times n} , n>1 \rbrace \right\rbrace $  
\end{itemize}

\end{enumerate}



% **** This ENDS THE EXAMPLES. DON'T DELETE THE FOLLOWING LINE:
\end{document}
