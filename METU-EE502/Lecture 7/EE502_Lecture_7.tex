

\documentclass[twoside]{article}
\setlength{\oddsidemargin}{0.25 in}
\setlength{\evensidemargin}{-0.25 in}
\setlength{\topmargin}{-0.6 in}
\setlength{\textwidth}{6.5 in}
\setlength{\textheight}{8.5 in}
\setlength{\headsep}{0.75 in}
\setlength{\parindent}{0 in}
\setlength{\parskip}{0.1 in}

%
% ADD PACKAGES here:
%

\usepackage{amsmath,amsfonts,graphicx}

%
\newcounter{lecnum}
\renewcommand{\thepage}{\thelecnum-\arabic{page}}
\renewcommand{\thesection}{\thelecnum.\arabic{section}}
\renewcommand{\theequation}{\thelecnum.\arabic{equation}}
\renewcommand{\thefigure}{\thelecnum.\arabic{figure}}
\renewcommand{\thetable}{\thelecnum.\arabic{table}}

%
% The following macro is used to generate the header.
%
\newcommand{\lecture}[4]{
   \pagestyle{myheadings}
   \thispagestyle{plain}
   \newpage
   \setcounter{lecnum}{#1}
   \setcounter{page}{1}
   \noindent
   \begin{center}
   \framebox{
      \vbox{\vspace{2mm}
    \hbox to 6.28in { {\bf EE502 - Linear Systems Theory II
	\hfill Spring 2032} }
       \vspace{4mm}
       \hbox to 6.28in { {\Large \hfill Lecture #1 \hfill} }
       \vspace{2mm}
       \hbox to 6.28in { {\it Lecturer: #2 \hfill } }
      \vspace{2mm}}
   }
   \end{center}
   \markboth{Lecture #1}{Lecture #1}

   \vspace*{4mm}
}


\renewcommand{\cite}[1]{[#1]}
\def\beginrefs{\begin{list}%
        {[\arabic{equation}]}{\usecounter{equation}
         \setlength{\leftmargin}{2.0truecm}\setlength{\labelsep}{0.4truecm}%
         \setlength{\labelwidth}{1.6truecm}}}
\def\endrefs{\end{list}}
\def\bibentry#1{\item[\hbox{[#1]}]}

%Use this command for a figure; it puts a figure in wherever you want it.
%usage: \fig{NUMBER}{SPACE-IN-INCHES}{CAPTION}
\newcommand{\fig}[3]{
			\vspace{#2}
			\begin{center}
			Figure \thelecnum.#1:~#3
			\end{center}
	}
% Use these for theorems, lemmas, proofs, etc.
\newtheorem{theorem}{Theorem}[lecnum]
\newtheorem{lemma}[theorem]{Lemma}
\newtheorem{proposition}[theorem]{Proposition}
\newtheorem{claim}[theorem]{Claim}
\newtheorem{corollary}[theorem]{Corollary}
\newtheorem{definition}[theorem]{Definition}
\newenvironment{proof}{{\bf Proof:}}{\hfill\rule{2mm}{2mm}}

% **** IF YOU WANT TO DEFINE ADDITIONAL MACROS FOR YOURSELF, PUT THEM HERE:

\begin{document}

% Lecture Details
\lecture{7}{Assoc. Prof. M. Mert Ankarali}



\section{Discrete-Time Linear Time Varying State Space Models} 

State-space representation of a (causal \& finite dimensional) LTV DT system is given by
%
\begin{align*}
  \mathrm{Let} \ x[k] &\in \mathbb{R}^n \ , \ y[k] \in \mathbb{R}^m \ ,\  u[k] \in
  \mathbb{R}^r , \\
  x[k+1] &= A[k] x[k] + B[k] u[k] , \\
  y[k] &= C[l] x[k] + D u[k] , \\
  \mathrm{where} \ G[k] &\in \mathbb{R}^{n \times n} \ , \ 
    B[k] \in \mathbb{R}^{n \times r} \ ,\  C[k] \in \mathbb{R}^{m \times n} \ , \ D[d] \in \mathbb{R}^{m \times r}
\end{align*}
%
Let's first assume that $u[k] = 0$, and find un-driven 
response.
%
\begin{align*}
  x[k+1] &= A[k] x[k] \\ y[k] &= C[k] x[k]
\end{align*}
%
Unlike LTV-CT systems we easily can compute the response iteratively
%
\begin{align*}
   x[0] &= I x[0] \quad , \quad y[0] = C[0] x[0]
   \\
  x[1] &= A[0] x[0] \quad , \quad y[1] = C[1] x[1]
  \\
  x[2] &= A[0] x[1] = A[1] A[0] x[0] \quad , \quad y[2] = C[2] x[2]
  \\
  x[3] &= A[2] x[2] = A[3] A[1] A[0] x[0] \quad , \quad y[3] = C[3] x[3]
  \\
  \vdots
  \\
  x[k] &= A[k-1] x[k-1] = A[k-1] A[k-2] \cdots A[1] A[0] x[0]  \quad , \quad y[k] = \quad , \quad y[k] = C[k] x[k]
  \\
  x[k] &= \prod_{i=0}^{k-1} A[k-1-i] 
\end{align*}
%
Motivated by the LTI case, we define the \textbf{state transition matrix}, which relates the state of the
undriven system at time $k$ to the state at an earlier time $m$
%
\begin{align*}
  x[k] &= \Phi[k,m] x[m] \ , \  k \geq m \ , \  \mathrm{where}
  \\
  \Phi[k,m] &= \left\lbrace \begin{array}{l} \prod_{i=0}^{k-1} A[k-1-i] \ , \ k > m 
  \\ I \quad \quad \quad \quad  \quad \quad  \quad \ \ , \ k = m 
  \end{array} \right.
\end{align*}
%
Note that state-transition matrix satisfies following important properties
undriven system at time $k$ to the state at an earlier time $m$
%
\begin{align*}
  \Phi[k,k] &= I
  \\
  x[k] &= \Phi[k,0] x[0]
  \\
  \Phi[k+1,m] = A[k] \Phi[k,m] 
\end{align*}
%
as you can see, the state-transition matrix satisfies the discrete dynamical state equations. 
%


Now let's consider input-only state response (i.e. $x[0] = 0$).
%
\begin{align*}
  x[k+1] &= G x[k] + H u[k] 
  \\
  \\
  x[1] &= H u[0]
  \\
  x[2] &= G x[1] + H u[1] = G H u[0] + H u[1] 
  \\
  x[3] &= G x[2] + H u[2] = G^2 H u[0] + G H u[1] + H u[2]
  \\
  x[4] &= G x[3] + H u[3] = G^3 H u[0] + G^2 H u[1] + G H u[2] + H
         u[3]
  \\
  \vdots
 \\
  x[k] &= G x[k-1] + H u[k-1] \\
         &= G^{k-1} H u[0] + G^{k-2} H u[1] +
         \cdots + G H u[k-2] + H u[k-1]
         \\
         &= \left[ \begin{array}{c|c|c|c|c} G^{k-1} H & G^{k-2} H &
         \cdots & G H & H \end{array} \right]
         \left[ \begin{array}{c}
                  u[0] \\ u[1] \\ \vdots \\ u[k-2] \\ u[k-1]
         \end{array} \right]
         \\
         &= \sum\limits_{j = 0}^{k-1} G^{k-j-1} H u[j]
         \\
         &= \sum\limits_{j = 0}^{k-1} G^{j} H u[k-j-1]
\end{align*}
%
Given that $\Psi[k] = G^k$
%
\begin{align*}
    x[k] &= \sum\limits_{j = 0}^{k-1} \Psi[k-j-1] H u[j]
\\
&= \sum\limits_{j = 0}^{k-1} \Psi[j] H u[k-j-1]
\end{align*}
%
If we combine homegeneous and driven responses we can simply obtain
%
\begin{align*}
    x[k] &= \Psi[k] x[0] + \sum\limits_{j = 0}^{k-1} \Psi[k-j-1] H u[j]
\\
&= \Psi[k] x[0] + \sum\limits_{j = 0}^{k-1} \Psi[j] H u[k-j-1]
\end{align*}
% 
whereas output at time $k$ has the form
%
\begin{align*}
    y[k] &= C \Psi[k] x[0] + C \left( \sum\limits_{j = 0}^{k-1}
           \Psi[k-j-1] H u[j] \right) + D u[k]
\\
&= C \Psi[k] x[0] + C \left( \sum\limits_{j = 0}^{k-1} \Psi[j] H
  u[k-j-1] \right) +  D u[k]
\end{align*}
% 

\subsection*{Z-domain Solution of State-Space Equations}

We already computed the transfer function under zero initial
conditions.
%
\begin{align*}
  Y(z) =  \left[ C \left(z I - G \right)^{-1} H + D \right] U(z)
\end{align*}
%

%
\begin{align*}
z \left( z I - G \right)^{-1} &= I + z^{-1} G + z^{-2} G^2 + z^{-3} G^3
  + \cdots
\\
\mathcal{Z}^{-1} \left[ z \left( z I - G \right)^{-1} \right] &=
I \, \delta[k] + G \, \delta[k-1] + G^2 \, \delta[k-2] + G^3 \,
  \delta[k-3] + \cdots 
\end{align*}
%

\newpage

\textbf{Example:} Consider the following state-space representation
%
\begin{align*}
  x[k+1] &= \left[ \begin{array}{ccc} 1 & 0 & 0\\ 0 & 1/2 & 0
    \\ 0 & 0 & -1 \end{array} \right] x[k] 
    + \left[ \begin{array}{c} 1 \\ 1 \\ 1\end{array} \right] u[k]
\\
 y[k] &= \left[ \begin{array}{ccc} 1 & 2 & 3 \end{array} \right] x[k] 
\end{align*}
%
\begin{itemize}
  \item Compute the closed form expression $\Psi[k]$ using the time
   expression

    \textbf{Solution:} The state-space representation is in Diagonal
    canonical form
    
\begin{align*}
  \Psi[k]  = G^k = \left[ \begin{array}{ccc} 1 & 0 & 0\\ 0 & 1/2 & 0
    \\ 0 & 0 & -1 \end{array} \right]^k 
    = \left[ \begin{array}{ccc} 1^k & 0 & 0\\ 0 & (1/2)^k & 0
    \\ 0 & 0 & (-1)^k \end{array} \right]
 = \left[ \begin{array}{ccc} 1 & 0 & 0\\ 0 & (1/2)^k & 0
    \\ 0 & 0 & (-1)^k \end{array} \right]
\end{align*}    

\item Compute the closed form expression $\Psi[k]$ using the z-domain 
  solution method

\textbf{Solution:}

\begin{align*}
  \Psi[k]  &= \mathcal{Z}^{-1} \left[   z \left( z I - G \right)^{-1} \right]
\\
&= \mathcal{Z}^{-1} \left[   z \left( \left[ \begin{array}{ccc} z-1 & 0 & 0\\ 0 & (z-1/2) & 0
    \\ 0 & 0 & z+1 \end{array} \right] \right)^{-1} \right]
\\
&= \mathcal{Z}^{-1} \left[   \left[ \begin{array}{ccc} \frac{z}{z-1} & 0 & 0\\ 0 & \frac{z}{z-1/2} & 0
    \\ 0 & 0 & \frac{z}{z+1} \end{array} \right] \right]
\\
&= \left[ \begin{array}{ccc} 1 & 0 & 0\\ 0 & (1/2)^k & 0
    \\ 0 & 0 & (-1)^k \end{array} \right] \quad \mathrm{for} k \geq 0
\end{align*}    

\item Compute the impulse response of the system from the time domain
solution

\textbf{Solution:}

\begin{align*}
  x[k] &= G^{k-1} H u[0] \quad \mathrm{for} \  k > 0 \\
  y[k] &= C G^{k-1} H \quad \mathrm{for} \ k > 0 \\ 
         &= \left[ \begin{array}{ccc} 1 & 2 & 3 \end{array} \right] 
                \left[ \begin{array}{ccc} 1 & 0 & 0\\ 0 & (1/2)^{k-1} & 0
    \\ 0 & 0 & (-1)^{k-1} \end{array} \right]
      \left[ \begin{array}{c} 1 \\ 1 \\ 1\end{array} \right] 
\\
&= \left[ \begin{array}{ccc} 1 & 2 & 3 \end{array} \right] 
\left[ \begin{array}{c} 1 \\ (1/2)^{k-1} \\ (-1)^{k-1} \end{array}
  \right] 
\\
y[k] &= 1 + 2 (1/2)^{k-1} + 3 (-1)^{k-1} \quad \mathrm{for} \ k > 0 
\end{align*}    

\item Compute the transfer function $\frac{Y(z)}{U(z)}$

\textbf{Solution:}

\begin{align*}
  T(z) &= C \left( z I - G \right)^{-1} H \\
         &= \left[ \begin{array}{ccc} 1 & 2 & 3 \end{array} \right] 
\left[ \begin{array}{ccc} z-1 & 0 & 0\\ 0 & z-1/2 & 0
    \\ 0 & 0 & z+1 \end{array} \right]^{-1}           
\left[ \begin{array}{c} 1 \\ 1 \\ 1\end{array} \right]
\\
&= \left[ \begin{array}{ccc} 1 & 2 & 3 \end{array} \right] 
\left[ \begin{array}{ccc} \frac{1}{z-1} & 0 & 0\\ 0 & \frac{1}{z-1/2} & 0
    \\ 0 & 0 & \frac{1}{z+1} \end{array} \right]           
\left[ \begin{array}{c} 1 \\ 1 \\ 1\end{array} \right]                                   
\\
&= 
\left[ \begin{array}{ccc} 1 & 2 & 3 \end{array} \right] 
\left[ \begin{array}{c} \frac{1}{z-1} \\ \frac{1}{z-1/2} \\
         \frac{1}{z+1}\end{array} \right]         
\\
  T(z) &= \frac{1}{z-1} + \frac{2}{z-1/2} + \frac{3}{z+1}
\end{align*}    

\item Compute the inverse Z-transform of the  transfer function 

\textbf{Solution:}

\begin{align*}
  t[k] &= \mathcal{Z}^{-1} \left[ \frac{1}{z-1} + \frac{2}{z-1/2} +
  \frac{3}{z+1} \right]
\\
&= \left( 1 + 2 (1/2)^{k-1} + 3 (-1)^{k-1} \right) h[k-1]
\end{align*}    
%
where $h[k]$ is the unit step function


\end{itemize}

% **** This ENDS THE EXAMPLES. DON'T DELETE THE FOLLOWING LINE:
\end{document}
