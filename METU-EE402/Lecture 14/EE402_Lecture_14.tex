%


\documentclass[twoside]{article}
\setlength{\oddsidemargin}{0.25 in}
\setlength{\evensidemargin}{-0.25 in}
\setlength{\topmargin}{-0.6 in}
\setlength{\textwidth}{6.5 in}
\setlength{\textheight}{8.5 in}
\setlength{\headsep}{0.75 in}
\setlength{\parindent}{0 in}
\setlength{\parskip}{0.1 in}

%
% ADD PACKAGES here:
%

\usepackage{amsmath,amsfonts,graphicx}


\newcounter{lecnum}
\renewcommand{\thepage}{\thelecnum-\arabic{page}}
\renewcommand{\thesection}{\thelecnum.\arabic{section}}
\renewcommand{\theequation}{\thelecnum.\arabic{equation}}
\renewcommand{\thefigure}{\thelecnum.\arabic{figure}}
\renewcommand{\thetable}{\thelecnum.\arabic{table}}

%
% The following macro is used to generate the header.
%
\newcommand{\lecture}[4]{
   \pagestyle{myheadings}
   \thispagestyle{plain}
   \newpage
   \setcounter{lecnum}{#1}
   \setcounter{page}{1}
   \noindent
   \begin{center}
   \framebox{
      \vbox{\vspace{2mm}
    \hbox to 6.28in { {\bf EE402 - Discrete Time Systems
	\hfill Spring 2018} }
       \vspace{4mm}
       \hbox to 6.28in { {\Large \hfill Lecture #1 \hfill} }
       \vspace{2mm}
       \hbox to 6.28in { {\it Lecturer: #2 \hfill } }
      \vspace{2mm}}
   }
   \end{center}
   \markboth{Lecture #1}{Lecture #1}

   \vspace*{4mm}
}

\renewcommand{\cite}[1]{[#1]}
\def\beginrefs{\begin{list}%
        {[\arabic{equation}]}{\usecounter{equation}
         \setlength{\leftmargin}{2.0truecm}\setlength{\labelsep}{0.4truecm}%
         \setlength{\labelwidth}{1.6truecm}}}
\def\endrefs{\end{list}}
\def\bibentry#1{\item[\hbox{[#1]}]}


\newcommand{\fig}[3]{
			\vspace{#2}
			\begin{center}
			Figure \thelecnum.#1:~#3
			\end{center}
	}

% Use these for theorems, lemmas, proofs, etc.
\newtheorem{theorem}{Theorem}[lecnum]
\newtheorem{lemma}[theorem]{Lemma}
\newtheorem{proposition}[theorem]{Proposition}
\newtheorem{claim}[theorem]{Claim}
\newtheorem{corollary}[theorem]{Corollary}
\newtheorem{definition}[theorem]{Definition}
\newenvironment{proof}{{\bf Proof:}}{\hfill\rule{2mm}{2mm}}

% **** IF YOU WANT TO DEFINE ADDITIONAL MACROS FOR YOURSELF, PUT THEM HERE:

\begin{document}

% Lecture Details
\lecture{14}{Asst. Prof. M. Mert Ankarali}

\par

\section*{Asymptotic Stability of LTI Systems}

\subsection*{Asymptotic Stability of CT-LTI Systems}

Let's consider the state-representation of an CT LTI system 
%
\begin{align*}
  \dot{x}(t) &= A x(t) + B u(t)
\\
  y(t) &= C x(t) + D u(t)
\end{align*}
%

Given CT-LTI system is called \textit{asymptotically stable} if,
with $u(t) = 0$ and $\forall x(0) \in \mathbb{R}^n$, we have
%
\begin{align*}
  \lim_{t \to \infty} || x(t) || = 0
\end{align*}

\textbf{Example:} Let's assume that system matrix $A$ is
diagonalizable and all eigenvalues are real. Find a
necessary and sufficient condition such that the state-space
representation is asymptotically stable. 

\textbf{Solution:} When $u(t) = 0$, we have
%
\begin{align*}
\dot{x}(t) &= A x(t)
\\
x(t) &= e^{A t} x_0
\end{align*}
%
Since $A$ is diagonalizable, we know that
%
\begin{align*}
  A &= P \left[ \begin{array}{cccc} \lambda_1 & 0 & \cdots & 0 
\\
0 & \lambda_2 &  \cdots & 0 
\\
\vdots &  &  \ddots &  
\\
0 &  &  &  \lambda_n
\end{array} \right] P^{-1}
\\
e^{A t} &= P \left[ \begin{array}{cccc} e^{\lambda_1 t} & 0 & \cdots & 0 
\\
0 & e^{\lambda_2 t} &  \cdots & 0 
\\
\vdots &  &  \ddots &  
\\
0 &  &  &  e^{\lambda_n t}
\end{array} \right] P^{-1}
\end{align*}

Let $P x_0 = [e_1 \ e_2 \ \cdots \ e_n ]^T, e_i \in \mathbb{R}$, then 
$x(t)$ can be expressed as

\begin{align*}
x(t) &= P \left[ \begin{array}{cccc} e^{\lambda_1 t} & 0 & \cdots & 0 
\\
0 & e^{\lambda_2 t} &  \cdots & 0 
\\
\vdots &  &  \ddots &  
\\
0 &  &  &  e^{\lambda_n t}
\end{array} \right] 
\left[ \begin{array}{c} e_1 \\ e_2 \\ \vdots \\ e_n \end{array}
  \right] 
\\
x(t) &= P \left(
\left[ \begin{array}{c} e_1 e^{\lambda_1 t} \\ 0 \\ \vdots \\
         0 \end{array} \right]
+
\left[ \begin{array}{c} 0 \\ e_2 e^{\lambda_2 t} \\ \vdots \\
         0 \end{array} \right]
+ \cdots
\left[ \begin{array}{c} 0 \\ 0 \\ \vdots \\
         e_n e^{\lambda_n t}  \end{array} \right]
\right)
\end{align*}

Then it is easy to see that
\begin{align*}
    \forall x_0 \in \mathbb{R}^n, \lim_{t \to \infty} || x(t) || = 0
    \iff \forall i \in \lbrace 1 , \ \cdots ,  \ n \rbrace, \lambda_i < 0
\end{align*}

\subsection*{Asymptotic Stability of DT-LTI Systems}

Now, let's consider the state-representation of an DT LTI system 
%
\begin{align*}
  x[k+1] &= G x[k] + H u[k]
\\
  y[k] &= C x[k] + D u[k]
\end{align*}
%

Given DT-LTI system is called \textit{asymptotically stable} if,
with $u[k] = 0]$ and $\forall x(0) \in \mathbb{R}^n$, we have
%
\begin{align*}
  \lim_{k \to \infty} || x[k] || = 0
\end{align*}

\textbf{Example:} Let's assume that system matrix $G$ is
diagonalizable
and all eigenvalues are real. Find a
necessary and sufficient condition such that the state-space
representation is asymptotically stable. 

\textbf{Solution:} When $u[k] = 0$, we have
%
\begin{align*}
x[k+1] &= G x[k]
\\
x(t) &= G^k x_0
\end{align*}
%
Since $G$ is diagonalizable, we know that
%
\begin{align*}
  G &= P \left[ \begin{array}{cccc} \lambda_1 & 0 & \cdots & 0 
\\
0 & \lambda_2 &  \cdots & 0 
\\
\vdots &  &  \ddots &  
\\
0 &  &  &  \lambda_n
\end{array} \right] P^{-1}
\\
G^k &= P \left[ \begin{array}{cccc} \lambda_1^k & 0 & \cdots & 0 
\\
0 & \lambda_2^l &  \cdots & 0 
\\
\vdots &  &  \ddots &  
\\
0 &  &  &  \lambda_n^k
\end{array} \right] P^{-1}
\end{align*}

Let $P x_0 = [e_1 \ e_2 \ \cdots \ e_n ]^T, e_i \in \mathbb{R}$, then 
$x[k]$ can be expressed as

\begin{align*}
x[k] &= P \left(
\left[ \begin{array}{c} e_1 \lambda_1^k\\ 0 \\ \vdots \\
         0 \end{array} \right]
+
\left[ \begin{array}{c} 0 \\ e_2 \lambda_2^k \\ \vdots \\
         0 \end{array} \right]
+ \cdots
\left[ \begin{array}{c} 0 \\ 0 \\ \vdots \\
         e_n \lambda_n^k \end{array} \right]
\right)
\end{align*}

Then it is easy to see that
\begin{align*}
    \forall x_0 \in \mathbb{R}^n, \lim_{k \to \infty} || x[k] || = 0
    \iff \forall i \in \lbrace 1 , \ \cdots ,  \ n \rbrace, |
  \lambda_i | < 1
\end{align*}

\section*{BIBO Stability of LTI Systems}

A CT-system written in state-space form 
is stable if and only if the poles of $H(s) = C \left[ s I -
  A\right]^{-1} B + D$ are located strictly in the open left
half plane. 

A DT-system written in state-space form 
is stable if and only if the poles of $H(z) = C \left[ z I -
  G\right]^{-1} H + D$ are located strictly inside the
unit circle.

We did not mention weather the system is SISO or MIMO. 
%
\begin{itemize}
  \item Do you think that BIBO notion is valid for MIMO systems? 
  \item Do you think that checking ``poles'' of $H(s)$ or $H(z)$ is
    a valid method for checking stability?
    \item What are the poles of $H(s)$ and $H(z)$?
\end{itemize}

\textbf{Example:} Consider the following state-space form of a
CT system 
%
\begin{align*}
\dot{x}(t) &= 
\left[ \begin{array}{cc} 0 & 1 \\ 1 & 0 \end{array} \right]
x(t)
+
\left[ \begin{array}{c} 0 \\ 1 \end{array} \right] u(t)
\\
y(t) &= \left[ \begin{array}{cc} 1 & -1 \end{array} \right] x(t)
\end{align*}

Is this system asymptotically stable? 

\textbf{Solution:} Let's compute the eigenvalues of $A$

\begin{align*}
\mathrm{det} \left( \left[ \begin{array}{cc} \lambda & -1 \\ -1 &
                                                                  \lambda \end{array}
                                                                  \right]
                                                                  \right)
                                                                  =
                                                                  \lambda^2
                                                                  - 1
\\
\lambda_{1,2} = \pm 1
\end{align*}

Thus the system is NOT Asymptotically Stable.

Is this system BIBO stable? 

\textbf{Solution:} Let's compute the $H(s)$

\begin{align*}
  H(s) &= \left[ \begin{array}{cc} 1 & -1 \end{array} \right]
\left[ \begin{array}{cc} s & -1 \\ -1 & s\end{array}
                                                                  \right]^{-1}
\left[ \begin{array}{c} 0 \\ 1 \end{array} \right]
\\
&=
\left[ \begin{array}{cc} 1 & -1 \end{array} \right]
\left[ \begin{array}{cc} s & 1 \\ 1 & s\end{array}
                                                                  \right]
\left[ \begin{array}{c} 0 \\ 1 \end{array} \right] \frac{1}{s^2 - 1}
\\
&=
\left[ \begin{array}{cc} 1 & -1 \end{array} \right]
\left[ \begin{array}{cc} s & 1 \\ 1 & s\end{array}
                                                                  \right]
\left[ \begin{array}{c} 0 \\ 1 \end{array} \right] \frac{1}{s^2 - 1}
\\
&=
\left[ \begin{array}{cc} 1 & -1 \end{array} \right]
\left[ \begin{array}{c} 1 \\ s \end{array} \right] \frac{1}{s^2 - 1}
\\
&= \frac{-(s-1)}{s^2 - 1} 
\\
&= \frac{-1}{s + 1} 
\end{align*}

Indeed, the system is BIBO Stable.

In conclusion

\begin{itemize}
  \item Asymptotic Stability $\rightarrow$ BIBO Stability 
  \item BIBO Stability $\not\rightarrow$ Asymptotic Stability
\end{itemize} 

% **** This ENDS THE EXAMPLES. DON'T DELETE THE FOLLOWING LINE:
\end{document}