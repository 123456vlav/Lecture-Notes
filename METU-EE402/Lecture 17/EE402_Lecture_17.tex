% 


\documentclass[twoside]{article}
\setlength{\oddsidemargin}{0.25 in}
\setlength{\evensidemargin}{-0.25 in}
\setlength{\topmargin}{-0.6 in}
\setlength{\textwidth}{6.5 in}
\setlength{\textheight}{8.5 in}
\setlength{\headsep}{0.75 in}
\setlength{\parindent}{0 in}
\setlength{\parskip}{0.1 in}

%
% ADD PACKAGES here:
%

\usepackage{amsmath,amsfonts,graphicx,mathdots}


\newcounter{lecnum}
\renewcommand{\thepage}{\thelecnum-\arabic{page}}
\renewcommand{\thesection}{\thelecnum.\arabic{section}}
\renewcommand{\theequation}{\thelecnum.\arabic{equation}}
\renewcommand{\thefigure}{\thelecnum.\arabic{figure}}
\renewcommand{\thetable}{\thelecnum.\arabic{table}}

%
% The following macro is used to generate the header.
%
\newcommand{\lecture}[4]{
   \pagestyle{myheadings}
   \thispagestyle{plain}
   \newpage
   \setcounter{lecnum}{#1}
   \setcounter{page}{1}
   \noindent
   \begin{center}
   \framebox{
      \vbox{\vspace{2mm}
    \hbox to 6.28in { {\bf EE402 - Discrete Time Systems
	\hfill Spring 2018} }
       \vspace{4mm}
       \hbox to 6.28in { {\Large \hfill Lecture #1 \hfill} }
       \vspace{2mm}
       \hbox to 6.28in { {\it Lecturer: #2 \hfill } }
      \vspace{2mm}}
   }
   \end{center}
   \markboth{Lecture #1}{Lecture #1}

   \vspace*{4mm}
}

\renewcommand{\cite}[1]{[#1]}
\def\beginrefs{\begin{list}%
        {[\arabic{equation}]}{\usecounter{equation}
         \setlength{\leftmargin}{2.0truecm}\setlength{\labelsep}{0.4truecm}%
         \setlength{\labelwidth}{1.6truecm}}}
\def\endrefs{\end{list}}
\def\bibentry#1{\item[\hbox{[#1]}]}


\newcommand{\fig}[3]{
			\vspace{#2}
			\begin{center}
			Figure \thelecnum.#1:~#3
			\end{center}
	}

% Use these for theorems, lemmas, proofs, etc.
\newtheorem{theorem}{Theorem}[lecnum]
\newtheorem{lemma}[theorem]{Lemma}
\newtheorem{proposition}[theorem]{Proposition}
\newtheorem{claim}[theorem]{Claim}
\newtheorem{corollary}[theorem]{Corollary}
\newtheorem{definition}[theorem]{Definition}
\newenvironment{proof}{{\bf Proof:}}{\hfill\rule{2mm}{2mm}}

% **** IF YOU WANT TO DEFINE ADDITIONAL MACROS FOR YOURSELF, PUT THEM HERE:

\begin{document}

% Lecture Details
\lecture{17}{Asst. Prof. M. Mert Ankarali}

\par

\section*{Discrete-time Luenberger Observer}

In general the state, $x[k]$, of a system
is not accessible and \textit{observers, estimators, filters})
have to be used to extract this information.
The output, $y[k]$, represents the measurements
which is a function of $x[k]$ and $u[k]$.
%
\begin{align*}
  x[k+1] &= G x[k] + H u[k]
  \\
  y[k] &= C x[k] + D u[k]
\end{align*}
%
A Luenberger observers is built using a ``simulated'' model of the 
system and the errors caused by the mismatched initial conditions 
$x_0 \neq \hat{x}_0$ (or other types of perturbations)
are reduced by introducing output error feedback.

Let's assume that the state vector of the simulated system
is$\hat{x}[k]$, then the state space equation of this
synthetic system takes the form
%
\begin{align*}
  \hat{x}[k+1] &= G \hat{x}[k] + H u[k]
  \\
  \hat{y}[k] &= C \hat{x}[k] + D u[k]
\end{align*}
%
Note that since $u[k]$ is the input that is supplied by the 
controller, we assume that it is known apriori. If $x[0] = \hat{x}[0]$ and
when there is no model mismatch or uncertainty in the system
then we expect that $x[k] = \hat{x}[k]$ and $y[k] = \hat{y}[k]$ 
for all $k$. When $x[0] \neq \hat{x}[0]$, then we should observe a 
difference between the measured and predicted output
$y[k] \neq \hat{y}[k]$. The core idea in Luenberger observer
is feeding the error in the output prediction 
$y[k] - \hat{y}[k]$ to the simulated system via a linear feedback gain.
%
\begin{align*}
  \hat{x}[k+1] &= G \hat{x}[k] + H u[k] + L \left( y[k] - \hat{y}[k] \right) 
  \\
  \hat{y}[k] &= C \hat{x}[k] + D u[k]
\end{align*}
%
In order to understand how a Luenberger observer works and
to choose a proper observer gain $L$, we define an error signal
$e[k] = x[k] - \hat{x}[k]$. The dynamics w.r.t $e[k]$ can be derived
as
%
\begin{align*}
  e[k+1] &= x[k+1] - \hat{x}[k+1]
        \\
     &= \left( G x[k] + H u[k] \right)
  - \left( G \hat{x}[k] + H u[k] + L \left( y[k] - \hat{y}[k] \right)
       \right)
\\
   e[k+1] &= \left( G - L C \right) e[k]
\end{align*}
%
where $e[0] = x[0] - \hat{x}[0]$ denotes the error in the initial
condition. 

If the matrix $\left( G - L C \right)$ is stable then the errors in
initial condition will diminish eventually. Moreover, in order
to have a good observer/estimator performance the observer
convergence should be sufficiently fast. 

\newpage

\section*{Observer Gain \& Pole Placement}

Similar to the state-feedback gain design,
the fundamental principle of ``pole-placement'' Observer design is that
we first define a desired closed-loop eigenvalue set and 
compute the associated desired characteristic polynomial. 
%
\begin{align*}
 \mathcal{E}^* &= \lbrace \lambda_1^* , \ \cdots, \  \lambda_n^*
                 \rbrace
  \\
  p^*(z) &= \left( z - \lambda_1^* \right) \cdots \left( z - \lambda_n^*
         \right)                         
  \\
  &= z^n + a_1^* z^{n-1} + \cdots + a_{n-1}^* z + a_n^*
\end{align*}
%
The necessary and sufficient condition on arbitrary observer pole-placement
is that the system should be fully Observable. Then, we tune $L$ such
that 
%
\begin{align*}
  \mathrm{det} \left( z I - ( G - L C ) \right) = p^*(z)
\end{align*}
%
\subsection*{Direct Design of Observer Gain}

If $n$ is small, the most efficient method could be the direct
design. 

\textbf{Example:} Consider the following DT system
%
\begin{align*}
 x[k+1] &= \left[ \begin{array}{cc} 1 & 0 \\ 0 & 2 \end{array} \right] x[k]
    + \left[ \begin{array}{c} 1 \\ 1 \end{array} \right] u[k]
\\
 y[k] &= \left[ \begin{array}{cc} 1 & -1 \end{array} \right] u[k]
\end{align*}
% 
Design an observer such that estimater poles are located at 
$\lambda_{1,2} = 0$ (Dead-beat Observer)

\textbf{Solution:} Desired characteristic equation can be computed as
%
\begin{align*}
  p^*(z) = z^2
\end{align*}
%
Let $L = \left[ \begin{array}{c} l_2 \\ l_1 \end{array} \right]$, then
the characteristic equation of $(G - L C)$ can be computed as
%
\begin{align*}
  \mathrm{det} \left( z I - ( G - L C ) \right) &= 
  \mathrm{det} \left(
  \left[ \begin{array}{cc} z - 1 + l_2 & -l_2 \\ l_1 & z - 2 - l_1 \end{array} \right]
  \right)
\\
&= z^2 + z (l_2 - l_1 - 3) + (l_1 - 2 l_2 + 2)
\end{align*}
%
If we match the equations
%
\begin{align*}
  l_2 - l_1 &= 3
\\
  -l_1 + 2 l_2 &= 2
\\
 l_2 &= -1
\\
 l_1 &= -4
\end{align*}
%
Thus $L = \left[ \begin{array}{c} -1 \\ -4 \end{array} \right]$

\subsection*{Design of Observer Gain Using Observable Canonical
Form}

Let's assume that the state-space representation is in observable 
canonical form 
%
\begin{align*}
G &= \left[ \begin{array}{ccccc} 0 & 0 & \cdots & 0 & -a_{n} 
              \\ 1 & 0 & \cdots & 0 & -a_{n-1} 
\\ \vdots & \vdots & \vdots & \vdots & \vdots
\\ 0 & 0 & \cdots & 0 & -a_2
    \\ 0 & 0 & \cdots & 1 & -a_1 \end{array} \right]
\quad , \quad 
H = \left[ \begin{array}{c} (b_n - b_0 a_n)  \\ (b_{n-1} - b_0
             a_{n-1}) \\ \vdots \\ (b_2 - b_0 a_2) \\   (b_1 - b_0
             a_1) 
\end{array} \right]
\\ C &= \left[ \begin{array}{ccccc} 0 & 0 & \cdots &  0 & 1 \end{array} \right]
\quad , \quad
D = b_0
\end{align*}
%
Let $L = \left[ \begin{array}{c} l_n \\ \vdots \\ l_1 \end{array} \right]$, then
$( G - L C )$ takes the form
%
\begin{align*}
\left( G - L C \right) &= \left[ \begin{array}{ccccc} 0 & 0 & \cdots & 0 & -a_{n} 
              \\ 1 & 0 & \cdots & 0 & -a_{n-1} 
\\ \vdots & \vdots & \vdots & \vdots & \vdots
\\ 0 & 0 & \cdots & 0 & -a_2
    \\ 0 & 0 & \cdots & 1 & -a_1 \end{array} \right]
- \left[ \begin{array}{c} l_n \\ \vdots \\ l_1 \end{array} \right]
\left[ \begin{array}{ccccc} 0 & 0 & \cdots &  0 & 1 \end{array}
                                                  \right]
\\
&= 
\left[ \begin{array}{ccccc} 0 & 0 & \cdots & 0 & -(a_{n} + l_n)
              \\ 1 & 0 & \cdots & 0 & - (a_{n-1} + l_{n-1})
\\ \vdots & \vdots & \vdots & \vdots & \vdots
\\ 0 & 0 & \cdots & 0 & - (a_2 + l_n)
    \\ 0 & 0 & \cdots & 1 & - (a_1 + l_n) \end{array} \right]
\end{align*}
%
The characterstic equation of $G - LC$ is simply given as
%
\begin{align*}
  p(z) = z^n + ( a_1 - l_1 ) z^{n-1} + \cdots + ( a_{n-1} - l_{n-1} )
  z + ( a_{n} - l_{n} )
\end{align*}
%
Let's assume that desired $p^*(z)$ is equal to
%
\begin{align*}
  p(z) = z^n + a_1^* z^{n-1} + \cdots + a_{n-1}^* 
  z + a_{n}^*
\end{align*}
%
Then, the observer gain $L$ is computed as
%
\begin{align*}
L^* = \left[ \begin{array}{c} a_n^* - a_n \\ \vdots \\ a_1^* - a_1 \end{array} \right]
\end{align*}

If the system is not in Observable canonical form, we can find a
transformation that outputs the Observable canonical form
representation
%
The Observability matrix of a state-space representation is given as
%
\begin{align*}
 O = \left[ \begin{array}{c} C \\ C G \\ \vdots \\ C G^{-1} \end{array} \right]
\end{align*}
%
Let's define a transformation matrix $Q$ as follows:
%
\begin{align*}
  Q &= \left( W O \right)^{-1} \quad , \quad x[k] = Q \bar{x}[k]
\\
 \bar{x}[k+1] &= \left[ Q^{-1} G Q \right] \bar{x}[k] + Q^{-1} H u[k]
\\ 
  y[k] &= C Q \bar{x}[k] + D u[k]
\end{align*}
%
where
%
\begin{align*}
  W = \left[ \begin{array}{ccccc} a_{n-1} & a_{n-2} & \cdots & a_1 & 1
               \\ 
a_{n-2} & a_{n-3} & \cdots & 1 & 0
\\ \vdots & \vdots & \iddots & & \vdots
\\ a_1 & 1 &  & & 
    \\ 1 & 0 & \cdots &  & 0 \end{array} \right] 
\end{align*}
%
Then it is given that
%
\begin{align*}
  \bar{G} = Q^{-1} G Q  &=  \left[ \begin{array}{ccccc} 0 & 0 & \cdots & 0 & -a_{n} 
              \\ 1 & 0 & \cdots & 0 & -a_{n-1} 
\\ \vdots & \vdots & \vdots & \vdots & \vdots
\\ 0 & 0 & \cdots & 0 & -a_2
    \\ 0 & 0 & \cdots & 1 & -a_1 \end{array} \right]
\\
  \bar{C}  = C Q & = \left[ \begin{array}{ccccc} 0 & 0 & \cdots & 0
    & 1 \end{array} \right]
\end{align*}
%
Let's compute $\bar{C} Q^{-1}$
%
\begin{align*}
\bar{C} Q^{-1} &= \bar{C} W O 
\\
&= \left[ \begin{array}{ccccc} 0 & 0 & \cdots & 0
    & 1 \end{array} \right]
 \left[ \begin{array}{ccccc} a_{n-1} & a_{n-2} & \cdots & a_1 & 1
               \\ 
a_{n-2} & a_{n-3} & \cdots & 1 & 0
\\ \vdots & \vdots & \iddots & & \vdots
\\ a_1 & 1 &  & & 
    \\ 1 & 0 & \cdots &  & 0 \end{array} \right] O^T 
  \\
&= \left[ \begin{array}{ccccc} 1 & 0 & \cdots & 0 & 0 \end{array}
                                                    \right]
\left[ \begin{array}{cccc} C & C G & \cdots & C G^{-1} \end{array}
                                                    \right]
\\
&= C
\end{align*}
%
A similar approach (but longer) can be used to show 
that $Q^{-1} G Q = \bar{G}$. 

We know how to design a
observer gain $\bar{L}$ for the Observable
canonical form. Given $\bar{L}$, Observer gain 
w.r.t. original state-space representation is 
computed as 
%
\begin{align*}
 L = Q \bar{L}
\end{align*}

\textbf{Example 2:} Consider the following DT system
%
\begin{align*}
 x[k+1] &= \left[ \begin{array}{cc} 1 & 0 \\ 0 & 2 \end{array} \right] x[k]
    + \left[ \begin{array}{c} 1 \\ 1 \end{array} \right] u[k]
\\
 y[k] &= \left[ \begin{array}{cc} 1 & -1 \end{array} \right] u[k]
\end{align*}
% 
Design an observer using the Observable canonical form 
such that estimater poles are located at $\lambda_{1,2} = 0$ (Dead-beat Observer)

\textbf{Solution:} Characteristic equation of $G$ can be derived as
%
\begin{align*}
 \mathrm{det} \left(  \left[ \begin{array}{cc} z-1 & 0 \\ 0 &
                                                              z-2 \end{array} \right]  \right)
 = z^2 - 3 z + 2
\end{align*}
%
Observability matrix can be computed as
%
\begin{align*}
 O = \left[ \begin{array}{cc} 1 & -1 \\ 1 &
                                                              -2 \end{array} \right]
\end{align*}
%
The matrix $W$ can be computed as
%
\begin{align*}
  W =  \left[ \begin{array}{cc} -3 & 1 \\ 1 & 0 \end{array} \right]
\end{align*}
%
Transformation matrix $Q$ can be computed as
%
\begin{align*}
  Q &= \left( W O^T \right)^{-1}
\\
&= \left( 
\left[ \begin{array}{cc} -3 & 1 \\ 1 & 0 \end{array} \right] 
\left[ \begin{array}{cc} 1 & -1 \\ 1 & -2 \end{array} \right]
\right)^{-1}
= \left( 
\left[ \begin{array}{cc} -2 & 1 \\ 1 & -1 \end{array} \right] 
\right)^{-1}
=
\left[ \begin{array}{cc} -1 & -1 \\ -1 & -2 \end{array} \right] 
\end{align*}
%
%
Given that desired characteristic polynomial is $p^*(z) = z^2$,
$\bar{L}$ of observable canonical from can be computed as
%
\begin{align*}
  \bar{L} &= \left[ \begin{array}{c} - a_2 \\ -
                                                       a_1 \end{array}
                                                       \right]
\\
&= \left[ \begin{array}{c} -2 \\ 3 \end{array}
                                                       \right]
\end{align*}
%
Finally Observer Gain $L$ can be computed as
%
%
\begin{align*}
 L &= Q \bar{L} = 
 \left[ \begin{array}{c} -1 \\ -4 \end{array} \right]
\end{align*}
% 
Not surprisingly the result is same with the one found in
first example.

\section*{Closed-Loop Observer \& State-Feedback}

In the state-feedback control policy the input is ideally defined
by the following law
%
\begin{align*}
 u[k] = - K x[k]
\end{align*}
%
However, as mentioned in Observer lecture, in general we don't have
direct access to the all states of the system. In this case, we learnt
how to design an Observer/Estimator of the states. In this respect,
it is natural to assume that in a closed-loop system, the control
policy that define the input should depend on the estimated states
%
\begin{align*}
 u[k] = - K \hat{x}[k]
\end{align*}
%
However the important question how this coupling affect the
closed-loop behavior, and even deeper question can be even 
use such a policy. The advantage of LTI systems is that 
state-feedback gain, and observer gain can be seperatelly
designed and we guarntee a stable closed-loop performance. 
In this section, we will analyze the coupled system
%
Equations of motion for the closed-loop observer \& state-feedback
based control system is given below
%
\begin{align*}
   x[k+1] &= G x[k] + H u[k]
  \\ 
   y[k] &= C x[k] + D u[k]
  \\
  \hat{x}[k+1] &= G \hat{x}[k] + H u[k] + L \left( y[k] - \hat{y}[k] \right) 
  \\
  \hat{y}[k] &= C \hat{x}[k] + D u[k]
  \\
   u[k] &= -K \hat{x}[k]
\end{align*}
%
If we eliminate $u[k]$ and $\hat{y}[k]$ we obtain following 
dynamical representation
%
\begin{align*}
   x[k+1] &= G x[k] - H K \hat{x}[k] 
  \\
  \hat{x}[k+1] &= G \hat{x}[k] - H K \hat{x}[k] + L C \left( x[k] -
                 \hat{x}[k] \right) 
  \\ 
   y[k] &= C x[k] - D K \hat{x}[k] 
\end{align*}
%
Now let's replace $\hat{x}[k]$ with $e[k] = x[k] - \hat{x}[k]$
%
\begin{align*}
   x[k+1] &= ( G - H K ) x[k] + H K e[k]
  \\
  e[k+1] &= ( G - LC ) e[k]
  \\ 
   y[k] &= (C - D K) x[k] + D K e[k]
\end{align*}
%
Now let's defina a state for the whole system, 
$z[k] = \left[ \begin{array}{c} x[k] \\ e[k] \end{array} \right]$
then the state-space representation is given by
%
\begin{align*}
  z[k+1] = \left[ \begin{array}{cc} (G - H K) & H K\\ 0_{n \times n}
                                              & (G - LC) \end{array}
                                                \right] z[k]
\\
 y[k] = \left[ \begin{array}{cc} (C - DK) & D K \end{array}
                                                \right] z[k]
\end{align*}
%
The system matrix is in block diagonal form and the eigenvalues
of this new system matrix is find by taking the union of eigenvalues
of $(G - H K)$ and eigenvalues of $(G - L C)$. Thus a seperate
pole-placement can be performed for the state-feedback controller
and the observer. 

% **** This ENDS THE EXAMPLES. DON'T DELETE THE FOLLOWING LINE:
\end{document}
%
