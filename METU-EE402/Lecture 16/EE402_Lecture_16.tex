% 


\documentclass[twoside]{article}
\setlength{\oddsidemargin}{0.25 in}
\setlength{\evensidemargin}{-0.25 in}
\setlength{\topmargin}{-0.6 in}
\setlength{\textwidth}{6.5 in}
\setlength{\textheight}{8.5 in}
\setlength{\headsep}{0.75 in}
\setlength{\parindent}{0 in}
\setlength{\parskip}{0.1 in}

%
% ADD PACKAGES here:
%

\usepackage{amsmath,amsfonts,graphicx,mathdots}


\newcounter{lecnum}
\renewcommand{\thepage}{\thelecnum-\arabic{page}}
\renewcommand{\thesection}{\thelecnum.\arabic{section}}
\renewcommand{\theequation}{\thelecnum.\arabic{equation}}
\renewcommand{\thefigure}{\thelecnum.\arabic{figure}}
\renewcommand{\thetable}{\thelecnum.\arabic{table}}

%
% The following macro is used to generate the header.
%
\newcommand{\lecture}[4]{
   \pagestyle{myheadings}
   \thispagestyle{plain}
   \newpage
   \setcounter{lecnum}{#1}
   \setcounter{page}{1}
   \noindent
   \begin{center}
   \framebox{
      \vbox{\vspace{2mm}
    \hbox to 6.28in { {\bf EE402 - Discrete Time Systems
	\hfill Spring 2018} }
       \vspace{4mm}
       \hbox to 6.28in { {\Large \hfill Lecture #1 \hfill} }
       \vspace{2mm}
       \hbox to 6.28in { {\it Lecturer: #2 \hfill } }
      \vspace{2mm}}
   }
   \end{center}
   \markboth{Lecture #1}{Lecture #1}

   \vspace*{4mm}
}

\renewcommand{\cite}[1]{[#1]}
\def\beginrefs{\begin{list}%
        {[\arabic{equation}]}{\usecounter{equation}
         \setlength{\leftmargin}{2.0truecm}\setlength{\labelsep}{0.4truecm}%
         \setlength{\labelwidth}{1.6truecm}}}
\def\endrefs{\end{list}}
\def\bibentry#1{\item[\hbox{[#1]}]}


\newcommand{\fig}[3]{
			\vspace{#2}
			\begin{center}
			Figure \thelecnum.#1:~#3
			\end{center}
	}

% Use these for theorems, lemmas, proofs, etc.
\newtheorem{theorem}{Theorem}[lecnum]
\newtheorem{lemma}[theorem]{Lemma}
\newtheorem{proposition}[theorem]{Proposition}
\newtheorem{claim}[theorem]{Claim}
\newtheorem{corollary}[theorem]{Corollary}
\newtheorem{definition}[theorem]{Definition}
\newenvironment{proof}{{\bf Proof:}}{\hfill\rule{2mm}{2mm}}

% **** IF YOU WANT TO DEFINE ADDITIONAL MACROS FOR YOURSELF, PUT THEM HERE:

\begin{document}

% Lecture Details
\lecture{16}{Asst. Prof. M. Mert Ankarali}

\par

\section*{State-Feedback \& Pole Placement}

Given a discrete-time state-evolution equation
%
\begin{align*}
  x[k+1] = G x[k] + H u[k]
\end{align*}
%
If direct measurements of all of the states of the
system (e.g. $y[k] = x[k]$) are available, one of the most
popular control methods is the linear state feedback
control,
%
\begin{align*}
 u[k] = - K x[k]
\end{align*}
%
which can be thought as a generalization of P controller
to the vector form. Under this control law, without any
reference signal, the system becomes an autonomous 
system
%
%
\begin{align*}
 x[k+1] = G x[k] + H (-K x[k])
  \\
 x[k+1] = \left(  G - H K \right) x[k] 
\end{align*}
%
The system matrix of this new autonomous system is
$\hat{G} = G - H K$. The important question is how we need to choose K?. 
Note that 
%
\begin{align*}
  K &\in \mathbb{R}^n \quad \mathrm{Single-Input} \\
  K &\in \mathbb{R}^{n \times p} \quad \mathrm{Multi-Input}
\end{align*}
%
As in all of the control design techniques, the most critical criterion is stability, thus we want all of the eigenvalues to be strictly inside the unit circle. However, we know that there could be different requirements on the poles/eigenvalues of the system.

The fundamental principle of ``pole-placement'' design is that
we first define a desired closed-loop eigenvalue set 
$\mathcal{E}^* = \lbrace \lambda_1^* , \ \cdots, \  \lambda_1^* \rbrace$, and
then if possible we choose $K^*$ such that the closed-loop
eigenvalues match the desired ones.

The necessary and sufficient condition on arbitrary pole-placement is that the system should be fully reachable.

In Pole-Placement, the first step is computing the desired characteristic polynomial. 
%
\begin{align*}
 \mathcal{E}^* &= \lbrace \lambda_1^* , \ \cdots, \  \lambda_n^*
                 \rbrace
  \\
  p^*(z) &= \left( z - \lambda_1^* \right) \cdots \left( z - \lambda_n^*
         \right)                         
  \\
  &= z^n + a_1^* z^{n-1} + \cdots + a_{n-1}^* z + a_n^*
\end{align*}
%
Then we tune $K$ such that 
\begin{align*}
  \mathrm{det} \left( z I - ( G - H K ) \right) = p^*(z)
\end{align*}

\subsection*{Direct Design of State-Feedback Gain}

If $n$ is small, the most efficient method could be the direct
design. 

\textbf{Example:} Consider the following DT system
%
\begin{align*}
 x[k+1] &= \left[ \begin{array}{cc} 1 & 0 \\ 0 & 2 \end{array} \right] x[k]
    + \left[ \begin{array}{c} 1 \\ 1 \end{array} \right] u[k]
\end{align*}
% 
Design a state-feedback rule such that poles are located at 
$\lambda_{1,2} = 0$ (Dead-beat gain)

\textbf{Solution:} Desired characteristic equation can be computed as
%
\begin{align*}
  p^*(z) = z^2
\end{align*}
%
Let $K = \left[ \begin{array}{cc} k_1 & k_2 \end{array} \right]$, then
the characteristic equation of $\hat{G}$ can be computed as
%
\begin{align*}
  \mathrm{det} \left( z I - ( G - H K ) \right) &= 
  \mathrm{det} \left(
  \left[ \begin{array}{cc} z - 1 + k_1 & k_2 \\ k_1 & z - 2 + k_2 \end{array} \right]
  \right)
\\
&= z^2 + z (k_1 + k_2 - 3) + (2 -2 k_1 - k_2)
\end{align*}
%
If we match the equations
%
\begin{align*}
z^2 + z (k_1 + k_2 - 3) + (2 -2 k_1 - k_2) &= z^2
\\
k_1 + k_2 &= 3
\\
2k_1 + k_2 &= 2
\\
k_1 &= -1
\\
k_2 &= 4
\end{align*}
%
Thus $K = \left[ \begin{array}{cc} -1 & 4 \end{array} \right]$.
Now let's compute $\hat{G}^k$ 
%
\begin{align*}
\hat{G} &= \left[ \begin{array}{cc} 2 & -4 \\ 1 & -2 \end{array} \right] 
\\
\hat{G}^2 &= \left[ \begin{array}{cc} 2 & -4 \\ 1 & -2 \end{array} \right] 
\left[ \begin{array}{cc} 2 & -4 \\ 1 & -2 \end{array} \right] = 
\left[ \begin{array}{cc} 0 & 0 \\ 0 & 0 \end{array} \right]
\\
&\vdots
\end{align*}
%
It can be seen that closed-loop system rejects all initial condition
perturbations in 2 steps. 

\subsection*{Design of State-Feedback Gain Using Reachable Canonical
Form}

Let's assume that the state-space representation is in reachable canonical form and we have access to all states of this form
%
\begin{align*}
x[k+1] &= \left[ \begin{array}{ccccc} 0 & 1 & 0 & \cdots & 0 \\ 0 & 0 & 1 &
                                                                      \cdots & 0
\\ \vdots & \vdots & \vdots & & \vdots
\\ 0 & 0 & 0 & \cdots & 1
    \\ -a_n & -a_{n-1} & -a_{n-2} & \cdots & -a_1 \end{array} \right] x[k] +
\left[ \begin{array}{c} 0\\ 0 \\ \vdots \\ 0
    \\ 1 \end{array} \right] u[k]
\end{align*}
%
Let $K = \left[ \begin{array}{ccc} k_n & \cdots & k_1 \end{array} \right]$, then
autonomous system takes the form
%
%
\begin{align*}
x[k+1] &= \left[ \begin{array}{ccccc} 0 & 1 & 0 & \cdots & 0 \\ 0 & 0 & 1 &
                                                                      \cdots & 0
\\ \vdots & \vdots & \vdots & & \vdots
\\ 0 & 0 & 0 & \cdots & 1
    \\ -a_n & -a_{n-1} & -a_{n-2} & \cdots & -a_1 \end{array} \right] x[k] -
\left[ \begin{array}{c} 0\\ 0 \\ \vdots \\ 0
    \\ 1 \end{array} \right] 
\left[ \begin{array}{ccc} k_n & \cdots & k_1 \end{array} \right]
x[k]
\\
x[k+1] &= \left[ \begin{array}{ccccc} 0 & 1 & 0 & \cdots & 0 \\ 0 & 0 & 1 &
                                                                      \cdots & 0
\\ \vdots & \vdots & \vdots & & \vdots
\\ 0 & 0 & 0 & \cdots & 1
    \\ -(a_n+k_n) & -(a_{n-1} + k_{n-1}) & -(a_{n-2} + k_{n-2}) &
                                                                  \cdots
                                                         & -(a_1 + k_1) \end{array} \right] x[k]
\end{align*}
%
Let $p^*(z) = z^2 + a_1^* z + \cdots + a_{n-1}^* z + a_n^*$, then $K$
can be computed as
%
\begin{align*}
  K = \left[ \begin{array}{ccc} (a^*_n - a_n) & \cdots & (a^*_1 - a_1) \end{array} \right]
\end{align*}

However, what if the system is not in Reachable canonical form. We can
find a transformation which finds the Reachable canonical
representation. 

The reachability matrix of a state-space representation is given as
%
\begin{align*}
  M = \left[ \begin{array}{c|c|c|c} H & G H & \cdots & G^{n-1} H\end{array} \right]
\end{align*}
%
Let's define a transformation matrix $T$ as follows:
%
\begin{align*}
  T &= M W \quad , \quad x[k] = T \hat{x}[k]
\\
 \hat{x}[k+1] &= \left[ T^{-1} G T \right] \hat{x}[k] + T^{-1} H u[k]
\end{align*}
%
where
%
\begin{align*}
  W = \left[ \begin{array}{ccccc} a_{n-1} & a_{n-2} & \cdots & a_1 & 1
               \\ 
a_{n-2} & a_{n-3} & \cdots & 1 & 0
\\ \vdots & \vdots & \iddots & & \vdots
\\ a_1 & 1 &  & & 
    \\ 1 & 0 & \cdots &  & 0 \end{array} \right] 
\end{align*}
%
Then it is given that
%
\begin{align*}
  T^{-1} G T &= \hat{G} =  \left[ \begin{array}{ccccc} 0 & 1 & 0 & \cdots & 0 \\ 0 & 0 & 1 &
                                                                      \cdots & 0
\\ \vdots & \vdots & \vdots & & \vdots
\\ 0 & 0 & 0 & \cdots & 1
    \\ -a_n & -a_{n-1} & -a_{n-2} & \cdots & -a_1 \end{array} \right]
\\
  T^{-1} H &= \hat{H} = \left[ \begin{array}{c} 0\\ 0 \\ \vdots \\ 0
    \\ 1 \end{array} \right]
\end{align*}
%
Let's compute $T \hat{H}$
%
\begin{align*}
T \hat{H} &= M W \hat{H} = M 
\left[ \begin{array}{ccccc} a_{n-1} & a_{n-2} & \cdots & a_1 & 1
               \\ 
a_{n-2} & a_{n-3} & \cdots & 1 & 0
\\ \vdots & \vdots & \iddots & & \vdots
\\ a_1 & 1 &  & & 
    \\ 1 & 0 & \cdots &  & 0 \end{array} \right] 
\left[ \begin{array}{c} 0\\ 0 \\ \vdots \\ 0
    \\ 1 \end{array} \right]
\\
&= \left[ \begin{array}{c|c|c|c} H & G H & \cdots & G^{n-1}
                                                    H\end{array}
                                                    \right]
\left[ \begin{array}{c} 1\\ 0 \\ \vdots \\ 0
    \\ 0 \end{array} \right]
\\
&= H
\end{align*}
%
A similar approach (but longer) can be used to show 
that $T^{-1} G T = \hat{G}$. We know how to design a
state-feedback gain $\hat{K}$ for the Reachable
canonical form. Given $\hat{K}$,
$u[k]$ is given as
%
\begin{align*}
  u[k] &= -\hat{K} \hat{x}[k]
\\ 
&= - \hat{K} T^{-1} \hat{x}[k]
\\
K &= \hat{K} T^{-1}
\end{align*}

\textbf{Example 2:} Consider the following DT system
%
\begin{align*}
 x[k+1] &= \left[ \begin{array}{cc} 1 & 0 \\ 0 & 2 \end{array} \right] x[k]
    + \left[ \begin{array}{c} 1 \\ 1 \end{array} \right] u[k]
\end{align*}
% 
Design a state-feedback rule using the Reachable canonical form
approach, such that poles are located at 
$\lambda_{1,2} = 0$ (Dead-beat gain)

\textbf{Solution:} Characteristic equation of $G$ can be derived as
%
\begin{align*}
 \mathrm{det} \left(  \left[ \begin{array}{cc} z-1 & 0 \\ 0 &
                                                              z-2 \end{array} \right]  \right)
 = z^2 - 3 z + 2
\end{align*}
%
The Reachability matrix can be computed as
%
\begin{align*}
  M = \left[ \begin{array}{c|c} H & G H \end{array} \right] = \left[ \begin{array}{cc} 1 & 1 \\ 1 & 2 \end{array} \right]
\end{align*}
%
The matrix $W$ can be computed as
%
\begin{align*}
  W =  \left[ \begin{array}{cc} -3 & 1 \\ 1 & 0 \end{array} \right]
\end{align*}
%
Transformation matrix, $T$ and its inverse $T^{-1}$ can be computed 
as
\begin{align*}
  T &= M W =  
\left[ \begin{array}{cc} 1 & 1 \\ 1 & 2 \end{array} \right]
\left[ \begin{array}{cc} -3 & 1 \\ 1 & 0 \end{array} \right]
= \left[ \begin{array}{cc} -2 & 1 \\ -1 & 1 \end{array} \right]
\\
T^{-1} &= \left[ \begin{array}{cc} -1 & 1 \\ -1 & 2 \end{array} \right]
\end{align*}
%
Given that desired characteristic polynomial is $p^*(z) = z^2$,
$\hat{K}$ of reachable canonical from can be computed as
%
\begin{align*}
  \hat{K} &= \left[ \begin{array}{cc} - a_2 & -
                                                       a_1 \end{array}
                                                       \right]
\\
&= \left[ \begin{array}{ccc} -2 & 3 \end{array}
                                                       \right]
\end{align*}
%
Finally $K$ can be computed as
%
\begin{align*}
  K &= \hat{K} T^{-1} = \left[ \begin{array}{ccc} -2 & 3 \end{array}
                                                       \right]
\left[ \begin{array}{cc} -1 & 1 \\ -1 & 2 \end{array} \right]
\\
&= \left[ \begin{array}{ccc} -1 & 4 \end{array}
                                                       \right]
\end{align*}
%
As expected this is the same result with the one found in Example 1 (Direct-Method).

% **** This ENDS THE EXAMPLES. DON'T DELETE THE FOLLOWING LINE:
\end{document}
