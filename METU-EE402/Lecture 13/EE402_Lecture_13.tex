%


\documentclass[twoside]{article}
\setlength{\oddsidemargin}{0.25 in}
\setlength{\evensidemargin}{-0.25 in}
\setlength{\topmargin}{-0.6 in}
\setlength{\textwidth}{6.5 in}
\setlength{\textheight}{8.5 in}
\setlength{\headsep}{0.75 in}
\setlength{\parindent}{0 in}
\setlength{\parskip}{0.1 in}

%
% ADD PACKAGES here:
%

\usepackage{amsmath,amsfonts,graphicx}


\newcounter{lecnum}
\renewcommand{\thepage}{\thelecnum-\arabic{page}}
\renewcommand{\thesection}{\thelecnum.\arabic{section}}
\renewcommand{\theequation}{\thelecnum.\arabic{equation}}
\renewcommand{\thefigure}{\thelecnum.\arabic{figure}}
\renewcommand{\thetable}{\thelecnum.\arabic{table}}

%
% The following macro is used to generate the header.
%
\newcommand{\lecture}[4]{
   \pagestyle{myheadings}
   \thispagestyle{plain}
   \newpage
   \setcounter{lecnum}{#1}
   \setcounter{page}{1}
   \noindent
   \begin{center}
   \framebox{
      \vbox{\vspace{2mm}
    \hbox to 6.28in { {\bf EE402 - Discrete Time Systems
	\hfill Spring 2018} }
       \vspace{4mm}
       \hbox to 6.28in { {\Large \hfill Lecture #1 \hfill} }
       \vspace{2mm}
       \hbox to 6.28in { {\it Lecturer: #2 \hfill } }
      \vspace{2mm}}
   }
   \end{center}
   \markboth{Lecture #1}{Lecture #1}

   \vspace*{4mm}
}

\renewcommand{\cite}[1]{[#1]}
\def\beginrefs{\begin{list}%
        {[\arabic{equation}]}{\usecounter{equation}
         \setlength{\leftmargin}{2.0truecm}\setlength{\labelsep}{0.4truecm}%
         \setlength{\labelwidth}{1.6truecm}}}
\def\endrefs{\end{list}}
\def\bibentry#1{\item[\hbox{[#1]}]}


\newcommand{\fig}[3]{
			\vspace{#2}
			\begin{center}
			Figure \thelecnum.#1:~#3
			\end{center}
	}

% Use these for theorems, lemmas, proofs, etc.
\newtheorem{theorem}{Theorem}[lecnum]
\newtheorem{lemma}[theorem]{Lemma}
\newtheorem{proposition}[theorem]{Proposition}
\newtheorem{claim}[theorem]{Claim}
\newtheorem{corollary}[theorem]{Corollary}
\newtheorem{definition}[theorem]{Definition}
\newenvironment{proof}{{\bf Proof:}}{\hfill\rule{2mm}{2mm}}

% **** IF YOU WANT TO DEFINE ADDITIONAL MACROS FOR YOURSELF, PUT THEM HERE:

\begin{document}

% Lecture Details
\lecture{13}{Asst. Prof. M. Mert Ankarali}

\par

\section*{Matrix Exponential, $e^{At}$}

Let's first review the matrix exponential, $e^{A t}$.
Let $t \in \mathbb{R}$ and $A \in \mathbb{R}^{nxn}$, then $e^{At}$ defined as
%
\begin{align*}
       e^{A t} &:= I + A t + \frac{1}{2 !} A^2 t^2 + \frac{1}{3 !} A^3
                 t^3 + \cdots 
\\
	e^{A t} &:= \sum\limits_{k=0}^{\infty} \frac{t^k}{k!} A^k
\end{align*}
%
which converges for all  $t \in \mathbb{R}$ and $A \in
\mathbb{R}^{nxn}$.

Now let's review some properties

\begin{itemize}

 \item \textbf{Claim:}
%
\begin{align*}
  \frac{d}{d t} e^{A t} &= A e^{A t} = e^{A t} A
\end{align*}

\textbf{Proof:} 
%
\begin{align*}
 	\frac{d}{dt} e^{A t} &= \frac{d}{dt} \left( \sum\limits_{k=0}^{\infty} \frac{t^{k}}{k!} A^k \right) =
	\sum\limits_{k=0}^{\infty} k \frac{t^{k-1}}{k!} A^k = \sum\limits_{k=1}^{\infty} \frac{t^{k-1}}{(k-1)!} A^k \\
	&= \sum\limits_{n=0}^{\infty} \frac{t^{n}}{n!} A^{n+1} = A \sum\limits_{n=0}^{\infty} \frac{t^{n}}{n!} A^{n}   = \left( \sum\limits_{n=0}^{\infty} \frac{t^{n}}{n!} A^{n} \right) A
	\\
	\frac{d}{dt} e^{A t}  &= A e^{A t} = e^{A t} A
\end{align*}

\item \textbf{Claim:} Let $t_1 , t_2 \in \mathrm{R}$ then

\begin{align*}
  e^{A t_1} e^{A t_2} = e^{A t_2} e^{A t_1} = e^{A \left( t_1 + t_2 \right)}
\end{align*}

\textbf{Proof:}
%
\begin{align*}
	e^{A t_1} e^{A t_2} &= \left(\sum\limits_{k=0}^{\infty} \frac{t_1^k}{k!} A^k \right) \left(\sum\limits_{j=0}^{\infty} \frac{t_2^j}{j!} A^j \right) 
	= \sum\limits_{k=0}^{\infty} \sum\limits_{j=0}^{\infty} \frac{t_1^k}{k!} \frac{t_2^j}{j!} A^{k+j}
\end{align*}
%
Let $n = k + j$ and $j = n - k$, then
%
\begin{align*}
	e^{A t_1} e^{A t_2} &= \sum\limits_{k=0}^{\infty} \sum\limits_{n=k}^{\infty} \frac{t_1^k \ t_2^{n-k}}{k! (n-k)!} A^n 
	= \sum\limits_{k=0}^{\infty} \sum\limits_{n=k}^{\infty} \frac{t_1^k \ t_2^{n-k}}{n!} \frac{n!}{k! (n-k)!} A^n 
    &= \sum\limits_{k=0}^{\infty} \sum\limits_{n=k}^{\infty} \frac{t_1^k \ t_2^{n-k}}{n!} 
    \left( \begin{array}{c} n \\ k \end{array} \right) A^n
\end{align*}
%
Since for $n<k,  \left( \begin{array}{c} n \\ k \end{array} \right) = 0$,
%
\begin{align*}
	e^{A t_1} e^{A t_2} &= \sum\limits_{k=0}^{\infty} \sum\limits_{n=0}^{\infty} \frac{t_1^k \ t_2^{n-k}}{n!} 
    \left( \begin{array}{c} n \\ k \end{array} \right) A^n 
    = \sum\limits_{n=0}^{\infty} \frac{A^n}{n!} \sum\limits_{k=0}^{\infty} t_1^k \ t_2^{n-k}
    \left( \begin{array}{c} n \\ k \end{array} \right) 
\end{align*}
%
Using \textit{binomial} theorem we find
%
\begin{align*}
	e^{A t_1} e^{A t_2} &= \sum\limits_{n=0}^{\infty} \frac{A^n}{n!} (t_1 + t_2)^n \\
	e^{A t_1} e^{A t_2} &= e^{A (t_1 + t_2)}
\end{align*}

Now let $t_1 = t$ and $t_2 = -t$, then we have
%
\begin{align*}
	e^{A t} e^{-A t} &= e^{A (t - t)} = I \quad \rightarrow \quad
                           \left( e^{A t} \right)^{-1} = e^{-A t}
\end{align*}

\item \textbf{Claim:} Let $A , B \in \mathrm{R}^{n \times n}$ and $A B
  = B A$, then 

\begin{align*}
	e^{A t} e^{B t} &= e^{B t} e^{A t} = e^{(A + B) t} 
\end{align*}

\textbf{Proof:} Possible mini project question

\par 

Note that if $A B \neq B A$ then

\begin{align*}
	e^{A t} e^{B t} \neq e^{(A + B) t} 
\end{align*}

\item \textbf{Claim:} Let $P \in \mathrm{R}^{n \times n}$ and
 $\mathrm{det}(P) \neq 0$, then

\begin{align*}
	e^{\left(P^{-1} A P \right) t} = P^{-1} e^{A t} P
\end{align*}

\textbf{Proof:} Possible mini project question

\end{itemize}

\section*{Solution of CT State-Space Equations}

CT state-space representation has the form
%
\begin{align*}
  \dot{x}(t) &= A x(t) + B u(t)
\\
 \dot{y}(t) &= C x(t) + D u(t)
\end{align*}
%
where $\mathrm{Let} \ x(t) \in \mathbb{R}^n \ , \ y(t) \in \mathbb{R}^p \ ,\  u(t) \in
  \mathbb{R}^q$

First consider the homogeneous solution, i.e. $u(t) = 0$ and $x(0) =
x_0$.
%
\begin{align*}
  \dot{x}(t) &= A x(t) \quad , \quad x(0) = x_0
\\
 \dot{y}(t) &= C x(t) 
\end{align*}
%
Let's test if $x(t) = e^{A t} x_0$ is a solution of the homogeneous 
equation
%
%
\begin{align*}
  x(0) &= e^{A 0} x_0 = x_0
  \\
  \dot{x}(t) - A x(t) &= \left( A e^{At} \right) x_0 - A e^{At} x_0 = 0
\end{align*}
%


Now let's compute the forced response. First let's analyze the
following derivative
%
\begin{align*}
\frac{d}{dt} \left[ e^{-A t} x(t) \right] = e^{-A t} \dot{x}(t) - e^{-A t} A
  x(t) = e^{-A t} \left[ \dot{x}(t) - A x(t) \right]
\end{align*}
%
Now using this relation let's solve the state-space equations
%
\begin{align*}
  \dot{x}(t) &= A x(t) + B u(t)
\\ 
  \dot{x}(t) - A x(t) &=  B u(t)
\\
  e^{-A t}  \left[ \dot{x}(t) - A x(t) \right] &=  e^{-A t} B u(t)
\\
  \frac{d}{dt} \left[ e^{-A t} x(t) \right] &=  e^{-A t} B u(t)
\\
  e^{-A t} x(t) &= x(0) + \int\limits_{0}^{t} e^{-A \tau} B u(\tau) d
                  \tau
\\
x(t) &= e^{A t} x_0 + \int\limits_{0}^{t} e^{A ( t - \tau ) } B u(\tau) d
                  \tau
\end{align*}
%
Thus the solution of a system in state-space form can be written as
%
\begin{align*}
  x(t) &= e^{A t} x_0 + \int\limits_{0}^{t} e^{A ( t - \tau ) } B u(\tau) d
                  \tau
\\
  y(t) &= C e^{A t} x_0 + \int\limits_{0}^{t} C e^{A ( t - \tau ) } B u(\tau) d
                  \tau + D u(t)
\end{align*}
%

The function $\Psi(t) = e^{A t}$ is called the state-transition matrix
of the system. 

\textbf{Example:} Let's assume that system is a SISO system and
$u(t) = \delta({t})$ (unit-impulse function) and $x_0 = 0$, compute 
the impulse response of the system, i.e. $y(t) = h(t)$,

\begin{align*}
  h(t) &= \int\limits_{0}^{t} C e^{A ( t - \tau ) } B \delta(\tau) d
                  \tau + D \delta(t)
\\
   &= C e^{A t } B + D \delta(t)
\end{align*}

\subsection*{S-Domain Solution of CT State-Space Equations}

First take the Laplace transform of state evaluation equation
%
\begin{align*}
\mathcal{L} \left[ \dot{x}(t) \right] &= \mathcal{L} \left[  A x(t) + B u(t) \right]
\\
s X(s) - x(0) &= A X(s) + B U(s)
\\
[s I - A]  X(s) &= x(0) + B U(s)
\\
X(s) &= [s I - A]^{-1} x_0 + [s I - A]^{-1} B U(s)
\\
Y(s) &= C [s I - A]^{-1} x_0 + \left[ C [s I - A]^{-1} B + D\right] U(s)
\end{align*}
%
If we relate time and s-domain solutions we obtain
%
\begin{align*}
  e^{A t} &= \mathcal{L}^{-1} \left[ [s I - A]^{-1} \right]
\\
  h(t) &= \mathcal{L}^{-1} \left[ C [s I - A]^{-1} B + D \right]
\end{align*}

\section*{Discretization of CT State-Space Equations}

Consider the CT system with the given state-space representation
%
%
\begin{align*}
  \dot{x}(t) &= A x(t) + B u(t)
\\
  y(t) &= C x(t) + D u(t)
\end{align*}
%
and suppose that the input is piece-wise constant over intervals of
length $T$. That is
%
\begin{align*}
  u(t) = u[k] \quad , \quad t \in ( kT \, \ (k+1) T)
\end{align*}
%
i.e. input of the system is the output of a ZOH operator. Let's derive the
the DT state-space equations with respect to the sampled-state 
$x[k] = x(kT)$.

Let's start with the state evolution equation
%
\begin{align*}
  x(t) &= e^{A t} x(0) + \int\limits_{0}^{t} e^{A ( t - \tau ) } B
         u(\tau) d \tau
\end{align*}
%
Due to time-invariant nature of the dynamics, we can 
generalize the initial time of the equation as
%
\begin{align*}
  x(t) &= e^{A (t - t_0)} x(t_0) + \int\limits_{t_0}^{t} e^{A ( t - \tau ) } B
         u(\tau) d \tau \quad , \quad t > t_0
\end{align*}
%
Now let $t_0 = k T$ and $t = (k+1) T$,
%
\begin{align*}
  x( (k+1) T ) &= e^{A T} x(k T) + \int\limits_{kT}^{(k+1)T} e^{A ( (k+1)T - \tau ) } B
         u(\tau) d \tau 
\end{align*}
%
Since the $u(t) = u[k]$ in this time interval
%
\begin{align*}
  x[k+1] &= e^{A T} x[k] + \int\limits_{kT}^{(k+1)T} e^{A ( (k+1)T -
           \tau ) } B d \tau \ u[k]
\end{align*}
%
Let $\lambda = (k+1)T -\tau $, then
%
\begin{align*}
  x[k+1] &= e^{A T} x[k] - \int\limits_{T}^{0} e^{A \lambda } B d \lambda \ u[k]
\\
&= \left[ e^{A T} \right] x[k] + \left[\left( \int\limits_{0}^{T}
  e^{A \lambda } d \lambda \right) B \right] u[k]
\end{align*}
% 
Given that 
%
\begin{align*}
  x[k+1] &= G x[k] + H u[k]
\end{align*}
%
$G$ and $H$ matrices can be extracted as
%
\begin{align*}
  G &= e^{A T}
\\
 H &= \left( \int\limits_{0}^{T}
  e^{A \lambda } d \lambda \right) B
\end{align*}
%
\textbf{Claim:} If $A$ is invertable then we also have
%
\begin{align*}
 H &= A^{-1} \left( e^{AT} - I \right) B = \left( e^{AT} - I \right)
     A^{-1}  B  
\end{align*}
%
\textbf{Proof:} Possible mini project question

Now let's consider the output equation
%
%
\begin{align*}
  y(t) &= C x(t) + D u(t)
         \\
 y(k T) &= C x( k T) + D u( k T) 
\\
 y[k] &= C x[k] + D u[k] 
\end{align*}
%
It can be seen that output equation matrices are not affected from
the discretization.

\textbf{Example:} Consider the following CT-plant transfer function. 
%
\begin{align*}
  \frac{Y(s)}{U(s)} = \frac{1}{s} + \frac{1}{s+\mathrm{ln}(2)}
\end{align*}
%
Find a CT state-space representation for this system


\textbf{Solution:}

\begin{align*}
 \dot{x}(t) &= \left[ \begin{array}{cc} 0 & 0 \\ 0 & -
         \mathrm{ln}(2) \end{array} \right] x(t)
    + \left[ \begin{array}{c} 1 \\ 1 \end{array} \right] u(t)
 \\
  y(t) &= \left[ \begin{array}{cc} 1 & 1 \end{array} \right] x(t)
\end{align*}

Compute the state-transition matrix

\textbf{Solution:}

\begin{align*}
 e^{A t} &= \left[ 
\begin{array}{cc} 
                     e^{0 t} & 0 \\ 0 & e^{-\mathrm{ln}(2) t } 
\end{array} \right] 
\\ 
 &= \left[ 
\begin{array}{cc} 
             1 & 0 \\ 0 & 0.5^t 
\end{array} \right] 
\end{align*}

Discretize the CT State-Space equation under zero hold operation 
and ideal sampling of the defined state variables, with $T = 1 s$.

\textbf{Solution:}
\begin{align*}
  G &=  e^{A T} = \left[ 
\begin{array}{cc} 
             1 & 0 \\ 0 & 0.5
\end{array} \right] 
\\
H &=  \left( \int\limits_{0}^{T}
  e^{A \lambda } d \lambda \right) B
= \left( \int\limits_{0}^{1}
\left[ 
\begin{array}{cc} 
             1 & 0 \\ 0 & 0.5^{\lambda}
\end{array} \right] 
  d \lambda \right) \left[ \begin{array}{c} 1 \\ 1 \end{array} \right]
=
\int\limits_{0}^{1}
\left[ 
\begin{array}{c} 
             1  \\ 0.5^{\lambda}
\end{array} \right]  d \lambda
\\
H &= \left[ \begin{array}{c} 
             1  \\ 0.721
\end{array} \right]  
\end{align*}
Full DT state-space formulation takes the form
%
\begin{align*}
 x[k+1] &= \left[ \begin{array}{cc} 1 & 0 \\ 0 & 0.5 \end{array} \right] x[k]
    + \left[ \begin{array}{c} 1 \\ 0.721 \end{array} \right] u[k]
 \\
  y[k] &= \left[ \begin{array}{cc} 1 & 1 \end{array} \right] x[k]
\end{align*}

Compute the DT pulse transfer function $Y(z)/U(z)$
%
 \textbf{Solution:}

\begin{align*}
 \frac{Y(z)}{R(z)} &= C \left[ z I - G \right]^{-1} H
\\
&= \left[ \begin{array}{cc} 1 & 1 \end{array} \right] 
\left[ z I - G \right]^{-1}
\left[ \begin{array}{c} 1 \\ 0.721 \end{array} \right]
\\
&= \left[ \begin{array}{cc} 1 & 1 \end{array} \right] 
\left[ 
\begin{array}{cc} 
             z-1 & 0 \\ 0 & z-0.5
\end{array} 
\right]^{-1}
\left[ \begin{array}{c} 1 \\ 0.721 \end{array} \right]
\\
&= \left[ \begin{array}{cc} 1 & 1 \end{array} \right] 
\left[ 
\begin{array}{cc} 
            \frac{1}{z-1} & 0 \\ 0 & \frac{1}{z-0.5}
\end{array} 
\right]
\left[ \begin{array}{c} 1 \\ 0.721 \end{array} \right]
\\
&= \frac{1}{z-1} + \frac{0.721}{z-0.5}
\end{align*}

Now discretize $Y(s)/U(s)$ directly under ZOH operation
%

\textbf{Solution:}
%
\begin{align*}
 \frac{Y(z)}{U(z)} &= \mathcal{Z} \left[ \frac{1 - e^{-s}}{s}
                     \frac{Y(s)}{U(s)} \right] = \frac{1}{z-1} + \frac{0.721}{z-0.5}
\end{align*}

% **** This ENDS THE EXAMPLES. DON'T DELETE THE FOLLOWING LINE:
\end{document}
