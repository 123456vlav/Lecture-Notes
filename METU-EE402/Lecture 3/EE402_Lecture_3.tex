%
% This is a borrowed LaTeX template file for lecture notes for CS267,
% Applications of Parallel Computing, UCBerkeley EECS Department.
% Now being used for CMU's 10725 Fall 2012 Optimization course
% taught by Geoff Gordon and Ryan Tibshirani.  When preparing 
% LaTeX notes for this class, please use this template.
%
% To familiarize yourself with this template, the body contains
% some examples of its use.  Look them over.  Then you can
% run LaTeX on this file.  After you have LaTeXed this file then
% you can look over the result either by printing it out with
% dvips or using xdvi. "pdflatex template.tex" should also work.
%

\documentclass[twoside]{article}
\setlength{\oddsidemargin}{0.25 in}
\setlength{\evensidemargin}{-0.25 in}
\setlength{\topmargin}{-0.6 in}
\setlength{\textwidth}{6.5 in}
\setlength{\textheight}{8.5 in}
\setlength{\headsep}{0.75 in}
\setlength{\parindent}{0 in}
\setlength{\parskip}{0.1 in}

%
% ADD PACKAGES here:
%

\usepackage{amsmath,amsfonts,graphicx}

%
% The following commands set up the lecnum (lecture number)
% counter and make various numbering schemes work relative
% to the lecture number.
%
\newcounter{lecnum}
\renewcommand{\thepage}{\thelecnum-\arabic{page}}
\renewcommand{\thesection}{\thelecnum.\arabic{section}}
\renewcommand{\theequation}{\thelecnum.\arabic{equation}}
\renewcommand{\thefigure}{\thelecnum.\arabic{figure}}
\renewcommand{\thetable}{\thelecnum.\arabic{table}}

%
% The following macro is used to generate the header.
%
\newcommand{\lecture}[4]{
   \pagestyle{myheadings}
   \thispagestyle{plain}
   \newpage
   \setcounter{lecnum}{#1}
   \setcounter{page}{1}
   \noindent
   \begin{center}
   \framebox{
      \vbox{\vspace{2mm}
    \hbox to 6.28in { {\bf EE402 - Discrete Time Systems
	\hfill Spring 2018} }
       \vspace{4mm}
       \hbox to 6.28in { {\Large \hfill Lecture #1 \hfill} }
       \vspace{2mm}
       \hbox to 6.28in { {\it Lecturer: #2 \hfill } }
      \vspace{2mm}}
   }
   \end{center}
   \markboth{Lecture #1}{Lecture #1}

   \vspace*{4mm}
}
%
% Convention for citations is authors' initials followed by the year.
% For example, to cite a paper by Leighton and Maggs you would type
% \cite{LM89}, and to cite a paper by Strassen you would type \cite{S69}.
% (To avoid bibliography problems, for now we redefine the \cite command.)
% Also commands that create a suitable format for the reference list.
\renewcommand{\cite}[1]{[#1]}
\def\beginrefs{\begin{list}%
        {[\arabic{equation}]}{\usecounter{equation}
         \setlength{\leftmargin}{2.0truecm}\setlength{\labelsep}{0.4truecm}%
         \setlength{\labelwidth}{1.6truecm}}}
\def\endrefs{\end{list}}
\def\bibentry#1{\item[\hbox{[#1]}]}

%Use this command for a figure; it puts a figure in wherever you want it.
%usage: \fig{NUMBER}{SPACE-IN-INCHES}{CAPTION}
\newcommand{\fig}[3]{
			\vspace{#2}
			\begin{center}
			Figure \thelecnum.#1:~#3
			\end{center}
	}
% Use these for theorems, lemmas, proofs, etc.
\newtheorem{theorem}{Theorem}[lecnum]
\newtheorem{lemma}[theorem]{Lemma}
\newtheorem{proposition}[theorem]{Proposition}
\newtheorem{claim}[theorem]{Claim}
\newtheorem{corollary}[theorem]{Corollary}
\newtheorem{definition}[theorem]{Definition}
\newenvironment{proof}{{\bf Proof:}}{\hfill\rule{2mm}{2mm}}

% **** IF YOU WANT TO DEFINE ADDITIONAL MACROS FOR YOURSELF, PUT THEM HERE:

\begin{document}

% Lecture Details
\lecture{3}{Asst. Prof. M. Mert Ankarali}

\par

\section*{Difference Equations}

In discrete-time domain, we have difference equations that replaces
differential equations. We are mainly interested in LTI systems, that
are represented by linear constant coefficient 
difference equations. Let $x[k]$ and $y[k]$ be the input and output
respectively, then an LTI difference equation can be expressed as 
%
\begin{align*}
a_0  y[k] + a_{1}  y[k-1] + ... + a_N y[k-N] &= b_{0}  x[k] + ... +
                                               b_M x[k-M] \\
\sum\limits_{n=1}^{N} a_n y[k - n] &= \sum\limits_{n=1}^{M} b_n x[k - n]
\end{align*}
%
Unlike ODEs difference equations are very easy to solve
computationally or simulate in computer environment. 
Let's consider the following first-order difference equation
%
\begin{align*}
y[k] &= \frac{1}{2} y[k-1] + x[k] \ \ , x[k] = 0 \ \& \ y[k] = 0 , \
       \mathrm{for} k < 0 
\end{align*}
%
Let's ``simulate'' the difference equation for $x[k] = \delta[k]$.
%
\begin{align*}
y[0] &= \frac{1}{2} y[-1] + x[0] = 0 + 1 = 1\\
y[1] &= \frac{1}{2} y[0] + x[1] = \frac{1}{2} + 0 = \frac{1}{2} \\
y[2] &= \frac{1}{2} \frac{1}{2} = \frac{1}{4}  \\
y[3] &= \frac{1}{2} \frac{1}{4} = \frac{1}{8} \\
\vdots \\
y[k] &= \left( \frac{1}{2} \right)^k
\end{align*}
%
Now let's simulate for $x[k] = u[k]$
%
\begin{align*}
y[0] &=  0 + 1 = 1\\
y[1] &= \frac{1}{2} + 1 \\
y[2] &= \frac{1}{4} + \frac{1}{2} + 1 \\
y[3] &= \frac{1}{8} + \frac{1}{4} + \frac{1}{2}  + 1 \\
\vdots \\
y[k] &= \frac{1}{2^k} + \cdots + \frac{1}{2}  + 1 = 2 -  \left( \frac{1}{2} \right)^k
\end{align*}
%
This is a great method for ``simulating'' using a computational
approach, but in general it may be very hard to get a closed 
form expression.
%
The most basic solution method is solving the difference equation
directly in time domain by trying to find a ``basis'' for the solution
space similar to the operation in ODEs. We try sequences/signals
of the form $\lambda^k \ , \ k >0$ to find a solution form for the
homogeneous equation. Let's apply this method for the first-order
difference equation above
%
\begin{align*}
y[k] = \lambda^k \ \rightarrow y[k] - \frac{1}{2} y[k-1] = 0 \\
\lambda^k - \frac{\lambda^{k-1}}{2} = 0 \\
\lambda^{k-1} \left( \lambda - \frac{1}{2}  \right) = 0 \\
\lambda - \frac{1}{2} = 0
\end{align*}
%
Where the last equation is the characteristic equation of the difference
equation. Since the characteristic equation has one root only, we
obtain a solution of the form
%
\begin{align*}
y[k] = y_h[k] + y_p[k] = C \left( \frac{1}{2} \right)^k + y_p[k]
\end{align*}
%
Let's assume that for $x[k] = u[k]$ particular solution has the 
form $y_p[k] = A \ \mathrm{for} \ k > 0$ then
%
\begin{align*}
A &= \frac{1}{2} A + 1 \ \rightarrow A = 2
\end{align*}
%
Now let's find $C$ using the fact that $y[k] = 0$ for $k<0$
%
\begin{align*}
y[0] &= \frac{1}{2} y[-1] + x[0] \ \rightarrow \  y[0] = 1
\\
1 &= C \left( \frac{1}{2} \right)^0 + 2 \rightarrow  C = -1
\end{align*}
%
Then the solution can be written as
%
\begin{align*}
  y[k] &= -\left( \frac{1}{2} \right)^k + 2
\end{align*}
%
\textbf{Example 1.1} Find the general form of the homogeneous solution
for the following difference equation
%
\begin{align*}
  y[k] - 3 y[k-1] + 2 y[k-2] &= x[k]
\end{align*}

\textbf{Solution:}
%
\begin{align*}
&\lambda^2 - 3 \lambda + 2 = 0
\\
&\lambda_1 = 1 \ \& \ \lambda_2 = 2 
\\
&y[k] = C_1 + C_2 2^k \ , k > 0 
\end{align*}
%
\textbf{Example 1.2} Now let's assume that $y[k] = 0$ for $k<0$
and $x[k] = 3^k$, then find $y[k]$ for $k \geq 0$.

\textbf{Solution:} First let's find a particular solution. 
Let's assume that $y_p[k] = A 3^k$, then
%
\begin{align*}
A 3^k - 3 A 3^{k-1} + 2 A 3^{k-2} &= 3^k \ \rightarrow \ A = 9/2
\\
y_p[k] &= 4.5 \ 3^k
\end{align*}
%
Now let's try to find $C_1$ and $C_2$
%
\begin{align*}
&y[k] - 3 y[k-1] + 2 y[k-2] = x[k]
\\
&y[0] = x[0] \ \rightarrow \ C_1 + C_2 = -3.5 \\
&y[1] - 3 y[0] = x[1] \ \rightarrow \ C_1 + C_2 2 = -7.5
\\
&C_1 = 0.5 \ \& \ C_2 = -4
\\
&y[k] = 0.5 - 4 \, 2^k + 4.5 \, 3^k \ , k > 0 
\end{align*}

\textbf{What about repeated roots?} Possible mini project question


\textbf{Example 2} Find the general form of the homogeneous solution
for the following difference equation
%
\begin{align*}
  y[k] + 4 y[k-2] &= x[k]
\end{align*}

\textbf{Solution:}
%
\begin{align*}
&\lambda^2 + 4 = 0 \ \rightarrow \ \lambda_{1,2} = \pm 2 j
\\
&y[k] = C_1 (2j)^k + C_2 (-2j)^k = C_1 2^k e^{j k \pi/2 } + C_2 2^k e^{-j
k  \pi/2 }  
\\
&y[k] = \bar{C}_1 2^k \frac{ e^{j k \pi/2} + e^{-j k \pi / 2} }{2} +
  \bar{C}_2 2^k \frac{ e^{j k \pi / 2} - e^{j k \pi / 2} }{2 j}
\\
&y[k] = \bar{C}_1 2^k \cos(k \pi /2) + \bar{C}_2 2^k \sin(k \pi /2)
\end{align*}

\textbf{How we can generalize this to arbitrary complex conjugate
  roots?} Possible mini project question

\textbf{What is the home message?} Similar to ODEs time domain
solution of difference equations is generally ``messy''.

\section*{Z-transform \& Difference Equations}

\subsection*{Difference Equations to Z-transform}

Let's consider the following difference equation with $y[n]$
and $x[n]$ be the strictly causal input--output pair.
%
\begin{align*}
a_0  y[k] + a_{1}  y[k-1] + ... + a_N y[k-N] &= b_{0}  x[k] + ... +
                                               b_M x[k-M] 
\end{align*}
%
Now let's assume that $\mathcal{Z}\lbrace x[k] \rbrace = X(z)$ and
$\mathcal{Z}\lbrace y[k] \rbrace = Y(z)$. If we take the Z-transform
fo the both sides of the equation by applying the shifting theorem 
we obtain
%
\begin{align*}
a_0 Y(z) + a_{1} z^{-1}  Y(z) + ... + a_N z^{-N} Y(z) &= 
b_{0}  X(z) + ... + b_M z^{-M} X(z)
\\
( a_0 + a_{1} z^{-1} + ... + a_N z^{-N} ) Y(z) &= 
( b_0 + b_{1} z^{-1} + ... + b_M z^{-M} ) X(z)
\\
\frac{Y(z)}{X(z)} = G(z) &= 
\frac{b_0 + b_{1} z^{-1} + ... + b_M z^{-M}}{a_0 + a_{1} z^{-1} + ... + a_N z^{-N}}
\\ &= 
z^{N-M} 
\frac{b_0 z^{M} + b_{1} z^{M-1} + ... + b_M}{a_0 z^{N} + a_{1} z^{N-1} + ... + a_N}
\end{align*}
%
Under ``zero initial conditions'' if we can find $X(z)$, then we can
compute $Y(z)$ using $Y(z) = G(z) X(z)$. After that we can take the inverse z-transform 
and compute $y[k]$.

\textbf{Example 3.1} Compute $y[k]$ using the Z-transform method
%
\begin{align*}
y[k] &= \frac{1}{2} y[k-1] + x[k]  \\
y[k] &= 0 , \ \mathrm{for} \ k < 0  \ \& \ x[k] = \delta[k]
\end{align*}
%
\textbf{Solution:}

\begin{align*}
Y(z) &= \frac{1}{2} Y(z)z^{-1} + X(z) \  \rightarrow \
       \frac{Y(z)}{X(z)} = G(z) = \frac{z}{z - 1/2}  \\
Y(z) &= \frac{z}{z - 1/2} \ \rightarrow \ y[k] = \left( \frac{1}{2} \right)^k
\end{align*}

\textbf{Example 3.2} Now let's compute $y[k]$ for $x[k] = u[k]$
%
\begin{align*}
Y(z) &= G(z)  X(z) \  \rightarrow \ Y(z) = \frac{z^2}{(z - 1/2)(z-1)}
\\
Y(z) &= - \frac{z}{z - 1/2} + 2\frac{z}{z - 1}
\\
y[k] &= 2 - \left( \frac{1}{2} \right)^k
\end{align*}

\textbf{Example 4} For the following difference equation, compute
$y[k]$ for $x[k] = 3^k u[k]$
%
\begin{align*}
  y[k] - 3 y[k-1] + 2 y[k-2] &= x[k]
\end{align*}

\textbf{Solution:}
%
\begin{align*}
 Y(z) (1 - 3 z^{-1} + 2 z^{-2}) &= X(z) \ \rightarrow \ G(z) =
  \frac{z^2}{z^2 - 3 z + 2 } = \frac{z^2}{(z-1)(z-2)}
\\
Y(z) &= \frac{z^3}{(z-1)(z-2)(z-3)} = 0.5 \frac{z}{z-1} - 4 \frac{z}{z-2} +
     4.5  \frac{z}{z-3}   
\\
y[k] &= \left( 0.5 - 4 \ 2^k + 4.5 \ 3^k \right) u[k]
\end{align*}

\subsection*{Z-transform to Difference Equations}

Sometimes the Z-domain transfer function of a system is given, and 
we may be supposed to find the difference equation representation. 
Let's assume that we have a general transfer function that can be
represented in terms of ratio of two polynomials in $z$ or $z^{-1}$
as given below
%
\begin{align*}
\frac{Y(z)}{X(z)} = G(z) &= 
\frac{b_0 + b_{1} z^{-1} + ... + b_M z^{-M}}{a_0 + a_{1} z^{-1} + ... + a_N z^{-N}} = 
z^{N-M} 
\frac{b_0 z^{M} + b_{1} z^{M-1} + ... + b_M}{a_0 z^{N} + a_{1} z^{N-1} + ... + a_N}
\end{align*}
%
In his case, I prefer to work with the polynomials that are written in
terms of $z^{-1}$. Let's manipulate the Z-domain equation to obtain
%
\begin{align*}
Y(z) ( a_0 + a_{1} z^{-1} + ... + a_N z^{-N} ) &= X(z) ( b_0 + b_{1}
  z^{-1} + ... + b_M z^{-M} )
\\
a_0 Y(z) + a_{1} z^{-1} Y(z) + ... + a_N z^{-N} Y(z) &= b_0 X(z) + b_{1}
  z^{-1} X(z) + ... + b_M z^{-M} X(z)
\end{align*}
%
Let's assume that $\mathcal{Z}^{-1} \lbrace Y(z) \rbrace = y[k]$ 
and $\mathcal{Z}^{-1} \lbrace X(z) \rbrace = x[k]$. If we take the 
inverse Z-transform of both sides by applying the shifting theorem
we obtain
%
\begin{align*}
a_0 y[k] + a_{1} y[k-1] + ... + a_N y[k-N] &= b_0 x[k] + b_{1}
x[k-1] + ... + b_M x[k-M]
\end{align*}
%
We can use this conversion to ``simulate'' a given discrete time transfer
function or realizing the given system (it may be a filter or
controller) to implement on an embedded platform.  

It can also be used for computationally finding the
inverse Z-transform of a given z-domain rational function. 
The next example will illustrate this feature. 

\textbf{Example 5} Find a computational solution for the inverse Z-transform of $H(z) =
\frac{z^{-1}}{1 - 2 z^{-1} + z^{-2}}$ by using the conversion from
Z-domain transfer function to difference equation concept. 

\textbf{Solution:} Let's assume that $H(z)$ is a ``transfer function''
not an arbitrary z-domain function. Then $\mathcal{Z}^{-1} \lbrace
H(z) \rbrace = h[k]$ becomes the impulse response of the ``system''. 
Thus we can assume some imaginary input-output pair $y[n]$ and
$x[n]$ where 
%
\begin{align*}
  \frac{Y(z)}{X(z)} = H(z) 
\end{align*}
%
If we can find a difference equation realization for $H(z)$
then we can simulate the difference equation by assuming
$x[k] = \delta[k]$ (i.e. unit impulse input). So let's find a
realization for the given $H(z)$ as
%
\begin{align*}
  \frac{Y(z)}{X(z)} &= \frac{z^{-1}}{1 - 2 z^{-1} + z^{-2}}
\\
Y(z) - 2 z^{-1} Y(z) + z^{-2} Y(z) &= z^{-1} X(z)
\\
y[k] - 2 y[k-1] + y[k-2] &= x[k-1]
\end{align*}
%
Now let's simulate the above equation for $x[k] = \delta[k]$
%
\begin{align*}
y[k] &= 2 y[k-1] - y[k-2]  + x[k-1]
\\
y[0] &= 2 y[-1] - y[-2]  + x[-1] = 0
\\
y[1] &= 2 y[0] - y[-1]  + x[0] = 1
\\
y[2] &= 2 y[1] - y[0]  + x[1] = 2
\\
y[3] &= 2 y[2] - y[1]  + x[2] = 3
\\
y[4] &= 4
\\
\cdots 
\\
y[k] &= k
\end{align*}
%




% **** This ENDS THE EXAMPLES. DON'T DELETE THE FOLLOWING LINE:
\end{document}
