% 


\documentclass[twoside]{article}
\setlength{\oddsidemargin}{0.25 in}
\setlength{\evensidemargin}{-0.25 in}
\setlength{\topmargin}{-0.6 in}
\setlength{\textwidth}{6.5 in}
\setlength{\textheight}{8.5 in}
\setlength{\headsep}{0.75 in}
\setlength{\parindent}{0 in}
\setlength{\parskip}{0.1 in}

%
% ADD PACKAGES here:
%

\usepackage{amsmath,amsfonts,graphicx}


\newcounter{lecnum}
\renewcommand{\thepage}{\thelecnum-\arabic{page}}
\renewcommand{\thesection}{\thelecnum.\arabic{section}}
\renewcommand{\theequation}{\thelecnum.\arabic{equation}}
\renewcommand{\thefigure}{\thelecnum.\arabic{figure}}
\renewcommand{\thetable}{\thelecnum.\arabic{table}}

%
% The following macro is used to generate the header.
%
\newcommand{\lecture}[4]{
   \pagestyle{myheadings}
   \thispagestyle{plain}
   \newpage
   \setcounter{lecnum}{#1}
   \setcounter{page}{1}
   \noindent
   \begin{center}
   \framebox{
      \vbox{\vspace{2mm}
    \hbox to 6.28in { {\bf EE402 - Discrete Time Systems
	\hfill Spring 2018} }
       \vspace{4mm}
       \hbox to 6.28in { {\Large \hfill Lecture #1 \hfill} }
       \vspace{2mm}
       \hbox to 6.28in { {\it Lecturer: #2 \hfill } }
      \vspace{2mm}}
   }
   \end{center}
   \markboth{Lecture #1}{Lecture #1}

   \vspace*{4mm}
}

\renewcommand{\cite}[1]{[#1]}
\def\beginrefs{\begin{list}%
        {[\arabic{equation}]}{\usecounter{equation}
         \setlength{\leftmargin}{2.0truecm}\setlength{\labelsep}{0.4truecm}%
         \setlength{\labelwidth}{1.6truecm}}}
\def\endrefs{\end{list}}
\def\bibentry#1{\item[\hbox{[#1]}]}


\newcommand{\fig}[3]{
			\vspace{#2}
			\begin{center}
			Figure \thelecnum.#1:~#3
			\end{center}
	}

% Use these for theorems, lemmas, proofs, etc.
\newtheorem{theorem}{Theorem}[lecnum]
\newtheorem{lemma}[theorem]{Lemma}
\newtheorem{proposition}[theorem]{Proposition}
\newtheorem{claim}[theorem]{Claim}
\newtheorem{corollary}[theorem]{Corollary}
\newtheorem{definition}[theorem]{Definition}
\newenvironment{proof}{{\bf Proof:}}{\hfill\rule{2mm}{2mm}}

% **** IF YOU WANT TO DEFINE ADDITIONAL MACROS FOR YOURSELF, PUT THEM HERE:

\begin{document}

% Lecture Details
\lecture{15}{Asst. Prof. M. Mert Ankarali}

\par

\section*{Reachability/Controllability, \& Observability}

\subsection*{Reachability \& Controllability of CT Systems}

For an LTI continuous time state-space representation
%
\begin{align*}
  \dot{x}(t) &= A x(t) + B u(t)
\\
  y(t) &= C x(t) + D u(t)
\end{align*}
%
\begin{itemize}
  \item A state $x_d$ is said to be \textbf{reachable}
  if there exist a finite time interval $t \in [0 ,t_f]$ and
  an input signal defined on this interval, $u(t)$, that transfers the state vector 
  $x(t)$ from the origin (i.e. $x(0) = 0$) to the state $x_d$ within
  this time interval, i.e. $x(t_f) = x_d$.
 
  \item A state $x_d$ is said ti be \textbf{controllable}
  if there exist a finite time interval $t \in [0 ,t_f]$ and
  an input signal defined on this interval, $u(t)$, that transfers the state vector 
  $x(t)$ from the initial state $x_d$ (i.e. $x(0) = x_d$) to the
  origin within this time interval, i.e. $x(t_f) = 0$.  
\end{itemize}

\begin{itemize}
  \item The set $\mathcal{R}$ of all reachable states is a linear
(sub)space: $\mathcal{R} \subset \mathbb{R}^n$
  \item The set $\mathcal{C}$ of all controllable states is a linear
(sub)space: $\mathcal{C} \subset \mathbb{R}^n$
\end{itemize}

For CT systems $x_d \in \mathcal{R}$ if and only if  $x_d \in
\mathcal{C}$, the Reachability and Controllability conditions are 
equivalent. 

\begin{itemize}
  \item If the reachable (or controllable) set is the entire state
    space, i.e., if $\mathcal{R} = \mathbb{R}^n$, then the system is
    called reachable (or controllable).
\end{itemize}

One way of testing reachability/controllability is checking the rank
(or the range space) the of reachability/controllability matrix
%
\begin{align*}
  \mathbf{M} = \left[ \begin{array}{cccc} B & A B & \cdots & A^{n-1} B \end{array} \right]
\end{align*} 
%

A CT system is reachable/controllable if and only of
%
\begin{align*}
  \mathrm{rank} ( \mathbf{M} ) &= n
\end{align*} 
%
or equivalently 
%
\begin{align*}
 \mathrm{Ra} ( \mathbf{M} ) = \mathbb{R}^n
\end{align*} 

%%%%%%%%%%%%%%%%%%%%%%%%%%

\subsection*{Reachability \& Controllability of DT Systems}

For LTI a discrete time state-space representation
%
\begin{align*}
  x[k+1] &= A x[k] + B u[k]
\\
  y[k] &= C x[k] + D u[k]
\end{align*}
%
\begin{itemize}
  \item A state $x_d$ is said to be \textbf{reachable}, if there exist
  an input sequence, $u[k]$, that transfers the state vector 
  $x[k]$ from the origin (i.e. $x[0] = 0$) to the state $x_d$ in finite
  number of steps, i.e. $x[k] = x_d$ for some $k \in \mathbb{Z}^+$.
 
  \item A state $x_d$ is said ti be \textbf{controllable},
  if there exist an input sequence, $u[k]$, that transfers the state vector 
  $x[k]$ from the initial state $x_d$ (i.e. $x[0] = x_d$) to the origin
  in finite number of steps, i.e. $x[k] = 0$ for some $k \in \mathbb{Z}^+$ 
\end{itemize}

\begin{itemize}
  \item The set $\mathcal{R}$ of all reachable states is a linear
(sub)space: $\mathcal{R} \subset \mathbb{R}^n$
  \item The set $\mathcal{C}$ of all controllable states is a linear
(sub)space: $\mathcal{C} \subset \mathbb{R}^n$
\end{itemize}

Unlike from CT systems the Reachability and Controllability conditions 
are not equivalent. 

\begin{itemize}
  \item $x_d \in \mathcal{R} \Rightarrow x_d \in \mathcal{C}$
  \item $x_d \in \mathcal{C} \not\Rightarrow  x_d \in \mathcal{R}$
  \item $\mathcal{R} \subset \mathcal{C}$
\end{itemize}

Thus Reachability implies Controllability but Controllability does not
necessarily implies Reachability. For this reason, the term of Reachability 
is generally preferred for DT systems.

\begin{itemize}
  \item If the reachable set is the entire state
    space, i.e., if $\mathcal{R} = \mathbb{R}^n$, then the system is
    called Reachable (and automatically Controllable).
  \item If the controllable set is the entire state
    space, i.e., if $\mathcal{C} = \mathbb{R}^n$, then the system is
    called Controllable. But there is no guarantee for Reachability.  
\end{itemize}

\textbf{Example:} Consider the following autonomous system
%
\begin{align*}
  x[k+1] &= \left[ \begin{array}{cc} 0 & 1 \\ 0 & 0 \end{array} \right] x[k] 
\end{align*}
%
What can we infer about the Reachability and Controllability of this
system.

\textbf{Solution:} Since this is an autonomous system, obviously 
%
\begin{align*}
 B &= \left[ \begin{array}{c} 0 \\ 0 \end{array} \right] 
\end{align*}
%
Thus input has no affect on the states. If $x[0] = 0$, then
$x[k] = 0 , \ \forall k > 0$. Thus the system is obviously 
NOT Reachable. 

Now let's compute $x[2]$ for a general $x[0] = x_0$,
 %
\begin{align*}
 x[2] &= \left[ \begin{array}{cc} 0 & 1 \\ 0 & 0 \end{array} \right]^2 x_0
= \left[ \begin{array}{cc} 0 & 0 \\ 0 & 0 \end{array} \right] x_0
= \left[ \begin{array}{c} 0 \\ 0 \end{array} \right]  
\end{align*}
%
Obviously $\forall x_0 \in \mathbb{R}^n \, x[2] = 0$, thus all
state-space is Controllable. 

\subsubsection*{Test of Reachability on DT Systems}

When $x[0] = 0$, the solution of $x[k]$ is given by

\begin{align*}
  x[k] &= \sum\limits_{j = 0}^{k-1} G^{k-j-1} H u[j]
         \\
&=
         \left[ \begin{array}{c|c|c|c|c} G^{k-1} H & G^{k-2} H &
         \cdots & G H & H \end{array} \right]
         \left[ \begin{array}{c}
                  u[0] \\ u[1] \\ \vdots \\ u[k-2] \\ u[k-1]
         \end{array} \right]
\end{align*}

Let 
\begin{align*}
 \mathbf{M}_k &= 
         \left[ \begin{array}{c|c|c|c|c} G^{k-1} H & G^{k-2} H &
         \cdots & G H & H \end{array} \right]
\\
 \mathbf{U}_k &=
         \left[ \begin{array}{c}
                  u[0] \\ u[1] \\ \vdots \\ u[k-2] \\ u[k-1]
         \end{array} \right]
\end{align*}
%
then if a state $x_d$ is reachable at $k$ steps, it should 
satisfy the following equation for some $\mathbf{U}_k$.
%
\begin{align*}
 \mathbf{M}_k \mathbf{U}_k &= x_d
\end{align*}
%
In order this matrix equation to have a solution $x_d$
should be in the range space of $\mathbf{M}_k$.
%
\begin{align*}
  x_d \in \mathrm{Ra} ( \mathbf{M}_k ) 
\end{align*}
%
It is fairly easy to see that 
%
\begin{align*}
  \mathrm{Ra} ( \mathbf{M}_k ) \subset \mathrm{Ra} ( \mathbf{M}_{k+1} )
\end{align*}
%
Thus increasing $k$ increases the chance of $x_d$ being in the
reachable subset. 

\textbf{Theorem:} For $k < n < l$
%
\begin{align*}
  \mathrm{Ra} ( \mathbf{M}_k ) \subset \mathrm{Ra} ( \mathbf{M}_{n} )
= \mathrm{Ra} ( \mathbf{M}_{l} )
\end{align*}
%
\textbf{Proof:} In order to prove this Theorem, we need to use a
different well-known theorem. 

\textbf{Cayley-Hamilton Theorem} states that every square matrix
satisfies its own characteristic equation. In other words, Let 
$A \in \mathbb{R}^{n \times n}$, and let $p(\lambda)$ be the
characteristic equation defined as
%
\begin{align*}
 p(\lambda) &= \mathrm{det} \left( \lambda I - A \right) 
\\
&= \lambda^n + a_1 \lambda^{n-1} + \cdots + a_{n-1} \lambda + a_n \lambda
\end{align*}
%
Then by Cayley-Hamilton theorem we conclude that
%
\begin{align*}
 p(G) &= G^n + a_1 G^{n-1} + \cdots + a_{n-1} G + a_n I = 0
\end{align*}
%
Using this we can see easily that 
%
\begin{align*}
 G^n B = -a_1 G^{n-1} B - \cdots - a_{n-1} G B - a_n I
\end{align*}
%
Now lets observe $M_{n+1}$
%
\begin{align*}
\mathbf{M}_{n+1} &= 
         \left[ \begin{array}{c|c|c|c|c} G^{n} H & G^{n-1} H &
         \cdots & G H & H \end{array} \right]
\end{align*}
% 
If we follow the Cayley-Hamilton theorem and associated derivations,
we can see that the first column $G^{n} H$ is a linear combination of
other columns, thus it can not increase the rank of the matrix. 

This the reachability matrix is defined as
%
\begin{align*}
 \mathbf{M} &= 
         \left[ \begin{array}{c|c|c|c|c} G^{n-1} H & G^{n-2} H &
         \cdots & G H & H \end{array} \right]
\end{align*}
%
where $n$ is the dimension of the state-space.

The DT system is called reachable if 
\begin{align*}
 \mathrm{rank} (\mathbf{M}) &= n 
\end{align*}
or equivalently
\begin{align*}
 \mathrm{Ra} (\mathbf{M}) &= \mathbb{R}^n
\end{align*}

\section*{Observability}

It turns out that it is more natural to think in terms of
``un-observability'' as reflected in the following definitions.

\begin{itemize}
  \item For CT systems, a state $x_o$ of a finite dimensional linear dynamical
    system is said to be unobservable, if with $x(0) = x_o$ and
    for every $u(t)$ we get the same $y(t)$ as we would with $x(0) =
    0$. 

  \item For DT systems, a state $x_o$ of a finite dimensional linear dynamical
    system is said to be unobservable, if with $x[0] = x_o$ and
    for every $u[k]$ we get the same $y[k]$ as we would with $x[0] =
    0$. 
\end{itemize}

    In other words, for both CT and DT systems an unobservable initial condition cannot be
    distinguished from the zero initial condition.

  The set $\bar{\mathcal{O}}$ of all unobservable states is a linear
  (sub)space: $\bar{\mathcal{O}} \subset \mathbb{R}^n$

\begin{itemize}
%
 \item If the unobservable set only contains the origin, 
   i.e., if $\bar{\mathcal{O}} = \lbrace 0 \rbrace$, 
%
 \item If the dimension of unobservable subspace is equal to 0,
   $\mathrm{dim} = \left( \bar{\mathcal{O}} \right) = 0$,
  %
 \item If any initial condition, $x(0)$ or $x[0]$, can be uniquely 
 determined from input-output measurement,
%
\end{itemize} 
then the system is called Observable. 

\subsubsection*{Test of Observability on CT Systems}

One way of testing Observability of CT systems is checking the rank
(or the range space, or null space) the of the Observability matrix
%
\begin{align*}
  \mathbf{O} =
\left[ \begin{array}{c}
C 
\\
C A
\\
C A^2 
\\
  \vdots
\\
C A^{n-1}
\end{array}
\right] 
\end{align*} 
%

A CT system is Observable if and only of
%
\begin{align*}
  \mathrm{rank} ( \mathbf{O} ) &= n
\end{align*} 
%
or equivalently 
%
\begin{align*}
 \mathrm{Ra} ( \mathbf{O} ) = \mathbb{R}^n
\end{align*} 
%
or equivalently 
%
\begin{align*}
 \mathrm{dim} \left ( \mathcal{N} ( \mathbf{O} ) \right) = 0
\end{align*} 

\subsubsection*{Test of Observability on DT Systems}

Without loss of generality, let's assume that $u[k] =0$.
Under this assumption, we know that 
%
\begin{align*}
  y[k] &= C G^k x_0
\end{align*}
%
Based on this solution we can write
%
\begin{align*}
  y[0] &= C x_0
\\
  y[1] &= C G x_0
\\
  y[2] &= C G^2 x_0
\\
  \vdots
\\
 y[k] &= C G^k x_0
\end{align*}
%
If we combine these equations matrix form we obtain
%
\begin{align*}
\left[ \begin{array}{c}
  y[0] 
\\
  y[1] 
\\
  y[2] 
\\
  \vdots
\\
 y[k] 
\end{array}
\right]
=
\left[ \begin{array}{c}
C 
\\
C G 
\\
C G^2 
\\
  \vdots
\\
C G^k
\end{array}
\right]
x_0
\end{align*}
%
Let 
%
\begin{align*}
\mathbf{Y}_k =
\left[ \begin{array}{c}
  y[0] 
\\
  y[1] 
\\
  y[2] 
\\
  \vdots
\\
 y[k] 
\end{array}
\right]
\quad , \quad
\mathbf{O}_k =
\left[ \begin{array}{c}
C 
\\
C G 
\\
C G^2 
\\
  \vdots
\\
C G^k
\end{array}
\right]
\end{align*}
%
Then the equation takes the simple form $\mathbf{Y}_k = \mathbf{O}_k
x_0$. If $x_0$ is an unobservable state, then for-all $k$ we should have
$\mathbf{O}_k x_0 = 0$, or equivalently $x_0 \in \mathcal{N} \left(
  \mathbf{O}_k \right)$ (Null-space). 

From this point, we can conclude that, the DT system is observable if
and only if, 
%
\begin{align*}
  \forall k \in \mathbb{Z}, \mathrm{dim} \left( \mathcal{N} \left(
  \mathbf{O}_k \right) \right) = 0
\end{align*}
%
However we don't need to test all $k \in \mathbb{Z}$. First of all
it should be obvious that we should take $k$ as large as possible
to guarantee weather $x_0$ is unobservable ot not. Formally speaking,
%
\begin{align*}
 \mathcal{N} \left( \mathbf{O}_{k+1} \right) \subset \mathcal{N} \left( \mathbf{O}_{k} \right)
\end{align*}
%
However from Cayley-Hamilton theorem, we know that $C A^n$ can be
written as a linear combination of $\lbrace C A^{n-1} , \ C A^{n-2} ,
\ \cdots , C A , \ C\rbrace$, thus we have 
%
%
\begin{align*}
 \mathcal{N} \left( \mathbf{O}_{n} \right) = \mathcal{N} \left( \mathbf{O}_{n-1} \right)
\end{align*}
%
For this reason it is necessary and sufficient to test $\mathbf{O}_{n-1}$
for observability. In conclusion, observability matrix is defined as
%
\begin{align*}
  \mathbf{O} =
\left[ \begin{array}{c}
C 
\\
C G 
\\
C G^2 
\\
  \vdots
\\
C G^{n-1}
\end{array}
\right] 
\end{align*}

The DT system is called Observable if
%
\begin{align*}
  \mathrm{rank} \left( \mathbf{O} \right) = n
\end{align*}
%
or equivalently 
%
\begin{align*}
  \mathrm{Ra} \left( \mathbf{O} \right) = \mathbb{R}^n
\end{align*}
%
\begin{align*}
 \mathrm{dim} \left ( \mathcal{N} ( \mathbf{O} ) \right) = 0
\end{align*} 

% **** This ENDS THE EXAMPLES. DON'T DELETE THE FOLLOWING LINE:
\end{document}